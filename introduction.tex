
% Deleted references: \citep{Levin2009}, \citet{Schurr2012}, \citep{Canham2004, Canham2006}

\section{Introduction}

A common, but rarely tested, assumption in ecology is that a species is more likely to be found where it performs the best. In other words, a species probability of occurrence across its range should be positively correlated to its per capita intrinsic growth rate \citep{McGill2012}. This hypothesis stems from the interpretation of Hutchinsonian niche theory \citep{Hutchinson1957, Maguire1973}, which posits that species are limited to locations where the environmental conditions (\ie any property outside of the organisms being) allow a population to persist. At the core of species distribution models, this hypothesis is used to identify the climatic variables that constrain species ranges, and their projection into the future allows potential range shifts to be forecasted. \\

This niche theory, in its most concise formulation, relates the population growth rate $ r $ to the species' niche: the hypervolume in the environmental factors space is the set, such that $ r \geqslant 0 $ \citep{Holt2009, Godsoe2017}. Formally, let $ r_i(\bm{E}, \bm{R}) $ be the growth rate of a focal species $ i $ when rare, namely, the intrinsic growth rate for a given environment $ \bm{E} $ and quantity of resources $ \bm{R} $. The equation
\[
	r_i(\bm{E}, \bm{R}) \geqslant 0,
\]
specifies that the fundamental niche corresponds to the locations where $ \bm{E} $ and $ \bm{R} $ allow positive growth. An equivalent representation is the lifetime number of recruits per individual, which is traditionally denoted by $ R_{0, i} $ \citep[where $ i $ is still the species index]{Pulliam2000, DeRoos1997}. A sustainable population requires that $ R_{0, i} (\bm{E}, \bm{R}) \geqslant 1 $. That is to say, on average an individual needs to at least replace itself over its lifespan. This definition of the niche allows the species $ i $ to exert an influence on the rates of other species of the community, while receiving feedbacks on its own demographic rates. Hereafter, we drop the index $ i $, but it is important to remember that $ R_{0, i} $ is a species-specific rate related to a species $ i $. \\

Difficulties in testing the niche theory are partially rooted in the challenge of measuring population growth rates \citep{McGill2012}. This is especially true for physiologically structured populations. For instance, the age or the size of an individual influences its reproduction success, feeding behaviour, or death probabilities. These age and size structures, when combined with density-dependences and environmental influences, might blur the relationship between a species' occurrence and its $ R_0 $ variations along an environmental gradient. This is where individual-based structured-population models can bring insight into addressing the complexity of population dynamics. Such models, however, are demanding in terms of their parameterisation, and are difficult to analyse. Good examples are forest trees, which can be modelled by spatially explicit simulators, which account for single-tree development and availability of light to an individual. These forest simulators focus upon the individual level, which is the relevant scale for studying competition and climate responses, but usually interesting questions relating to biogeography (including tree species distributions) lie at the population level, that is to say, $ R_0 $. \\

The relationship between tree species ranges and population growth has recently come under scrutiny, both in Europe \citep{Thuiller2014} and in north-eastern North America \citep{McGill2012}. Little correspondence has been found between $ r $ and tree species distributions due to uncertainty associated with demographic parameters \citep{Thuiller2014}. Negative correlations were even found, which at first glance, challenges the common assumption that a species is most abundant in its optimal environment. The inclusive niche was proposed as an alternative explanation to these negative correlations \citep{McGill2012}. This explanation states that weak competitors have their fundamental niche reduced to a smaller realised space, in a trade-off between competitive ability and environmental tolerance \citep{Serrano2015}. Therefore, weak competitors can be more abundant in suboptimal environments. \\

Dynamic Range Models (DRMs) have been developed recently to get closer to individual demographic rates, while scaling up to the population level. DRMs are hierarchical statistical models that relate species' abundance to environmental data using two latent variables: the population growth rate $ r $, and dispersal. These models account for uncertainties in the data as well \citep{Pagel2012}. Therefore, DRMs allowed uncertainties arising from the demographic level, the biogeographic level, and the observer level to be tracked. \\

Currently, no standard method exists that would derive a single performance index from demographic rates \citep{Purves2009}. Thus, all studies linking species distributions to individual performance lie in the midst of `uncharted territory'. The aforementioned studies of \citeauthor{McGill2012}, \citeauthor{Thuiller2014}, or \citeauthor{Pagel2012} have all explored different ways of linking distributions to population performance. They have all agreed that combining the three vital rates, namely, individual growth, mortality, and fecundity into $ r $ is difficult, and that $ r $ itself cannot be easily derived from censuses. Building upon these manuscripts, we will derive $ r $ from a forest dynamics model that uses the three vital rates. In this paper, we focus exclusively on radial growth and mortality, and make them life-stage dependent, which is of primary importance in propagating uncertainties up to $ r $ \citep[$ \lambda $ in his article]{Clark2003b}. \\

Our main objective in this study is to investigate whether the distribution of North American tree species is driven by the effect of climate and light competition on individual demography. The Abundant-Centre Hypothesis postulates that demographic performance should decline toward range margins. We therefore investigate two predictions that (\textit{i}) per capita growth rate should vary with climate and as a result, (\textit{ii}) per capita growth rate should decline at range margins where occurrence probabilities tend to zero. We represent forest stand dynamics with a cohort-based model that relies on McKendrick-von Foerster equations \citep{Strigul2008}. This model relates individual tree demography to cohort dynamics, thereby accounting for neighbourhood competition and ontogenic variation in demography. We derive a formula from the McKendrick-von Foerster equations to combine individual tree growth, mortality and fecundity rates into $ R_0 $, the per capita growth rate. We then evaluate how components of tree demography (individual growth and mortality) respond to climate, individual size, and competition. We expect an optimal climate for each species within the middle of its range, whith better performance under light than shade conditions. Thereafter, we test if $ R_0 $ is positively correlated to the probability of occurrence and if it declines towards range limits. The analysis is performed for the 14 most abundant tree species that are native to eastern North America.

