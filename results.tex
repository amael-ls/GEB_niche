\section{Results}

\subsection{Competition and climate effects on individual growth and mortality}

We found that individual growth and mortality are weakly related to climate, and that individual tree size and competition are predominant drivers of their variation (Fig. \ref{fig::aic_waic}).

Individual tree growth is, regardless of species, best explained by annual mean temperature and annual precipitation, together with the plot random effect, and year within plot. The conditional $ R_c^2 $ was much higher than the marginal $ R_m^2 $, indicating that the random structure and more specifically, that local conditions, dominates explanatory variables. According to the calculated ratio $ \varphi $ (equation \ref{eq::ratio}) and $ R^2 $, climate explains little variation in individual tree growth compared to tree \textit{dbh} (Fig. \ref{fig::aic_waic}; Supporting Information Appendix S3).

Our analysis revealed that growth is higher in the overstorey for all species (Fig. \ref{fig::over_under}). Moreover, we found that the response to overstorey competition corresponds to the shade tolerance: individual growth of shade-tolerant species is less responsive to canopy gaps than shade-intolerant species (Fig. \ref{fig::groups}; Table S3.6 in Supporting Information S3).
\begin{figure}
\begin{subfigure}[t]{.48\textwidth}
	\centering
	\input{graphs/G_over-under-storey_averaged}
	\caption{Overstorey versus understorey growth}
	\label{fig::over_under}
\end{subfigure}
\hfill
\begin{subfigure}[t]{.48\textwidth}
	\centering
	\input{graphs/groups}
	\caption{Response of radial growth to light}
	\label{fig::groups}
\end{subfigure}
\medskip
%% Second row
\begin{subfigure}[t]{.48\textwidth}
	\centering
	\input{graphs/M_over-under-storey_averaged}
	\caption{Overstorey versus understorey mortality}
	\label{fig::over_under_mu}
\end{subfigure}
\hfill
\begin{subfigure}[t]{.48\textwidth}
	\centering
	\input{graphs/groups_mortality}
	\caption{Response of mortality to light}
	\label{fig::groups_mu}
\end{subfigure}
\caption{\subref{fig::over_under} \& \subref{fig::over_under_mu} Overstorey versus understorey growth and mortality of the 14 parameterised species for an averaged individual (\ie all the explanatory variables of equations \eqref{eq::glmm_growth} and \eqref{eq::glmm_mortality} are set to their average values). The line is the identity function. \subref{fig::groups} \& \subref{fig::groups_mu} Response of species-specific radial growth $ G $ and mortality $ \mu $ to light, grouped by three levels of shade tolerance: Low (L), Medium (M), and High (H). Growth of species reaching the canopy increases much more rapidly for shade-intolerant species than tolerant ones, and the mortality decreases much more strongly for shade-intolerant species than tolerant. See Supporting Information S3 for the parameters.}
\label{fig::growthMortalityResults}
\end{figure}

For mortality, the best model for all the species included the lowest annual temperature ($ T_m $) and the three contiguous driest months ($ P_d $). Diameter at breast height, competition, and climate best explained tree mortality when combined, but are equivalent when taken separately, which differs from growth (Fig. \ref{fig::aic_waic}). All the Markov chains converged, regardless of parameters and species (R-hat histogram in Supporting Information Appendix S4). 
\begin{figure}
	\centering
	\input{graphs/aic_waic}
	\caption{Performance $ \varphi $ (growth, \CircSteel, equation \ref{eq::ratio}) and $ \psi $ (mortality, \MoveUp, equation \ref{eq::psi}) for four models (the closer to zero, the better). The black symbols represent the 14 species, and the coloured symbols are the average for each model. The model `climate' is the 2\textsuperscript{nd}-order polynomial containing $ T_a $ and $ P_a $ (growth), or $ T_m $ and $ P_d $ (mortality). The model `competition' includes only canopy status as a predictor, and the model `dbh' is the 2\textsuperscript{nd}-order polynomial containing \textit{dbh}. The best model (written in a R-language style) for growth is $ \text{growth} \sim 1 + (1 | x) + (1 | t) + (1 | x:t) + (cs + dbh + dbh^2) * (T_a + T_a^2 + P_a + P_a^2) $. Mortality is best explained by $ \text{mortality} \sim 1 + \text{offset}\big( \log(\Delta t) \big) + cs*(T_m + T_m^2 + P_d + P_d^2) + dbh + dbh^2 $. All the tested models can be found in Supporting Information Appendix S3 (equations G2 and M7 are the best models for growth and mortality, respectively). All species have the same best model; thus, the black dots are hidden by the two coloured dots for the last column of this figure. \label{fig::aic_waic}}
\end{figure}
Mortality is greater in the understorey for all species, except for \textit{Tsuga canadensis} and \textit{Fagus grandifolia}. Low shade-tolerant species responded more negatively to competition than did highly tolerant species (Fig. \ref{fig::over_under_mu} and \ref{fig::groups_mu}).

There is considerable uncertainty in parameter estimates for both rates (Supporting Information Appendix S3). More specifically, the mortality functions of \textit{Abies balsamea, Betula papyrifera, Fagus grandifolia, Picea rubens, Pinus strobus} and \textit{Populus tremuloides} are bell-shaped or flat curves, but they could be U-shaped responses (which is more likely to be expected; \citet{Lines2010}) according to the posterior distributions of their regression coefficients. \textit{Abies balsamea} and \textit{Populus tremuloides} are the two exceptions, given that they compensate for their negative mortality response to \textit{dbh}\textsuperscript{2}, with steep slopes related to \textit{dbh}. The effect of competition is significant for most species and interacts with temperature, whilst interactions between competition and precipitation are mostly non-significant for both vital rates (Supporting Information Figures S4, and csv files S7 and S8). Thus, the advantage of having greater access to water resources for large trees might be compensated by their higher moisture demands.

\subsection{Relationship of demography and competition to tree distribution}
The $ \tilde \rho_0 $ equation \eqref{eq::rho0} relates species-specific demographic functions that were estimated in the previous section to population performance. Demographic performance $ \tilde \rho_0 $ varies significantly, but there is no systematic relationship across ranges (Figs. \ref{fig::mainResults}, and maps in Supporting Information Appendix S5).
\begin{figure}
	\centering
	\begin{tikzpicture}[colorbar arrow/.style={
			shape=single arrow,
			double arrow head extend=0.125cm,
			shape border rotate=90,
			minimum height=8cm,
			shading=#1
		}]
		% --- Map R0 (middle figure)
		\node[inner sep=0pt] (acerub) at (0,0)
			{\includegraphics[scale=0.25]{graphs/28728-ACE-RUB_R0_h=10m_scaled}};
		\node[below = 0.01 of acerub] (labelR0)
			{\textbf{Population growth rate} $ \bm{\tilde \rho_0 \in [0, 1]} $};
		
		\draw[white] (6,-3.9) -- (6,3.9) node [midway, above = 0.25cm, sloped, text = black] {Growth, mortality, $ \tilde \rho_0 $};
		\node[colorbar arrow=RmapShading] at (6,0) {};
		\node[anchor = west] (low) at (6.5,-3.9) {Low};
		\node[anchor = west] (high) at (6.5,3.9) {High};

		% --- Maps demography (top figure)
		% Top left (growth)
		\node[inner sep=0pt, above = 0.5cm of acerub, anchor = south east] (growth)
			{\includegraphics[scale=0.25]{graphs/28728-ACE-RUB_averageG_h=10m}};
		\node[below = 0.01 of growth] {\textbf{Growth} $ \bm{\bar G \in [2, 5]} $};
		
		% Top right (mortality)
		\node[inner sep=0pt, above = 0.5cm of acerub, anchor = south west] (mortality)
			{\includegraphics[scale=0.25]{graphs/28728-ACE-RUB_averageM_h=10m}};
		\node[below = 0.01 of mortality] {\textbf{Mortality} $ \bm{\bar \mu \in [0, 1]} $};

		% --- Gradient in the geographical space
		\node[left = 0.5cm of labelR0] (labelGrad)
			{$ \bm{\tilde \rho_0} $ \textbf{increasing directions}};
		\node[above = 0.01cm of labelGrad, inner sep=0pt, anchor = south] (grad)
			{\input{graphs/azimuth_ACE-RUB}};

		% Correlation text
		\node[above = 0.01cm of grad] {$ \text{Cor}(\tilde \rho_0, P_{\text{occ}}) = -0.5 $};

		% --- Arrows
		\draw[arrow] (growth) -- (acerub);
		\draw[arrow] (mortality) -- (acerub);
	\end{tikzpicture}
	\caption{Example results for \textit{Acer rubrum}, with a canopy height $ \s_c = 10 $ m, and with 2006--2010 climate data. Population growth rate $ \tilde \rho_0 $ results from $ G $, $ \mu $ (averaged on the maps) and constant fecundity. For many species, including \textit{Acer rubrum}, $ \tilde \rho_0 $ does not increase towards the centre of the distribution (green square). If the Abundant-Centre Hypothesis were validated, then the red arrows on the map (or equivalently, the blue and orange arrows on the left-hand side) would point towards the green square. These two arrows are the respective northern and southern averages of the vector field $ \nabla \tilde \rho_0 $ that is depicted by the white arrows on the map of $ \tilde \rho_0 $. The correlation between $ \tilde \rho_0 $ and the probability of occurrence $ P_{\text{occ}}$ is $ -0.5 $, which is the smallest correlation for all species. Overall, we found no rule relating occurrence probabilities to demography (see Fig. \ref{fig::3correls}). The green line delimits \textit{Acer rubrum}'s range (see Supporting Information Table S5.3). \label{fig::mainResults}}
\end{figure}

If the Abundant-Centre Hypothesis is to be confirmed, then vector fields $ \nabla \tilde \rho_0 $ (white and red arrows on the maps and Fig. \ref{fig::mainResults}) should point towards species' centroids. On one hand, $ \tilde \rho_0 $ of six species (\textit{Acer saccharum, Acer rubrum, Picea glauca, Pinus strobus, Thuja occidentalis}, and \textit{Tsuga canadensis}) increases from south-east to north-west (Fig. \ref{fig::grad_cols}). On the other hand, it increases towards the south for five species (\textit{Abies balsamea, Betula papyrifera, Picea mariana, Picea rubens}, and \textit{Populus tremuloides}). Except for \textit{Picea mariana}, note that the four other species have a negative slope associated with the quadratic \textit{dbh} term in the mortality model (parameter $ \beta_j^{(7)} $ in equation \eqref{eq::glmm_mortality}). The three remaining species (\textit{Betula alleghaniensis, Fagus grandifolia}, and \textit{Picea banksiana}) do no exhibit a clear direction within their respective distribution $ \Omega $. Correlations between $ \tilde \rho_0 $ and the orthodromic distance to the closest edge of $ \Omega $ are negative for most species (Supporting Information Appendix S6), which corroborates that $ \tilde \rho_0 $ is not highest in the centre of species distributions, thereby challenging the Abundant-Centre Hypothesis (Fig. \ref{fig::grad_cols}).
\begin{figure}[htb]
    \centering
	%% First row
	\begin{subfigure}{0.3\textwidth}
		\centering
		\input{graphs/azimuth_ABI-BAL_10}
		\caption{\textit{Abies balsamea} (0.48)}
		\label{fig::abibal_az}
	\end{subfigure}
	\hfil
	\begin{subfigure}{0.3\textwidth}
		\centering
		\input{graphs/azimuth_ACE-RUB_10}
		\caption{\textit{Acer rubrum} (-0.5)}
		\label{fig::acerub_az}
	\end{subfigure}
	\hfil
	\begin{subfigure}{0.3\textwidth}
		\centering
		\input{graphs/azimuth_ACE-SAC_10}
		\caption{\textit{Acer saccharum} (-0.18)}
		\label{fig::acesac_az}
	\end{subfigure}
	\medskip
	%% Second row
	\begin{subfigure}{0.3\textwidth}
		\centering
		\input{graphs/azimuth_BET-ALL_10}
		\caption{\textit{Betula alleghaniensis} (0.25)}
		\label{fig::betall_az}
	\end{subfigure}
	\hfil
	\begin{subfigure}{0.3\textwidth}
		\centering
		\input{graphs/azimuth_BET-PAP_10}
		\caption{\textit{Betula papyrifera} (0.48)}
		\label{fig::betpap_az}
	\end{subfigure}
	\hfil
	\begin{subfigure}{0.3\textwidth}
		\centering
		\input{graphs/azimuth_FAG-GRA_10}
		\caption{\textit{Fagus grandifolia} (-0.37)}
		\label{fig::faggran_az}
	\end{subfigure}
	\medskip
	%% Third row
	\begin{subfigure}{0.3\textwidth}
		\centering
		\input{graphs/azimuth_PIC-GLA_10}
		\caption{\textit{Picea glauca} (0.08)}
		\label{fig::picgla_az}
	\end{subfigure}
	\hfil
	\begin{subfigure}{0.3\textwidth}
		\centering
		\input{graphs/azimuth_PIC-MAR_10}
		\caption{\textit{Picea mariana} (0.27)}
		\label{fig::picmar_az}
	\end{subfigure}
	\hfil
	\begin{subfigure}{0.3\textwidth}
		\centering
		\input{graphs/azimuth_PIC-RUB_10}
		\caption{\textit{Picea rubens} (-0.26)}
		\label{fig::picrub_az}
	\end{subfigure}
	\medskip
	%% Forth row
	\begin{subfigure}{0.3\textwidth}
		\centering
		\input{graphs/azimuth_PIN-BAN_10}
		\caption{\textit{Pinus banksiana} (-0.1)}
		\label{fig::pinban_az}
	\end{subfigure}
	\hfil
	\begin{subfigure}{0.3\textwidth}
		\centering
		\input{graphs/azimuth_PIN-STR_10}
		\caption{\textit{Pinus strobus} (0.02)}
		\label{fig::pinstr_az}
	\end{subfigure}
	\hfil
	\begin{subfigure}{0.3\textwidth}
		\centering
		\input{graphs/azimuth_POP-TRE_10}
		\caption{\textit{Populus tremuloides} (0.45)}
		\label{fig::poptre_az}
	\end{subfigure}
	\medskip
	%% Fifth row
	\begin{subfigure}{0.3\textwidth}
		\centering
		\input{graphs/azimuth_THU-OCC_10}
		\caption{\textit{Thuja occidentalis} (0.24)}
		\label{fig::thuocc_az}
	\end{subfigure}
	\hfil
	\begin{subfigure}{0.3\textwidth}
		\centering
		\input{graphs/azimuth_TSU-CAN_10}
		\caption{\textit{Tsuga canadensis} (0.17)}
		\label{fig::tsucan_az}
	\end{subfigure}
	\hfil
	\begin{subfigure}{0.3\textwidth}
		\centering
		\input{graphs/legend_cols}
	\end{subfigure}
	\caption{Species-specific averaged direction of increase in $ \tilde \rho_0 $ for the northern region (blue arrows) and southern region (orange arrows). The black square represents the species-specific centroid of the distribution $ \Omega $, and is the reference point that defines the northern region (everything north of the centroid) and the southern region. If population performances were higher in the centre of the distribution, then the arrows would point towards the centroid. The number within parentheses is the correlation between $ \tilde \rho_0(\s = 10 \text{m}) $ and the random forest. See Fig. \ref{fig::3correls} and Supplementary Information Appendix S6 for the correlations, and Figures S5 for the azimuths. \label{fig::grad_cols}}
\end{figure}

The lifetime number of recruits per individual $ \tilde \rho_0 $ aggregates the three vital rates into a species-specific performance measure. We computed correlations between occurrence probabilities $ P_{\text{occ}} $ and individual growth and mortality to disentangle their effects upon $ \tilde \rho_0 $ (Supporting Information Appendix S6). Overall, we found no rule relating occurrence probabilities with demography. When $ P_{\text{occ}} $ is positively correlated with individual growth, and negatively correlated with mortality, then the correlation of $ P_{\text{occ}} $ with $ \tilde \rho_0 $ is also positive (\eg \textit{Betula papyrifera}, Supporting Information Appendix S6). In this case, $ P_{\text{occ}} $ correlations with $ \tilde \rho_0 $ are consistently higher than $ P_{\text{occ}} $ correlations with demography (Supporting Information Appendix S8). However, when $ P_{\text{occ}} $ is (counter-intuitively) positively correlated with mortality rates, then the correlation of $ P_{\text{occ}} $ with $ \tilde \rho_0 $ drops or even becomes negative (\eg \textit{Fagus grandifolia}, Supporting Information S6). These results demonstrate that there may be compensatory strategies (such as a higher recruitment rather than our constant fecundity function), which could explain why $ P_{\text{occ}} $ is positively correlated with mortality.

The correlation between $ \tilde \rho_0 $ and the probability of occurrence $ P_{\text{occ}} $ did not show any trend in the absence of competition (varies from $ - 0.49 $ to $ 0.60 $), nor was there a trend with a canopy height of $ 10 $ metres (ranges within $ - 0.51 $ to $ 0.47 $). The correlation decreases with competition for most boreal species (\textit{Betula papyrifera, Picea glauca, Picea mariana, Picea rubens, Pinus banksiana} and \textit{Thuja occidentalis}, but not \textit{Abies balsamea}). For the shade-intolerant \textit{Populus tremuloides}, accounting for competition increased the correlation between $ \tilde \rho_0 $ and $ P_{\text{occ}} $, while the correlations for the non-boreal shade-tolerant species (\textit{Acer rubrum, Acer saccharum, Betula alleghaniensis, Fagus grandifolia, Pinus strobus} and \textit{Tsuga canadensis}) are similar with and without competition (Fig. \ref{fig::3correls}). SDMs that were based on random forests accurately described occurrence probabilities for all species ($ R_{\text{Tjur}}^2 $ ranging from $ 0.74 $ to $ 0.85 $).
\begin{figure}
	\centering
	\input{graphs/3correlations}
	\caption{Species-specific correlations of $ \tilde \rho_0 $ and the SDM without competition (\MoveUp), or with competition (canopy height $ \s = 10 $ m, \CircSteel). The three colours correspond to the shade tolerance level (the darker, the more shade-tolerant). The values can be found in Supporting Information S6. On this figure, we sorted the data by increasing order of correlation with competition (\CircSteel) rather than alphabetical order. \label{fig::3correls}}
\end{figure}
