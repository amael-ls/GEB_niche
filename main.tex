\documentclass[letterpaper, 12pt]{article}

%%%%%%%%%%%%%%%%%%%    PACKAGES    %%%%%%%%%%%%%%%%%%%
%% Package version
\listfiles % Then check the .log file

%% Font
\usepackage{fontspec}
	
\usepackage{marvosym}

%% Languages
\usepackage{polyglossia}
	\setdefaultlanguage[variant=british]{english}

%% Marges, space...
\usepackage[top=2.5cm, bottom=2.5cm, left=2.5cm, right=2.5cm]{geometry}
\usepackage{setspace} % [nodisplayskipstretch] pour option space in equation

\usepackage{indentfirst}

\usepackage[bottom]{footmisc}
\usepackage{footnote}

% https://tex.stackexchange.com/questions/279/how-do-i-ensure-that-figures-appear-in-the-section-theyre-associated-with
\usepackage[section]{placeins} % To place figures in the section it is declared

\usepackage[nolists,tablesfirst,nomarkers]{endfloat} % To place fig & tables at the end

%% Numbering figures in sections rather than continuously http://tex.stackexchange.com/questions/28333/continuous-v-per-chapter-section-numbering-of-figures-tables-and-other-docume
\usepackage{chngcntr}
\counterwithin{figure}{section}

%% Graphics
\usepackage[luatex]{graphicx}
	\graphicspath{{graphs/}}

\usepackage{caption}
\usepackage{subcaption}

\usepackage[dvipsnames, svgnames]{xcolor}

\usepackage{tikz}
	\usetikzlibrary{arrows, plotmarks, decorations.markings}
	\tikzstyle{arrow} = [->,>=stealth,thick,rounded corners=4pt,line width=1pt]
	\usetikzlibrary{shadows}
	\usetikzlibrary{shadings}
	\usetikzlibrary{positioning} % relative coodinate
	\usetikzlibrary{tikzmark, calc} % calc, to calculate coordinate
	\usetikzlibrary{decorations.pathmorphing} % to snake an arrow
	\usetikzlibrary{shapes.arrows}
	\usetikzlibrary{patterns}
	\tikzset{
		invisible/.style={opacity=0},
		visible on/.style={alt={#1{}{invisible}}},
		alt/.code args={<#1>#2#3}{%
		\alt<#1>{\pgfkeysalso{#2}}{\pgfkeysalso{#3}} % \pgfkeysalso doesn't change the path
		},
	} % end tikzset. Code from http://tex.stackexchange.com/questions/136143/tikz-animated-figure-in-beamer
% Gradient shadings command
\makeatletter
\def\createshadingfromlist#1#2#3{%
	\pgfutil@tempcnta=0\relax
	\pgfutil@for\pgf@tmp:={#3}\do{\advance\pgfutil@tempcnta by1}%
	\ifnum\pgfutil@tempcnta=1\relax%
		\edef\pgf@spec{color(0)=(#3);color(100)=(#3)}%
	\else%
		\pgfmathparse{50/(\pgfutil@tempcnta-1)}\let\pgf@step=\pgfmathresult%
		\pgfutil@tempcntb=1\relax%
		\pgfutil@for\pgf@tmp:={#3}\do{%
			\ifnum\pgfutil@tempcntb=1\relax%
				\edef\pgf@spec{color(0)=(\pgf@tmp);color(25)=(\pgf@tmp)}%
			\else%
	        \ifnum\pgfutil@tempcntb<\pgfutil@tempcnta\relax%
				\pgfmathparse{25+\pgf@step/4+(\pgfutil@tempcntb-1)*\pgf@step}%
				\edef\pgf@spec{\pgf@spec;color(\pgfmathresult)=(\pgf@tmp)}%
	        \else%
	        	\edef\pgf@spec{\pgf@spec;color(75)=(\pgf@tmp);color(100)=(\pgf@tmp)}%
	        \fi%
	    \fi%
	    \advance\pgfutil@tempcntb by1\relax%
	    }%
	\fi%
	\csname pgfdeclare#2shading\endcsname{#1}{100}\pgf@spec%
}

% --- Define colours
\definecolor{Rblack}{RGB}{0,0,0}
\definecolor{Rgrey}{RGB}{42,51,68}
\definecolor{RdeepBlue}{RGB}{17,34,170}
\definecolor{RmidBlue}{RGB}{32,88,220}
\definecolor{RlightBlue}{RGB}{90,130,234}
\definecolor{RuglyGreen}{RGB}{253,236,190}
\definecolor{RyellowPea}{RGB}{253,219,91}
\definecolor{RdeepYellow}{RGB}{250,185,53}
\definecolor{Rsalmon}{RGB}{253,152,89}
\definecolor{Rred}{RGB}{182,52,58}

\definecolor{lightAvailability_grey}{RGB}{20,20,20}

\createshadingfromlist{RmapShading}{vertical}{Rblack,Rgrey,RdeepBlue,RmidBlue,RlightBlue,RuglyGreen,RyellowPea,RdeepYellow,Rsalmon,Rred}
\createshadingfromlist{lightAvailability}{vertical}{lightAvailability_grey,white}
\createshadingfromlist{RmapShading_h}{horizontal}{Rblack,Rgrey,RdeepBlue,RmidBlue,RlightBlue,RuglyGreen,RyellowPea,RdeepYellow,Rsalmon,Rred}

%% Tab, list...
\usepackage{booktabs}
\usepackage{longtable}
	\LTcapwidth=\textwidth

\usepackage[export]{adjustbox} % To use valign in tabular environment

\usepackage{listings}

\usepackage{multirow}

\usepackage{enumerate}

\usepackage{enumitem}% http://ctan.org/pkg/enumitem

%% Table of content
\setcounter{tocdepth}{-2} % To avoid printing the sections from the main body in ToC
\addto\captionsbritish{ % Replace "british" with the language used in babel
	\renewcommand{\contentsname}{Supporting Information contents}
}

%% Links
\usepackage{url}
\usepackage[luatex, colorlinks=true, linkcolor=NavyBlue, urlcolor=MidnightBlue, citecolor=PineGreen]{hyperref}

%% bibliography
\usepackage{csquotes}
\usepackage[style=apa, natbib=true, sorting=ynt]{biblatex}
\addbibresource{article.bib}

%% mathematics
\usepackage{amsthm}
\usepackage{amsmath}
	\allowdisplaybreaks % Autoriser découpe formules entres pages
\usepackage{amssymb}
\usepackage{bbold}
\usepackage{dsfont}
\usepackage{mathrsfs}
\usepackage{bm}
\usepackage{xfrac}
\usepackage{etoolbox} % For renumbering (cf below, counter for model)

\usepackage{thmbox} % cf after for "theorem" definitions

\usepackage{gensymb}

\usepackage{siunitx}

% Creation of a new counter for the models, from:
% https://tex.stackexchange.com/questions/84254/how-to-create-new-counter-of-equation
\newcounter{growthCounter}
\renewcommand*{\thegrowthCounter}{G\arabic{growthCounter}}

\makeatletter
\def\@equationname{equation}
\newenvironment{g}[1]{%
    \def\mymathenvironmenttouse{#1}%
    \ifx\mymathenvironmenttouse\@equationname%
        \refstepcounter{growthCounter}%
    \else
        \patchcmd{\@arrayparboxrestore}{equation}{growthCounter}{}{}%          doesn't change output?
        \patchcmd{\print@eqnum}{equation}{growthCounter}{}{}%
        \patchcmd{\incr@eqnum}{equation}{growthCounter}{}{}%
    \fi
    \csname\mymathenvironmenttouse\endcsname%
}{%
    \ifx\mymathenvironmenttouse\@equationname%
        \tag{\thegrowthCounter}%
    \fi
    \csname end\mymathenvironmenttouse\endcsname%
}
\makeatother


\newcounter{mortalityCounter}
\renewcommand*{\themortalityCounter}{M\arabic{mortalityCounter}}

\makeatletter
\def\@equationname{equation}
\newenvironment{m}[1]{%
    \def\mymathenvironmenttouse{#1}%
    \ifx\mymathenvironmenttouse\@equationname%
        \refstepcounter{mortalityCounter}%
    \else
        \patchcmd{\@arrayparboxrestore}{equation}{mortalityCounter}{}{}%          doesn't change output?
        \patchcmd{\print@eqnum}{equation}{mortalityCounter}{}{}%
        \patchcmd{\incr@eqnum}{equation}{mortalityCounter}{}{}%
    \fi
    \csname\mymathenvironmenttouse\endcsname%
}{%
    \ifx\mymathenvironmenttouse\@equationname%
        \tag{\themortalityCounter}%
    \fi
    \csname end\mymathenvironmenttouse\endcsname%
}
\makeatother

%% Line numbers --conflict with amsthm-- pdf http://texblog.org/2012/02/08/adding-line-numbers-to-documents/
\usepackage{lineno}
\linenumbers

%%%%%%%%%%%%%%%%%   NEW COMMANDES    %%%%%%%%%%%%%%%%%
%% Text
\newcommand {\ie}{\textit{i.e., }}
\newcommand {\eg}{\textit{e.g., }}
\newcommand {\cf}{\textit{cf} }
\newcommand\bsc[1]{\textsc{\MakeLowercase{#1}}} % Only if there is no french babel
\newcommand {\thup}[1]{{#1}\textsuperscript{th}}

%% Math
\newcommand {\s}{{s}^{*}}
\newcommand {\sst}{\tilde{s}^{*}} % s* stable d'où le st
\newcommand {\N}{\tilde{N}}
\newcommand {\A}{\mathscr{A}}
\newcommand {\K}{\mathcal{K}}
\renewcommand{\S}{\mathscr{S}}
\newcommand{\R}{\mathds{R}}
\newcommand{\Prob}{\mathds{P}}
\newcommand{\F}{\mathcal{F}}

\DeclareMathOperator{\logit}{logit}

%%%%%%%%%%%%%%%%%%   HYPHENATION    %%%%%%%%%%%%%%%%%%
% \hyphenation{cal-cu-lus}

%%%%%%%%%%%%%%%%%   THEOREM STYLE    %%%%%%%%%%%%%%%%%
\newtheoremstyle{theo}{\topsep}{\topsep}{\itshape}{}{\bfseries}{.}{\newline}{\thmname{#1} \thmnumber{#2} \thmnote{~: \textit{#3}}}
\theoremstyle{theo}
\newtheorem{rem}{Remark}[section]
\newtheorem{defi}{Definition}[section]
\newtheorem{assum}{Assumption}[section]
\newtheorem{nota}{Notation}[section]

%%%%%%%%%%%%%%%%%%%%%   OTHERS   %%%%%%%%%%%%%%%%%%%%%
% http://tex.stackexchange.com/questions/13304/which-package-version-am-i-using
\listfiles % Add \listfiles to your preamble and then look at the .log file. This will tell you the current version of all the packages loaded

%%%%%%%%%%%%%%%%%%%%%%%%%%%%%%%%%%%%%%%%%%%%%%%%%%%%%%

\title{Climate-induced variation in the demography of 14 tree species is not enough to explain their distribution in eastern North America}
\author{}

\begin{document}
\maketitle

\begin{onehalfspace}


\begin{abstract}
\noindent \textbf{Aim} Dynamic range models are proposed to investigate species distribution and project range shift under climate change. They are based on the Hutchinsonian niche theory, specifying that the occurrence of a species in an environmental space should be limited to positions where the intrinsic growth rate is positive. Evaluating population growth rate is however difficult for physiologically structured populations, such as forest stands, due to size-induced individual variation in performance. Therefore, we still have a limited understanding of which aspect of tree demography contributes the most to their geographical range limit. Here, we develop and test a model for tree species distribution based on demography. \\ % 110 words

\noindent \textbf{Location} North America (57--124 \degree W, 26--52 \degree N). \\ % 3 words

\noindent \textbf{Time period} 1963--2010. \\ % 2 words

\noindent \textbf{Major taxa studied} 14 tree species. \\ % 6 words

\noindent \textbf{Methods} We develop a structured-population model based upon the McKendrick--von Foerster equations to represent forest dynamics. We then derive lifetime reproduction per individual $ R_0 $ in the absence of density-dependence. Using forest inventory data, we assess for each species how tree demography varies with climate. We test the model by comparing $ R_0 $ and a probability of occurrence within species ranges. \\ % 71 words

\noindent \textbf{Results} Our analysis shows that $ R_0 $ is a non-linear relationship of individual growth, mortality, and fecundity. We find that both growth and mortality rates vary across species distributions, yet climate explains little of the observed variation. Individual size and neighbourhood competition are the primary explanatory variables of tree demography. Finally, we find that $ R_0 $ weakly relates to occurrence probability, with no systematic decline in population growth rates towards the range limits. \\ % 73

\noindent \textbf{Main conclusions} The spatial and size-induced variations in tree growth and mortality does not explain range limits. We propose that range limits of trees could either be explained by variation in establishment or by processes happening at larger spatial scales.  \\ % 41 words

\noindent {\bf Keywords}: Climate, Competition, Demography, Population growth rate, Niche theory, Species distribution, Structured-population model, Upscaling

\end{abstract}


% Deleted references: \citep{Levin2009}, \citet{Schurr2012}, \citep{Canham2004, Canham2006}

\section{Introduction}

A common, although rarely tested, assumption in ecology is that a species is more likely to be found where it performs the best. In other words, species probability of occurrence across its range should be positively correlated to the per capita intrinsic growth rate \citep{McGill2012}. This hypothesis stems from the interpretation of Hutchinsonian niche theory \citep{Hutchinson1957, Maguire1973}, which poses that species are limited to locations where the environmental conditions (\ie any property outside of the considered organisms) allow a population to persist. At the core of species distribution models, this hypothesis is used to identify the climatic variables that are constraining species ranges, and their projection in the future allows to forecast potential range shifts. \\

This niche theory, in its more concise formulation, relates the population growth rate $ r $ to the species' niche: the hypervolume in the environmental factors space is the set such that $ r \geqslant 0 $ \citep{Holt2009, Godsoe2017}. Formally, let $ r_i(\bm{E}, \bm{R}) $ be the growth rate of a focal species $ i $ when rare, namely the intrinsic growth rate for a given environment $ \bm{E} $ and amount of resources $ \bm{R} $. The equation
\[
	r_i(\bm{E}, \bm{R}) \geqslant 0,
\]
specifies that the fundamental niche corresponds to the locations where $ \bm{E} $ and $ \bm{R} $ allow positive growth. An equivalent representation is the lifetime number of recruits per individual, traditionally denoted by $ R_{0, i} $ \citep[where $ i $ is still the species index]{Pulliam2000, DeRoos1997}. A sustainable population requires $ R_{0, i} (\bm{E}, \bm{R}) \geqslant 1 $, that is to say, in average an individual needs, at least, to replace itself over its lifespan. This definition of the niche allows the species $ i $ to exert influence on the rates of other species of the community and get feedbacks on its own demographic rates. Hereafter, we drop the $ i $ index, but it is important to keep in mind that $ R_{0, i} $ is a species-specific rate related to a species $ i $. \\

Difficulties in testing the niche theory partially root in the challenge of measuring population growth rates \citep{McGill2012}. This is especially true for physiologically structured populations; for instance, the age or the size of an individual influence its reproduction success, feeding behaviour, or death probabilities. These age and size structures, combined with density-dependences and environmental influences, might blur the relationship between species' occurrence and $ R_0 $ variations along an environmental gradient. This is where individual-based structured-population models bring insight to addressing the complexity of population dynamics. Such models are however demanding to parameterise, and difficult to analyse. Good examples are forest trees, which can be modelled by spatially explicit simulators, accounting for single tree development and light availability to an individual. These forest simulators focus on the individual level, which is the relevant scale for studying competition and climate response, but usually questions of biogeographical interests (such as tree species distributions) lie at the population level, that is to say $ R_0 $. \\

The relationship between tree species range and population growth has recently been under scrutiny, both in Europe \citep{Thuiller2014} and north-eastern North America \citep{McGill2012}. Little correspondence was found between $ r $ and tree species distributions due to uncertainty on the demographic parameters \citep{Thuiller2014}. Negative correlations were even found, which at first glance, challenges the common assumption that a species is more abundant at its optimal environment. The inclusive niche was proposed as an alternative explanation to these negative correlations \citep{McGill2012}. This niche states that weak competitors have their fundamental niche reduced to a smaller realised space, in a trade-off between competitive ability and environmental tolerance \citep{Serrano2015}. Therefore, weak competitors can be more abundant in suboptimal environments. \\

Dynamic Range Models (DRMs) were recently developed to get closer to individual demographic rates, while up-scaling to the population level. DRMs are hierarchical statistical models that relate species' abundance to environmental data using two latent variables: the population growth rate $ r $ and dispersal. These models account for uncertainties in the data as well \citep{Pagel2012}. Therefore, DRMs allowed to track uncertainties raising from the demographic level, the biogeographic level, and the observer level. \\

There is currently no standard method to derive a single performance index from demographic rates \citep{Purves2009}. Therefore, all the studies linking species distributions to individuals' performance are in the midst of an `uncharted territory'. The aforementioned studies of \citeauthor{McGill2012}, \citeauthor{Thuiller2014}, or \citeauthor{Pagel2012} all explored different ways of linking distribution to population performance. They all agreed that combining the three vital rates, namely individual growth, mortality, and fecundity into $ r $ is difficult, and that $ r $ itself is not easily derived from census. Building on these manuscripts, we will derive $ r $ from a forest dynamics model that uses the three vital rates. In this article, we focus exclusively on radial growth and mortality, and make them life-stage dependent, which is of primary importance to propagate uncertainties up to $ r $ \citep[$ \lambda $ in his article]{Clark2003b}. \\

Our main objective in this study is to investigate if the distribution of North American tree species is driven by the effect of climate and light competition on individual demography. The abundant centre hypothesis postulates that demographic performance should decline toward range margins. We therefore investigate two predictions that (\textit{i}) per capita growth rate should vary with climate and as a result, (\textit{ii}) per capita growth rate should decline at range margins where occurrence probabilities tend to zero. We represent forest stand dynamics with a cohort-based model relying on the McKendrick-von Foerster equations \citep{Strigul2008}. This model relates individual tree demography to cohort dynamics, accounting for neighbourhood competition and ontogenic variation in demography. We derive a formula from the McKendrick-von Foerster equations to combine individual tree growth, mortality and fecundity rates into $ R_0 $, the per capita growth rate. We then evaluate how components of tree demography (individual growth and mortality) respond to climate, individual size, and competition. We expect an optimal climate for each species in the middle of the range, and better performance under light than shade conditions. Then after, we test if $ R_0 $ is positively correlated to the probability of occurrence and if it declines towards range limits. The analysis is performed for the 14 most abundant tree species in eastern North America.



% [ALS] Deleted references: Monsi2004, Pacala1994, Breiman2001, \citep[p. 121-122]{Zuur2009}, \citep{Vehtari2019}

\section{The Model}
\subsection{Model structure}
We model forest dynamics using a physiologically structured population model (PSPM), based on \citet{Strigul2008}, which we have made spatially explicit. We first provide general definitions that are related to PSPMs and then describe our own. A physiologically structured model distinguishes individuals who are at different stages of development. PSPMs are based upon individual states (hereafter, \textit{i}-states), namely, a collection of variables that exhibit two properties \citep[for an overview of PSPMs]{DeRoos1997}:
\begin{enumerate}[label=(\roman*)]
	\item \textit{i}-states completely determine the individual's growth rate, death rate and birth rate  at any given time (possibly together with the present environmental state), and its influence on the environment,
	\item \textit{i}-states future values are completely determined by their present values, together with the intervening environmental history as encountered by the individual of concern.
\end{enumerate}
The environment is accounted for by the environmental state (hereafter, \textit{e}-state). Formally, an \textit{e}-state is a collection of biotic and abiotic factors that characterise the environment in which an individual lives and that affect individual performance. In this paper, we consider two kinds of \textit{e}-states: (\textit{i}) feedback loops, that both influence and are influenced by individuals of all species; and (\textit{ii}) external forcing factors, which are imposed on the population. By definition, the former requires a dynamic descriptions. We ignore random variation among individuals of the same \textit{i}-state and which experience the same \textit{e}-state. As a result, the model represents cohorts of trees rather than individuals.

Cohort dynamics of species $ j $ and diameter $ s $ and that are located in $ x $ are modelled by a spatially-explicit version of the von Foerster--perfect plasticity approximation model \citep[hereafter, von Foerster--PPA]{Strigul2008}:
\begin{align}
	\frac{\partial N_j(s, x, t)}{\partial t} &= - \frac{\partial G_j \big(s, x; \s(x, t) \big) N_j(s, x, t)}{\partial s} - \mu_j\big(s, x; \s(x, t) \big) N_j(s, x, t) \label{eq::dynamics_x} \\
	N_j(0, x, t) &= \frac{1}{G_j(0, x; \s(x, t))} \int_0^{\infty} N_j(s, x, t) F_j\big(s; \s(x, t) \big) \, ds \label{eq::recruitment_x}
\end{align}
where $ N_j $ is the number of trees of species $ j $ per unit size per unit space, it as a density and only $ \int \int N \, ds dx $ can be considered as a number of individuals. $ G $ is the growth rate of individuals, $ \mu $ is the mortality rate, and $ F $ is the effective fecundity function (see Table \ref{tab::notations} for list of notations, definitions and units for each variable and parameter). Although we developed the model with a dispersal kernel, we decided to use a $ \delta $-Dirac distribution in this paper to maintain model tractability. Therefore, the dispersion is localised to patch $ x $, and does not appear in Equation \eqref{eq::recruitment_x}. The three demographic rates are affected by a size threshold $ \s $, which is a feedback loop that is defined formally below. External factors (at a location $ x $) only influence $ G $ and $ \mu $. Equation \eqref{eq::dynamics_x} describes cohort demography, while equation \eqref{eq::recruitment_x} describes recruitment. Together, \eqref{eq::dynamics_x}--\eqref{eq::recruitment_x} represent the structured-population dynamics of trees. We use tree diameter at breast height (\textit{dbh}) as a single \textit{i}-state and then employ allometric functions from \citet{Purves2007} to compute tree height and crown diameter.

% Work on the paragraph
Competition for light is the major driver of forest dynamics in north eastern North American forests, especially for saplings \citep{Pacala1996, Purves2007}. Nitrogen should have little influence in closed-canopy forests or in understorey conditions, and would be difficult to simulate \citep{Kobe2006}. We thus limit the feedback loop solely to light availability. Competition for water has a non-negligible effect on such forest dynamics at all light levels \citep{Kobe2006}. We also know that temperature impacts metabolic rates \citep{Brown2004}, therefore, we base the external factors on temperature and precipitation variables.

Competition for light is represented by a critical height that partitions the forest into the understorey and the overstorey (Fig. \ref{fig::ppa}). The vertical position of trees, thereafter, determines growth and mortality. The equation describing the feedback loop between cohort dynamics and light availability is:
\begin{equation}
	1 = \sum_{j = 1}^{n} \int_{\s(x)}^{\infty} N_j(s, x, t) \A_j \big(s; \s(x, t)\big) \, ds \label{eq::ppa_x}
\end{equation}
where $ \A $ denotes the cross-sectional area of the crown of an individual of size $ s $ (Fig \ref{fig::ppa}). This last equation \eqref{eq::ppa_x} defines a size threshold $ \s $, which differentiates the behaviour of individuals that are above $ \s $ from those that are below. Hence, $ \s $ is such that the sum of the area of individual crowns equals the area of the plot being considered. Trees with a height below $ \s $ are fully shaded whereas trees above $ \s $ are in the overstorey and receive direct sunlight. When the canopy is open, a positive value of $ \s $ cannot be attained and, therefore, $ \s $ is set to $ 0 $. It should be noted that when the size variable is the height of trees, then $ \s $ is independent of the species. However, when the dbh is the size variable, $ \s $ then becomes species-specific (due to the species-specificity of the allometric functions relating height to dbh). Equation \eqref{eq::ppa_x} is adapted from \citet{Strigul2008} and, therefore, obeys the \textit{perfect-plasticity approximation} assumption (PPA): the canopy is a collection of small crowns that can be reorganised such that the occupied area is maximised. Despite this optimisation of sun exposure is considered theoretical, the combination of PPA with the von-Foerster equations can accurately reproduce forest dynamics at stand scales, within a relatively homogeneous physical environment \citep{Strigul2008, Purves2008}. This set of three equations (\ref{eq::dynamics_x}--\ref{eq::ppa_x}) is a tractable PSPM, which allows us to derive analytically $ R_0 $.

\begin{figure}
	\centering
	\input{graphs/ppaStrigul}
	\caption{Traditionally, it is assumed that light availability decreases progressively from the canopy to the forest floor, obeying a Beer's law (left arrow). Here we assume there is a threshold $ \s $ that splits the forest into two parts, and defines therefore two light levels (right arrow): $ L_u $ and $ L_o $ for understorey and overstorey light respectively. Hence the light is a stepwise function of canopy height. The threshold $ \s $ is the maximum height at which the canopy is closed; at this particular height, the sum of all the cross section equals the plot area. $ \A(s, \sigma) $ is the area of the cross section of the crown at height $ \sigma $ of an individual of height $ s $. \label{fig::ppa}}
\end{figure}

\begin{table}[h]
	\centering
	\caption{Notations used in this paper (sorted in alphabetical order---using equivalents for the Greek letters). $ T $ stands for time unit, $ \ell $ for individual tree length unit, and $ a $ for the spatial unit (a length if the forest is one-dimensional, and an area, if it is a two-dimensional forest).} \label{tab::notations}
	\begin{tabular}{@{}cll@{}}
		\toprule
		\textbf{Symbol} & \multicolumn{1}{c}{\textbf{Definition}} & \multicolumn{1}{c}{\textbf{Unit}} \\
		$ \A $ & Area of the cross-section of the crown &  $ a $ \\
		$ \text{age}_{\max} $ & Maximum age from \citet{Burns1990, Burns1990a} & $ T $ \\
		$ \text{dbh}_{\max} $ & Maximum dbh from \citet{Burns1990, Burns1990a} & $ \ell $ \\
		$ \F $ & Number of seeds per tree's crown area per time & $ a^{-1} T^{-1} $ \\
		$ F $ & Effective fecundity function, \ie number of germinating seeds & $ T^{-1} $ \\
		$ F_{\text{Purves}} $ & Effective fecundity function (D. Purves, 2008) & $ T^{-1} $ \\
		$ G $ & Growth of individuals & $ \ell T^{-1} $ \\
		$ \mu $ & Mortality rate & $ T^{-1} $ \\
		$ N $ & `Density' of trees & $ \ell^{-1} a^{-1} $ \\
		$ \Omega $ & Landscape or expert map, $ \Omega \subseteq \mathds{R}^2 $ & $ a $ \\
		$ \varphi $ & Ratio of $ \Delta \text{AIC}_c $ for models with $ \text{VIF} < 20 $ & - \\
		$ R_0 $ & Net population growth rate & - \\
		$ \rho_0 $ & Net population growth rate using $ F_{\text{Purves}} $ and $ s_{\infty}^{x} $ & - \\
		$ \tilde \rho_0 $ & Standardised $ \rho_0 $ & - \\
		$ s $ & Size of individuals (either dbh or height) & $ \ell $ \\
		$ \s $ & Size threshold that separates the forest into two strata & $ \ell $ \\
		$ \s_c $ & Constant size threshold, value set to 0 m or 10 m for the maps & $ \ell $ \\
		$ s_{\max}^{x} $ & Maximal dbh that a tree would have at location $ x $ without competition & $ \ell $ \\
		$ s_{\infty}^{x} $ & Upper bound of integration in $ \rho_0 $ formula & $ \ell $ \\
		$ t $ & Time & $ T $ \\
		$ x $ & Space variable, $ x \in \Omega $ & - \\
		\bottomrule
	\end{tabular}
\end{table}

\subsection{Net reproduction rate $ R_0 $}
Henceforth, the height $ \s $ that separates shaded trees from sun-exposed trees is referred to as `competition'. We used the method of characteristics to calculate the net reproduction rate at a location $ x $ within a landscape $ \Omega $ as a function of a constant competition $ \s_c $. This mathematical technique is used to solve certain partial-differential equations, as commonly used in transport equations. For example, equation \eqref{eq::dynamics_x} describes the advection of trees growing at a non-constant speed $ G $ along the size axis (either height or diameter). Characteristics allow us to follow individuals throughout their lifespan, \ie they represent the trajectories of individuals in the time-size plane (fig. \ref{fig::chara} for an example). The derivation of $ R_0 $ is detailed in Supporting Information \ref{app::calc_R0::sec::R0}; it should be remembered three underlying assumptions that were made for this calculation: (\textit{i}) $ \s $ is considered fixed and known at a value $ \s_c $; (\textit{ii}) only trees larger than $ \s_c $ can reproduce \citep{Strigul2008}; and (\textit{iii}) dispersal is limited to the patch, as stated by equation \eqref{eq::recruitment_x}. Subsequently, the net reproduction rate in a patch $ x $ is consequently:
\begin{equation} \label{eq::R0sol}
	R_0 (x, \s_c) = \exp \left[-\int_0^{\s_c}\frac{\mu(s, \s_c, x)}{G(s, \s_c, x)} \, ds \right] \times \int_{\s_c}^{\infty} \frac{F(s, \s_c)}{G(s, \s_c, x)} \exp \left[-\int_{\s_c}^{s} \frac{\mu(\sigma, \s_c, x)}{G(\sigma, \s_c, x)} \, d\sigma \right] \, ds
\end{equation}
Equation \eqref{eq::R0sol} can be divided into two biological processes:
\begin{itemize}
	\item $ \exp\left[-\int_0^{\s_c}\frac{\mu(s, \s_c, x)}{G(s, \s_c, x)} \, ds \right] $ is the proportion of individuals that survive up to the canopy of height $ \s_c $ in plot $ x $,
	\item $ \int_{\s_c}^{\infty} \frac{F(s, \s_c)}{G(s, \s_c, x)} \exp \left[-\int_{\s_c}^{s} \frac{\mu(\sigma, \s_c, x)}{G(\sigma, \s_c, x)} \, d\sigma \right] \, ds $ is the expected production of offspring of an individual that is located in $ x $, during its lifespan. It has two subterms:
	\begin{itemize} % [label=$ \circ $]
		\item $ \frac{F(s, \s_c)}{G(s, \s_c, x)} $ is the number of offspring per unit time from individuals that grow at speed $ G $,
		\item $ \exp \left[-\int_{\s_c}^{s} \frac{\mu(\sigma, \s_c, x)}{G(\sigma, \s_c, x)} \, d\sigma \right] $ is the survivorship of trees of size $ s $. Since $ \mu $ and $ G $ are both positive functions, survivorship is a decreasing function of $ s $.
	\end{itemize}
\end{itemize}

From equation \eqref{eq::R0sol}, we understand that the reproduction rate $ R_0 $ can be strengthened by reducing the competition $ \s_c $, by accelerating the average understorey growth $ G(s < \s_c) $ or diminishing the average mortality rate $ \mu $, or by enhancing the fecundity $ F $. The mathematical proof of these three mechanisms is found in the Supporting Information \ref{app::calc_R0::sec::3asser}. The formula \eqref{eq::R0sol} is a generalisation of the $ R_0 $ that was derived by Purves and is now valid for any individual growth, mortality, or fecundity function. The same assertions can be drawn from this study (Supporting Information \ref{app::purves2009}); however, due to the complexity of the computations we cannot assert that an averagely faster overstorey growth rate leads to an increase in $ R_0 $. For the same reason, we cannot calculate the value of $ \s_c $ for a population at equilibrium (\ie when $ R_0 = 1 $), except when the demographic rates are easier to compute, as step functions \citep[Supporting Information \ref{app::purves2009} for the proof]{Purves2009}. The net reproductive rate $ R_0 $ derived here is a heuristic tool that relates the individual tree performance under a competition constraint $ \s_c $ to the population performance and is thus the key to addressing our second objective.

\subsection{Data}
We parameterised the demographic functions $ G $ (individual growth) and $ \mu $ (mortality) using data that were obtained from permanent sample plots of the Forest Inventory and Analysis (FIA, USDA Forest Service), the Ministère des Forêts, de la Faune et des Parcs du Québec, the Ministry of Natural Resources and Forestry of Ontario, Ministry of Natural Resources of New Brunswick, and the forest products company Domtar (Fig. \ref{fig::mapDatabase} for a map of the data). After removing plots that experienced fire or logging, there were \num{7704442} individual measurements (106 species distributed among \num{132240} plots). A record consists of the tree identity, the species, the year at which the individual has been measured, the \textit{dbh}, the latitude and longitude of the plot, and the tree's status (alive or dead). Measurements occurred between 1963 and 2010, and frequencies of measurement range from $ \sfrac{1}{1} \text{ yr}^{-1} $ to $ \sfrac{1}{40} \text{ yr}^{-1} $ (with $ 96 \, \% $ of the data between $ \sfrac{1}{3} \text{ yr}^{-1} $ to $ \sfrac{1}{15} \text{ yr}^{-1} $). Both radial growth and mortality are highly variable across the distributions of the individuals (Supporting Information \ref{app::database}).

Climate data were extracted for each plot using \textsc{anusplin} software \citep{McKenney2011} based upon the latitude and longitude coordinates of the permanent plots. Note that for privacy reasons, the FIA offset plot locations up to $ 1.6 $ km \citep{Gray2012} which might therefore imply mismatches between the real climate of the plot and the climate that we assigned. We selected 19 climatic variables (Supporting Information \ref{app::database}, Table \ref{tab::bioclim}) covering the period 1958-2010 with a spatial resolution of 60 arc seconds ($ \approx 2 \, \text{km}^2 $). To account for climate variability prior to each tree measurement, we averaged each temperature and precipitation variable over a period of 5 years using a moving average (5 years is the most frequent interval among measurements that are included in the database with $ 38.8 \, \% $ of observations).

We calculated individual tree height and crown area from allometric functions and parameters that are provided in \citet{Purves2007}. We considered the dbh of all trees, although most of Canadian inventories start with individuals having diameters greater than $ 100 $ mm, while USA inventories start at $ 127 $ mm (5 inches). Some trees have been recorded as dead and then alive; we considered the last living state being true and ignored `resurrection' events in mortality estimation. By definition, growth and mortality rates require at least two measurements from an individual tree. After calculating the threshold $ \s $ in each plot, single-measured trees were removed from the dataset. For the growth analysis, we eliminated dead trees and individuals that have either a non-positive dbh increment or a radial growth greater than 25mm per year. We parameterised the model for the 14 most abundant species in north-eastern North America (Table \ref{tab::species}), but we considered all of the 106 species in the database for the computation of $ \s $. The 14-species dataset that is used in this paper contains \num{69954} plots (75 \% in the USA, and 25 \% in Canada), for a total of \num{3816854} individual measurements. \textit{Abies balsamea} is the most measured species with \num{822265} individual measurements, whereas \textit{Tsuga canadensis} is the least frequently measured with \num{66008} individual measurements. The climatic and geographical ranges of each species can be found in Table \ref{tab::spaceRange} and Fig. \ref{fig::speciesClimRange} in Supporting Information \ref{app::database}.

%%%% RESTART HERE -----------------------------------------  L 711 in Word
\subsection{Parameterisation of demographic rate functions}
We used linear mixed models to parameterise the individual growth ($ G $) and mortality ($ \mu $) rates as a function of climate, canopy status (understorey if below $ \s $, overstorey otherwise), and size (dbh). For the fecundity, we used the functions and values from \citet{Purves2008}. We tested linear and quadratic functional forms for the temperature, precipitation, and dbh effects, which were all normalised using the standard score. To get an optimal climate in the quadratic case, it is necessary to have a negative (positive) slope in front of the squared climate variables for the growth (mortality). However, constraining the parameters would force the optimal climate to be within the data used for parameterisation, thus we did not set any constraint. The 19 variables of temperature and precipitation allow us to try different combinations, however we preselected certain climatic assemblages based on the interpretability of the models and the literature (Supporting Information \ref{app::glmm}). Each set of species-specific demographic parameters was estimated separately. We based the model comparison on information criteria and $ R^2 $. We ranked the models for each species, and selected the model that in average fits the best according to information criteria. Then, this model was imposed on the 14 species. The R scripts used to format the data and to estimate demographic parameters are available on github (see Data Availability Statement). 

\subsubsection{Growth}
For the individual growth rate, we assumed a lognormal distribution, hence $ \log(G) $ is normally distributed. We normalised the logarithm of growth
\[
	Y_{G} = \frac{\log(G) - \mathrm{E}[\log(G)]}{\mathrm{sd}[\log(G)]},
\]
and used the following model:
\begin{equation} \label{eq::glmm_growth}
\begin{split}
	E[Y_G^{i, j}] = & \beta_{j, (x, t)[i]}^{(0)} + \beta_j^{(1)}  \text{canopy status} \, + \\
	& \left(\beta_j^{(2)} + b_j^{(2)} \text{canopy status} \right) T \, +
	\left(\beta_j^{(3)} + b_j^{(3)} \text{canopy status} \right) T^2 \, + \\
	& \left(\beta_j^{(4)} + b_j^{(4)} \text{canopy status} \right) P \, +
	\left(\beta_j^{(5)} + b_j^{(5)} \text{canopy status} \right) P^2 \, + \\
	& \left( \beta_j^{(6)} + b_j^{(6, 1)} T + b_j^{(6, 2)} T^2 +
		b_j^{(6, 3)} P + b_j^{(6, 4)} P^2 \right) \text{dbh} \, + \\
	& \left( \beta_j^{(7)} + b_j^{(7, 1)} T + b_j^{(7, 2)} T^2 +
		b_j^{(7, 3)} P + b_j^{(7, 4)} P^2 \right) \text{dbh}^2
\end{split}
\end{equation}
where $ E[Y_G] $ is the expected value of the normalised logarithm of growth, and $ T $ and $ P $ are the associated explanatory temperature and precipitation, respectively. The canopy status is a boolean (true for the canopy trees, and false otherwise). The indices $ i $ and $ j $ stand for individual and species respectively, and $ (x, t)[i] $ index denotes the group effects (plot $ x $ and year $ t $ of the $ i^{\text{th}} $ individual). The random effects (with a maximum of three effects: spatial, temporal, and plot-specific temporal respectively) are normally distributed:
\begin{align*}
	\beta_{j}^{(0, x)} &\sim \mathcal{N}(0, \bm{\sigma}_{x}^{\text{growth}}) \\
	\beta_{j}^{(0, t)} &\sim \mathcal{N}(0, \bm{\sigma}_{t}^{\text{growth}}) \\
	\beta_{j}^{(0, xt)} &\sim \mathcal{N}(0, \bm{\sigma}_{xt}^{\text{growth}})
\end{align*}
Mixed models allow us to group individuals by plots and years of the second measurement to consider spatial and temporal structures. The plot effect comprehend the variation driven by local factors such as soil condition and disturbance history, while the year (within plot identity) represents the temporal variation that is not included in climate. The $ \beta $s are the regression coefficients, and the $ b $s correspond to different variable interactions. Climate interacts with the crowding effect (canopy status) to account for climate response variation to the individual's neighbourhood. Lastly, as bigger trees may be favoured or disadvantaged by climate, dbh interacts with climatic variables $ T $ and $ P $. \\

We used the package \textit{lme4} \citep{lme4} to estimate the parameters ($ \beta $s and $ b $s). We tested sub-models of equation \eqref{eq::glmm_growth} to evaluate the impact of competition and climate on $ G $. We used a top-down strategy to select first the random effect structure, and secondly the fixed structure. We compared models with the Akaike Information Criteria ($ \text{AIC}_c $), but kept models with a maximum Variance Inflation Factor (VIF) lower than $ 20 $ to avoid correlations between variables \citep{Zuur2010}. We chose $ 20 $ because of one model for growth that have reasonable maximum VIFs (below 6) for 11 species over 14, and a VIF of 19 for \textit{Fagus grandifolia}. We judged that a VIF beyond $ 20 $ implies too much collinearity (F. Guillaume Blanchet, \textit{discussion}). We denote by $ \Delta \text{AIC}_c $ the difference between a model $ i $ and the model that have the smallest $ \text{AIC}_c $:
\[
	\Delta \text{AIC}_c = \text{AIC}_c^{(i)} - \text{AIC}_c^{(\min)}
\]
The best model has the lowest $ \text{AIC}_c $, or equivalently $ \Delta \text{AIC}_c = 0 $. The $ \Delta \text{AIC}_c $ are calculated unconstrainedly with respect to the best model, but, due to the VIF constraint, the selected model is not the model with $ \Delta \text{AIC}_c = 0 $. Therefore, using exclusively the models satisfying the constraint $ \text{VIF} < 20 $, we computed the common logarithm of the ratio between each model's $ \Delta \text{AIC}_c $ and the minimum $ \Delta \text{AIC}_c $:
\begin{equation} \label{eq::ratio}
	\varphi = \log_{10}\left[ \frac{\Delta \text{AIC}_c}{\min \left( \Delta \text{AIC}_c \right)} \right]
\end{equation}
This ratio represents, within the subset of models satisfying the constraint $ \text{VIF} < 20 $, how many times (in power of 10) is the best model compared to the other models. The common logarithm provides a convenient scale to compare the models, and the best constrained model has $ \varphi = \log_{10}(1) = 0 $.

\subsubsection{Mortality}
For the mortality rate, the response variable $ Y $ is a boolean describing the transition state between two records (true if there is a transition from alive to dead, and false if the individual stays alive). The observation of a mortality event depends on the survey interval $ \Delta t $ \citep{Lines2010}. To survive from $ t_0 $ to $ t_1 = t_0 + \Delta t $, an individual $ i $ must survive each year:
\[
	\Prob[\text{survival } i: \, t_0 \to t_1] = \big(1 - \Prob[\text{annual mortality } i]\big)^{\Delta t}
\]
where $ \Prob $ stands for probability. Thus, the probability of observing a mortality event within a span $ \Delta t $ is:
\[
	\Prob[\text{mortality } i: \, t_0 \to t_1] = 1 - \big(1 - \Prob[\text{annual mortality } i]\big)^{\Delta t}
\]

We assumed $ Y $ follows a binomial distribution and used the complementary log-log link function $ g $ to account for the time between two surveys (offset on the intercept, Supporting Information \ref{app::glmm} for a short study of the clog-log function):
\[
	g \big(E[Y_\mu^{i, j}] \big) = \log \left( -\log(1 - E[Y_\mu^{i, j}]) \right)
\]
We worked with the following model:
\begin{equation}\label{eq::glmm_mortality}
\begin{split}
	g \big(E[Y_\mu^{i, j}] \big) = & \beta_{j}^{(0)}  + \beta_j^{(1)}  \text{canopy status} +
		\text{offset}\big(\log(\Delta t_i) \big) \, + \\
	& \left(\beta_j^{(2)} + b_j^{(2)} \text{canopy status} \right) T \, +
	\left(\beta_j^{(3)} + b_j^{(3)} \text{canopy status} \right) T^2 \, + \\
	& \left(\beta_j^{(4)} + b_j^{(4)} \text{canopy status} \right) P \, +
	\left(\beta_j^{(5)} + b_j^{(5)} \text{canopy status} \right) P^2 \, + \\
	& \beta_j^{(6)} \text{dbh} \, + \beta_j^{(7)} \text{dbh}^2
\end{split}
\end{equation}
with the same notations to the growth model. We did not include any group effect, as some plots have only one record for certain years and species, while other plots have no dead trees recorded (problem also faced in \citet{Kunstler2019}). To minimise the uncertainty of death events and to have enough measures per species per time interval, we limited the dataset to measurements with $ \Delta t \in [5, 11] $ ($ 74.9 \% $ of the mortality database). Since the probability of transition from dead to dead is 1, we kept records up to the first death event only (or all the measurements if there were no death event). \\

For the mortality, none of the GLMMs from the package \textit{lme4} converged. We suspect that despite the amount of data, there is little information due to the rarity of tree mortality. We therefore used the package \textit{rstanarm} \citep{rstanarm} which provides GLMMs in a Bayesian framework, and removed the climate-dbh interactions. We used 4 chains and 3000 iterations for each chain. We kept the default priors of \textit{rstanarm}, that are Gaussian distributions for the regression coefficients $ \beta $s, and exponential distribution for the standard deviation. Parameter values used later in the analysis are the medians of the posterior distributions. We used the Widely Applicable Information Criteria, WAIC \citep[and reference therein]{Hooten2015} to chose the best model. WAIC is based on the posterior predictive distribution (the distribution to predict new data) and is valid for hierarchical models. However, it assumes that the data are independent given a set of parameters, which could be a problem for our spatial data. Similarly to radial growth, we compared the WAICs on the common logarithm scale:
\begin{equation} \label{eq::psi}
	\psi = \log_{10}(\Delta \text{WAIC} + 1)
\end{equation}
The best model has $ \Delta \text{WAIC} = 0 $, which implies $ \psi $ is also null for the best model. Although there are $ R^2 $ for Bayesian regression models, they cannot be compared: the explained variance can only be interpreted in the context of a single model \citep{Gelman2018}. Hence, we exclusively based our choice on WAIC. We checked the convergence of the selected model for all the parameters and species with the Gelman-Rubin statistic \citep[R-hat diagnostic]{Gelman1992}. Usually a chain with a R-hat larger than 1.05 is considered non-convergent; at convergence, R-hat should be 1.

\subsubsection{Fecundity}
The forest inventory data have few records of trees with a dbh smaller than 10 cm, thus we could not parameterise the fecundity function $ F(s, \s) $. Instead, we used the fecundity function defined and parameterised in \citet{Purves2008}:
\begin{align} \label{eq::fecundity_fct}
	F_{\text{Purves}}(s, \s) = \F \times \A(s, \s) && \F = 0.0071
\end{align}
where $ \F $ is the number of seeds produced per sun-exposed tree crown area per unit time (Tab. \ref{tab::notations}). In this particular case, we defined a new quantity $ \rho_0 $ as the net reproduction rate $ R_0 $ with the reproduction function $ F_{\text{Purves}} $. The notation $ R_0 $ is strictly reserved to the general case, where the fecundity function is not restricted to the function of Purves.

By definition, $ \rho_0 $ corresponds to the net reproduction rate when the fecundity function is independent of climate (\ie spatially constant). Once $ \rho_0 $ is calculated over the landscape $ \Omega $, we associate a convenient normalised quantity:
\begin{equation} \label{eq::rho0_scaled}
	\tilde \rho_0(x, \s) = \frac{\rho_0(x, \s) - \min_{x \in \Omega} \big(\rho_0(x, \s) \big)}{\max_{x \in \Omega} \big(\rho_0(x, \s) \big) - \min_{x \in \Omega} \big(\rho_0(x, \s) \big)}
\end{equation}
This cancels the value of $ \F $, which was a difficult parameter to estimate in \citet{Purves2008}, and bounds $ \tilde \rho_0 $ between $ 0 $ and $ 1 $.

\subsection{Occurrence probability}
We evaluated for each species the correlation between $ \tilde \rho_0 $ and the probability of occurrence $ P_{occ} $ derived from a random forest \citep[R package]{randomForest}. This is a way to transform discrete presence and absence data into probabilities which are continuous data. It has been shown that random forests perform well in predicting tree species distributions \citep{Prasad2006}. Using a continuous probability rather than the presence and absence data can be useful in the case where climatic conditions are favourable but species are absent because of stochasticity, and alternatively where species are present but should not occur.

We trained the algorithm with coordinates of sample plots data set where there is at least one species recorded. We associated a 0 to each couple of coordinates--species if there were no record for that species in $ (x, y) $ after 1996, and 1 otherwise. We set the random forest with $ 2000 $ trees, and $ 12 $ predictors (over 19) to be chosen at each tree node. More trees add precision and bound the generalisation error (that is, the true error of the population as opposed to the training error only), without overfitting the data \citep{Prasad2006}. The explanatory variables set contained 19 bioclimatic variables (Supporting Information \ref{app::database} for the variables and \ref{app::randomForest} for the equation). The performance of the random forest was evaluated with the $ R^2 $ from \citet{Tjur2009}, later denoted by $ R_{\text{Tjur}}^2 $.

\subsection{Species distribution maps}
We downloaded an expert range map $ \Omega_j $ \citep{Prasad2003, Little1971} for each species $ j $, and calculated $ \tilde \rho_0 $ for each location within $ \Omega_j $ for an averaged climate over 5 years (from 2006 to 2010) and a canopy height $ \s $ of either $ 0 $ m (no competition) or $ 10 $ m (the average of the distribution of $ \s $ of the database is $ 9.6 $ m). Hereafter, we drop the index $ j $ and only use $ \Omega $, although the expert range maps are species-specific.

We investigated within the bounding box of our data (orange rectangle in Fig. \ref{fig::mapDatabase}), how $ \tilde \rho_0 $ relates to the orthodromic distance from the closest edge of the distribution defined by \citet{Little1971}, and mapped the variations of $ \tilde \rho_0 $. In order to avoid spatial extrapolations of the demographic rates, we decided to limit our study to the region of parameterisation (for instance \textit{Betula papyrifera} goes up to Alaska). To evaluate in which direction $ \tilde \rho_0 $ increases, we computed its gradient for each pixel of $ \Omega $ using the algorithm of \citet{Ritter1987}. Then, we separated $ \Omega $ into two regions---northern and southern to the centroid of $ \Omega $---and we averaged the gradients for each region. If $ \tilde \rho_0 $ decreases towards range limits, then both gradients should point toward the centre of the distribution. That is to say, the gradient of the northern region should point southward, and the gradient of the southern region should point northward. \\

We must integrate up to infinite to compute $ \tilde \rho_0 $ (see equations \eqref{eq::R0sol} and \eqref{eq::rho0_scaled}), but we found numerically that the integral beyond a height of 45 meters is negligible (because $ \lim_{\text{s} \to \infty} G(s) = 0 $ and $ \mu > 0 $). To be closer to real maximum dbh and height of trees for a given location $ x \in \Omega $ (rather than arbitrarily setting a maximum height of 45 m), we determined the upper boundary $ s_{\infty}^{x} $ of the integral of $ \tilde \rho_0 $ in three steps:
\begin{enumerate}
	\item Get species-specific maximal age ($ \text{age}_{\max} $) and maximal dbh ($ \text{dbh}_{\max} $) from \citet{Burns1990, Burns1990a}. The data are in Tab. \ref{tab::database}, Supporting Information \ref{app::database}
	\item Compute which dbh a tree would have at the location $ x $ with the associated climate $ \text{clim}_x $ if it were spending all its life in the overstorey up to the maximal age (using ODE45 from Matlab):
	\[
		s_{\max}^{x} = \int_0^{\text{age}_{\max}} G^{\text{(overstorey)}}\big(s(t), \text{clim}_x \big) \, dt
	\]
	\item Set the local `infinite dbh':
	\begin{equation} \label{eq::s_inf}
		s_{\infty}^{x} = \min(\text{dbh}_{\max}, \text{s}_{\max}^{x})
	\end{equation}
	and replace $ \infty $ in equation \eqref{eq::R0sol} by $ s_{\infty}^{x} $. If $ s_{\infty} < \s_c $, which happen when trees cannot reach the canopy before reaching their maximal age, we set $ R_0 $ to 0.
\end{enumerate}

Therefore, equation \eqref{eq::rho0} combines the equations \eqref{eq::R0sol}, \eqref{eq::fecundity_fct}, and \eqref{eq::s_inf} to measure species performance:
\begin{equation} \label{eq::rho0}
	\rho_0(x, \s_c) = \exp \left[-\int_0^{\s_c}\frac{\mu(s, \s_c, x)}{G(s, \s_c, x)} \, ds \right] \times \int_{\s_c}^{s_{\infty}^{x}} \frac{\F \A(s, \s_c)}{G(s, \s_c, x)} \exp \left[-\int_{\s_c}^{s} \frac{\mu(\sigma, \s_c, x)}{G(\sigma, \s_c, x)} \, d\sigma \right] \, ds
\end{equation}
from which we can derive $ \tilde \rho_0 $ (using equation \eqref{eq::rho0_scaled}). Equation \eqref{eq::rho0} accounts for spatial differences in the maximum dbh that are not included in the radial growth model. Its major difference with \eqref{eq::R0sol} is the upper bound of the expected production of offsprings, which makes equation \eqref{eq::rho0} specific to this paper, while equation \eqref{eq::R0sol} is a generalisation of \citet{Purves2009}.

\section{Results}

\subsection{Competition and climate effect on tree demography}

We found that individual growth and mortality are weakly related to climate, and that individual tree size and competition are predominant drivers of their variation (Fig. \ref{fig::delta_aic} and Fig. \ref{fig::delta_waic}). The best model for tree growth takes into account the plot random effect, and the year within plot, regardless of species. The conditional $ R_c^2 $ was much higher than the marginal $ R_m^2 $ (estimated with the package MuMIn \citep{MuMIn}), indicating that the random structure, and more specifically that local conditions, dominates explanatory variables. The best explanatory climatic variables were the annual mean temperature and the annual precipitation. However, according to the calculated ratio $ \varphi $ (equation \ref{eq::ratio}) and $ R^2 $, climate explain little variation of individual tree growth compared to tree dbh (\ref{fig::delta_aic} and Fig. Tab. \ref{tab::acsa_fixeff}).

\begin{figure}
	\centering
	\input{graphs/figure_deltaAICc}
	\caption{Performance $ \varphi $ (equation \ref{eq::ratio}) of four radial growth models (the closer to zero, the better). The black dots represent the 14 species, and the orange dots are the average for each model. The model "climate" is the 2\textsuperscript{nd} order polynomial containing $ T_a $ and $ P_a $, the model "competition" has only the canopy status explanatory variable, and the model "dbh" is the 2\textsuperscript{nd} order polynomial containing dbh (Supporting Information \ref{app::glmm}, equations \eqref{eq::model7}, \eqref{eq::model18}, \eqref{eq::model19}, and \eqref{eq::model22} respectively). The selected model corresponds to the best model under the constraint $ \text{VIF} < 20 $. The size variable (dbh) alone explains in average $ 3.84 $ times more than the climate variables alone. All species have the same selected model (\ie best model under the constraint $ \text{VIF} < 20 $), thus the black dots are hidden behind the orange dot for the last column. \label{fig::delta_aic}}
\end{figure}

Our analysis revealed that growth is higher in the overstorey for all species (Fig. \ref{fig::over_under}). Moreover, we found that the response to overstorey competition corresponds to the shade tolerance: individual growth of shade tolerant species is less responsive from being in the overstorey (Fig. \ref{fig::groups}, and Tab. \ref{tab::cs} in Supporting Information \ref{app::glmm}).
\begin{figure}
\begin{subfigure}[t]{.48\textwidth}
	\centering
	\input{graphs/G_over-under-storey_averaged}
	\caption{Overstorey versus understorey growth (in mm) of the 14 parameterised species for an averaged individual (\ie all the explanatory variables of equation \eqref{eq::glmm_growth} are set to the average). The line is the identity function.}
	\label{fig::over_under}
\end{subfigure}
\hfill
\begin{subfigure}[t]{.48\textwidth}
	\centering
	\input{graphs/groups}
	\caption{Response of species-specific radial growth $ G $ to light, grouped by three levels of shade tolerance: Low (L), Medium (M), and High (H). Growth of species reaching the canopy increases much more for shade-intolerant species than tolerant. See Tab. \ref{tab::cs} for the parameters in Supporting Information \ref{app::glmm}.}
	\label{fig::groups}
\end{subfigure}
\caption{Effect of competition on individual tree growth.}
\label{fig::growthResults}
\end{figure}

For mortality, the best model for all the species accounted for the lowest annual temperature ($ T_m $) and the three contiguous driest months ($ P_d $). Diameter at breast height, competition, and climate explain the best mortality when combined, but are equivalent when taken separately, which differ from growth (Fig. \ref{fig::delta_waic}). All the chains converged, regardless of parameters and species (R-hat histogram in Fig. \ref{fig::rhat_conv}. 
\begin{figure}
	\centering
	\input{graphs/figure_deltaWAIC}
	\caption{Performance $ \psi = \log_{10}(\Delta \text{WAIC} + 1) $ (equation \ref{eq::psi}) of four mortality models (the closer to zero, the better). The black dots represent the 14 species, and the orange dots are the average for each model. The model "climate" is the 2\textsuperscript{nd} order polynomial containing $ T_m $ and $ P_d $, the model "competition" has only the canopy status explanatory variable, and the model "dbh" is the 2\textsuperscript{nd} order polynomial containing dbh (Supporting Information \ref{app::glmm}, equations \eqref{eq::model_mu9}, \eqref{eq::model_mu10}, and \eqref{eq::model_mu13} respectively). The selected model corresponds to the best model (equation \eqref{eq::model_mu7}, Supporting Information \ref{app::glmm}). All species have the same best model, thus the black dots are hidden by the orange dot for the last column. \label{fig::delta_waic}}
\end{figure}
Mortality is larger in the understorey for all species except for \textit{Tsuga canadensis} and \textit{Fagus grandifolia}. Low shade tolerant species responded more negatively to competition than highly tolerant species (Fig. \ref{fig::over_under_mu} and \ref{fig::groups_mu}).
\begin{figure}
\begin{subfigure}[t]{.48\textwidth}
	\centering
	\input{graphs/M_over-under-storey_averaged}
	\caption{Overstorey versus understorey mortality of the 14 parameterised species for an averaged individual (\ie all the explanatory variables of equation \eqref{eq::glmm_mortality} are set to the average). The line is the identity function.}
	\label{fig::over_under_mu}
\end{subfigure}
\hfill
\begin{subfigure}[t]{.48\textwidth}
	\centering
	\input{graphs/groups_mortality}
	\caption{Response of species-specific mortality to light, grouped by three levels of shade tolerance: Low (L), Medium (M), and High (H). Mortality of species reaching the canopy decreases much more for shade-intolerant species than tolerant. See Tab. \ref{tab::cs} in Supporting Information \ref{app::glmm}.}
	\label{fig::groups_mu}
\end{subfigure}
\caption{Effect of climate and competition on individual tree mortality.}
\label{fig::mortalityResults}
\end{figure}

There is considerable uncertainty in parameters' estimate for both rates (Figs. \ref{fig::12speciesG_dbh} and \ref{fig::12speciesM_dbh} for individual growth and mortality respectively). More specifically, the mortality functions of climate of \textit{Abies balsamea, Betula papyrifera, Fagus grandifolia, Picea rubens, Pinus strobus} and \textit{Populus tremuloides} are bell-shaped or flat curves, but could be U-shaped curves (which is more expected \citep{Lines2010}) according to the posterior distribution of their regression coefficients. \textit{Abies balsamea} and \textit{Populus tremuloides} are the two exceptions as they compensate for their negative response of mortality to dbh\textsuperscript{2} by high slopes related to dbh. The effect of competition is significant for almost all species and interact with temperature. However, interactions of competition and precipitation are mostly non-significant for both vital rates. Thus, the advantage of having access to more water resources by large trees might be compensated by their higher demand of water (Supporting Information \ref{app::confInt} and XYZxyz---csv file).

\subsection{Relation of demography and competition to tree distribution} % tab::corr_R0_presAbsData
The $ \tilde \rho_0 $ equation \eqref{eq::rho0} relates population performance to the individual demography estimated in the previous section. Demographic performance $ \tilde \rho_0 $ varies significantly (Fig. \ref{fig::grad_cols} and maps in Supporting Information \ref{app::maps}) but there is no systematic relationship across ranges. On the one hand, $ \tilde \rho_0 $ of six species \textit{Acer saccharum, Acer rubrum, Picea glauca, Pinus strobus, Thuja occidentalis} and \textit{Tsuga canadensis} increases from south- east to north-west (almost south-north for \textit{Acer rubrum}). On the other hand, it increases towards the south for five species \textit{Abies balsamea, Betula papyrifera, Picea mariana, Picea rubens} and \textit{Populus tremuloides}. Note that excepted \textit{Picea mariana}, the four other species have a negative slope associated to the quadratic dbh term of the mortality model (parameter $ \beta_j^{(7)} $ in equation \eqref{eq::glmm_mortality}). The three remaining species (\textit{Betula alleghaniensis, Fagus grandifolia} and \textit{Picea banksiana}) do no have any clear direction within their respective distribution $ \Omega $. The correlations between $ \tilde \rho_0 $ and the orthodromic distance to the closest edge of $ \Omega $ are negative for most of the species (Fig. \ref{fig::3correls_dist} and Tab. \ref{tab::R0correl_dist}), which corroborates that $ \tilde \rho_0 $ is not higher in the centre of species distributions.

The lifetime number of recruits per individual $ \tilde \rho_0 $ aggregates the three vital rates into a species-specific performance measure. We computed correlations between occurrence probabilities $ P_{\text{occ}} $ and individual growth and mortality to disentangle their effects upon $ \tilde \rho_0 $. Overall, we found no rule relating occurrence probabilities with demography. When $ P_{\text{occ}} $ is positively correlated with individual growth, and negatively correlated with mortality, then the correlation of $ P_{\text{occ}} $ with $ \tilde \rho_0 $ is also positive (\eg \textit{Betula papyrifera}, Fig. \ref{fig::demog_Pocc5-8}). In this case, $ P_{\text{occ}} $ correlations with $ \tilde \rho_0 $ are consistently higher than $ P_{\text{occ}} $ correlations with demography (Supporting Information \ref{app::randomForest}, Figs. \ref{fig::demog_Pocc1-4}--\ref{fig::demog_Pocc13-14}). However, when $ P_{\text{occ}} $ is (counter-intuitively) positively correlated to mortality rates, then the correlation of $ P_{\text{occ}} $ with $ \tilde \rho_0 $ drops or even becomes negative (\eg \textit{Fagus grandigolia}, Fig. \ref{fig::demog_Pocc5-8}). These results demonstrate there might be compensatory strategies (such as a higher recruitment than our constant fecundity function) which could explain why $ P_{\text{occ}} $ is positively correlated to mortality.
\begin{figure}[htb]
    \centering
	%% First row
	\begin{subfigure}{0.25\textwidth}
		\input{graphs/azimuth_ABI-BAL}
		\caption{\textit{Abies balsamea}}
		\label{fig::abibal_az}
	\end{subfigure}
	\hfil
	\begin{subfigure}{0.25\textwidth}
		\input{graphs/azimuth_ACE-RUB}
		\caption{\textit{Acer rubrum}}
		\label{fig::acerub_az}
	\end{subfigure}
	\hfil
	\begin{subfigure}{0.25\textwidth}
		\input{graphs/azimuth_ACE-SAC}
		\caption{\textit{Acer saccharum}}
		\label{fig::acesac_az}
	\end{subfigure}
	\medskip
	%% Second row
	\begin{subfigure}{0.25\textwidth}
		\input{graphs/azimuth_BET-ALL}
		\caption{\textit{Betula alleghaniensis}}
		\label{fig::betall_az}
	\end{subfigure}
	\hfil
	\begin{subfigure}{0.25\textwidth}
		\input{graphs/azimuth_BET-PAP}
		\caption{\textit{Betula papyrifera}}
		\label{fig::betpap_az}
	\end{subfigure}
	\hfil
	\begin{subfigure}{0.25\textwidth}
		\input{graphs/azimuth_FAG-GRA}
		\caption{\textit{Fagus grandifolia}}
		\label{fig::faggran_az}
	\end{subfigure}
	\medskip
	%% Third row
	\begin{subfigure}{0.25\textwidth}
		\input{graphs/azimuth_PIC-GLA}
		\caption{\textit{Picea glauca}}
		\label{fig::picgla_az}
	\end{subfigure}
	\hfil
	\begin{subfigure}{0.25\textwidth}
		\input{graphs/azimuth_PIC-MAR}
		\caption{\textit{Picea mariana}}
		\label{fig::picmar_az}
	\end{subfigure}
	\hfil
	\begin{subfigure}{0.25\textwidth}
		\input{graphs/azimuth_PIC-RUB}
		\caption{\textit{Picea rubens}}
		\label{fig::picrub_az}
	\end{subfigure}
	\medskip
	%% Forth row
	\begin{subfigure}{0.25\textwidth}
		\input{graphs/azimuth_PIN-BAN}
		\caption{\textit{Pinus banksiana}}
		\label{fig::pinban_az}
	\end{subfigure}
	\hfil
	\begin{subfigure}{0.25\textwidth}
		\input{graphs/azimuth_PIN-STR}
		\caption{\textit{Pinus strobus}}
		\label{fig::pinstr_az}
	\end{subfigure}
	\hfil
	\begin{subfigure}{0.25\textwidth}
		\input{graphs/azimuth_POP-TRE}
		\caption{\textit{Populus tremuloides}}
		\label{fig::poptre_az}
	\end{subfigure}
	\medskip
	%% Fifth row
	\begin{subfigure}{0.25\textwidth}
		\input{graphs/azimuth_THU-OCC}
		\caption{\textit{Thuja occidentalis}}
		\label{fig::thuocc_az}
	\end{subfigure}
	\hfil
	\begin{subfigure}{0.25\textwidth}
		\input{graphs/azimuth_TSU-CAN}
		\caption{\textit{Tsuga canadensis}}
		\label{fig::tsucan_az}
	\end{subfigure}
	\hfil
	\begin{subfigure}{0.25\textwidth}
		\input{graphs/legend_cols}
	\end{subfigure}
	\caption{Species-specific averaged direction of increase of $ \tilde \rho_0 $ for the northern region (blue arrows) and southern region (orange arrows). The black square represents the species-specific centroid of the distribution $ \Omega $, and is the reference point to define the northern region (everything north to the centroid) and the southern region. If population performances were higher in the center of the distribution, then the arrows would point towards the centroid. The azimuths of the averaged directions are in Tab. \ref{tab::azimuth} and are mapped with the centroids in their corresponding geographical space (Supporting Information \ref{app::maps}) \label{fig::grad_cols}}
\end{figure}

The correlation between $ \tilde \rho_0 $ and the probability of occurrence $ P_{\text{occ}} $ did not show any trend in absence of competition (varies from $ - 0.49 $ to $ 0.60 $), neither it did with a canopy height of $ 10 $ meters (ranges within $ - 0.51 $ to $ 0.47 $). The correlation decreases with competition for most boreal species (\textit{Betula papyrifera, Picea glauca, Picea mariana, Picea rubens, Pinus banksiana} and \textit{Thuja occidentalis}, but not \textit{Abies balsamea}). For the non-boreal shade-intolerant species (\textit{Populus tremuloides}), accounting for competition increased the correlation between $ \tilde \rho_0 $, while the correlations for the non-boreal shade-tolerant species (\textit{Acer rubrum, Acer saccharum, Betula alleghaniensis, Fagus grandifolia, Pinus strobus} and \textit{Tsuga canadensis}) are similar with and without competition (Fig. \ref{fig::3correls}). The SDMs based on random forests described accurately occurrence probabilities for all species ($ R_{\text{Tjur}}^2 $ ranging from $ 0.74 $ to $ 0.85 $). We controlled that using the SDMs rather than the presence and absence data did not change any trend in the correlation results (Fig. \ref{fig::correl_rf_vs_presAbs}).
\begin{figure}
	\centering
	\input{graphs/3correlations}
	\caption{Species-specific correlations of $ \tilde \rho_0 $ and the SDM without competition (\MoveUp), or with competition (canopy height $ \s = 10 $ m, \CircSteel). The three colours correspond to the shade tolerance level (the darker, the more shade-tolerant). The values can be found in Tab \ref{tab::R0correlSDM}. On this figure, we sorted the data by increasing order of correlation with competition (\CircSteel) rather than alphabetical order. \label{fig::3correls}}
\end{figure}


\section{Discussion}
We developed a model to investigate how climate and competition determine continental-scale variation in tree demography. We modelled variation in radial growth and mortality and, when combined with values for the effective fecundity from the literature, we derived the population growth rate $ R_0 $ for size-structured populations. We re-described $ R_0 $ as $ \rho_0 $ when using the fecundity function of \citet{Purves2008} and scaled it accordingly (\ie $ \tilde \rho_0 $) to emphasise spatial variation in demography under constant fecundity. We then correlated $ \tilde \rho_0 $ with occurrence probabilities where competition was absent or present, and found marked variation among species and competition levels, which ranged from negative to positive. Our method advances previous analyses of ontogenic growth \citep{McGill2012, Thuiller2014}, by including explicit representations of the complex history of forest stands (\ie tree cohorts), the abiotic environment, and species interactions. These three mechanisms commonly shape species responses to climate change, together with dispersal, evolution and physiology \citep{Urban2016}. Yet, adding demography, environment, and species interactions for size-structured population models comes with trade-offs: more detailed models are data-intensive and might require specific information, while increasing the complexity of parameter estimations. Our approach was computationally challenging, given that it combines \num{3816854} tree measurements to climate data and competition (computed using \num{7704442} measurements). Finally, we tested whether species performance declines towards species range limits, while accounting for competition, and found no support for this hypothesis for most species. Overall, our results demonstrate an extreme variability in growth and mortality rates and the difficulties that relate tree demography to species distributions.

\subsection{From climate to occurrence through demography}
\subsubsection{Climate effect on radial growth and mortality}
Both individual growth and mortality were highly variable along the broad investigated climatic gradient. There is a long tradition in dendrochonological studies (\eg \citet{Aussenac2017}) of relating variation in annual growth of individual trees to inter-annual variation in climate. Yet, it appears that the effect of climate on demography is more difficult to detect when comparing the average performance of individual trees across a large biogeographic area, especially when we integrate growth and mortality over the entire life cycle of a forest stand. Many other factors may condition forest dynamics beyond the effect of climate \citep{Zhang2015}. Tree demography is likely to be a high-dimensional process \citep{Clark2011} that is affected by several individual level constraints, such as genetics, soil properties, micro-topography, forest composition, pests, and tree and stand history. Tree mortality is also a slow and cumulative process that may be difficult to represent with discrete information such as stem diameter, average climate, and neighbourhood composition. Models such as the ones that we investigated could not properly account for external causes of death, such as physical damage \citep[uprooting, stem breakage, crushing by other falling trees]{Larson2010} or dieback. Our observations that plot random effects are large relative to the effects of climatic variables is good evidence that this individual/site level variation is driving much of the uncertainty in tree demography across the climatic gradient. Including all of these extra variables in growth and mortality models (provided we have proper information and functions to represent their effects) would certainly aid in reducing the uncertainty in the functions that relate demography to climate. That being said, individual conditions such as past history and micro-site properties are highly variable and unpredictable in nature. This means that even though we could improve further growth and mortality models by trimming residual variation, the uncertainty in some of the conditioning variables would propagate and still make demography highly variable across climatic gradients. In other words, the stochastic variation in individual conditions is simply overwhelming the deterministic effect that climate alone exerts on growth and mortality, thereby precluding any significant effect of demography on species distributions.

\subsubsection{Competition effect on radial growth and mortality}
Despite the substantial variability that we observed, we found that radial growth and mortality are strongly influenced by asymmetrical competition for light. We have found a relationship between shade tolerance and the effect of light competition for both growth ($ G $) and mortality ($ \mu $). It suggests that not only understorey tree survival, but also understorey tree growth could help to quantify tree species competitive ability, although this analysis should be approached with caution \citep{Feng2018}. Notwithstanding that the response of growth to climate being mediated by the canopy status, we did not further explore the interaction between climate and light competition for our 14 species. We already know that the growth response of these species to adverse climate conditions is buffered by neighbourhood interspecific competition \citep{Aussenac2019}. Understanding the effect that light competition plays in tree demography is, therefore, of primary importance and requires dedicated studies, given that light and water availability act non-linearly on plant performance \citep{Holmgren2012}. The dominance of size and competition effects over climate in our models demonstrates the importance of considering population structure in forest dynamics. Our derivation of an integrated measure of performance across the life cycle is needed to properly investigate the variation in demography across large areas. \\

Our two-state canopy model cannot compare with a ten-layer model as proposed by \citet{Lischke1998}, which might explain why competition had little effect on correlations between probabilities of occurrence and species performance. However, two layers were sufficient to detect a joint effect of climate and competition for light on species distributions. In the absence of competition, we found that every species exhibited positive population growth rates across most of their ranges (Table S5.3). When adding competition (canopy height $ \s_c = 10 $ m), boreal species had a substantial reduction in the proportion of the range with positive growth, from $ 0.01 \% $ (\textit{Thuja occidentalis}) to $ 99 \% $ (\textit{Pinus banksiana}). It is possible that we underestimated $ \tilde \rho_0 $ in northern locations because we had set a standard canopy height of $ 10 $ m across regions, and we did not consider variation in the allometric relationships that we used. Among the other species, \textit{Populus tremuloides} lost $ 66 \% $ of suitable locations within its range $ \Omega $ with the addition of competition, while \textit{Acer rubrum} lost $ 28 \% $ (see Supporting Information Appendix S5). \textit{Populus tremuloides} is classified as very shade-intolerant \citep{Burns1990a}; thus, declines in its growth due to competition could be expected, as recruitment does not occur continuously in the understorey. We did not include disturbances in our model, but these may play a major role in maintaining this pioneer species in Eastern Canada \citep{Nlungu-Kweta2017}. Population growth rates were positive for all the other species within their distributions, regardless of the presence or absence of competition. This response supports previous studies showing that species geographic distributions ($ \Omega $) are within species' ecological niches \citep{Lee-Yaw2016, Csergo2017}.

\subsubsection{Relationship between demography and distribution}
The net reproduction rate $ \rho_0 $ (and equivalently, $ \tilde \rho_0 $) is a heuristic tool to summarise how individual growth, mortality, and seed production collectively define species persistence. We found four drivers that we can alter to change $ \rho_0 $: the competition, the average understorey growth, the average mortality rate, and the fecundity function. The combination of these four drivers can represent the growth-survival trade-off (\eg the ratio of mortality over radial growth $ \sfrac{\mu}{G} $ appearing in equation \eqref{eq::rho0}) and the stature-recruitment trade-off (distinguishes long-lived pioneers from short-lived breeders). These two trade-offs are important for understanding forest dynamics \citep[for tropical forest]{Ruger2020}. \\

We found little support for the hypothesis that tree species should be distributed where they perform the best: there are very low positive correlations between $ \tilde \rho_0 $ and occurrence probabilities. We compared the effects of using random forests rather than presence/absence data, but found no difference in the trends (Supporting Information Appendix S6). Given the high $ R_{\text{Tjur}}^{2} $, we cannot attribute the lack of correlation to the random forest algorithm used to smooth the occurrence probabilities for the different tree species across their distributions. Our study adds to a growing body of literature on this subject \citep[and references therein]{Holt2020}. For instance, a similar lack of correlation has been found for European trees \citep[using matrix projection models]{Csergo2017} and for western North American species along moisture gradients \citep{Bohner2019}. Local interactions are hypothesised to preclude SDMs from predicting population growth rates when fitted to macroclimate data \citep{Csergo2017}. Although competition for light is important, our results show that there might be other mechanisms underlying abundance or population growth rates. For example, certain species have undergone negative density-dependence \citep[for rare species]{Yenni2012}, while certain common species are limited by plant-soil feedbacks \citep[\textit{Acer saccharum}]{Solarik2019}. \\

Competition influenced correlations of the population growth rate $ \tilde \rho_0 $ with the probability of occurrence ($ P_{occ} $) and with distance to the closest edge (Figs. \ref{fig::3correls}; Supporting Information Appendix S6). For all the boreal species, except for \textit{Abies balsamea}, competition attenuated the signal between $ P_{occ} $ and $ \tilde \rho_0 $ by reducing the correlation to a value closer to 0. As previously discussed, a canopy height of $ 10 $ m might be too great in comparison to what is generally found beyond a certain latitude. For the same reason, the correlations between $ \tilde \rho_0 $ and distance to the closest edge might have been masked by competition. For \textit{Betula papyrifera, Picea mariana} and \textit{Thuja occidentalis}, for example, we found support for declining performance toward species range edges, but exclusively in open canopies. Thus, competition might have an important role at the northern border of certain species through its influence on demography. The effects of competition at range edges have recently been investigated for European trees, with a different outcome: competition was a strong determinant of vital rates, but its effect was not stronger at the edge than at the centre of the distribution \citep{Kunstler2019}. In spite of this response, the authors also found a weak support for declining performance at the range edge (the Abundant Centre Hypothesis). For most species, we found that $ \tilde \rho_0 $ monotonically increased towards their northern or southern boundaries but not toward their centres (Fig. \ref{fig::grad_cols} and Supporting Information S5).


\subsection{Tree demography beyond growth and mortality}
\subsubsection{Fecundity}
It is difficult to cover the full range of a tree's life cycle and study its specific relationships with climate. Our model does not account for spatial variability in seed production, seed survival or germination, for which very little is known and documented. We used $ \tilde \rho_0 $ instead of $ \rho_0 $ to tease apart the fecundity term. This measure can be compared directly between species to observe trends in population performance within a landscape when species' fecundities are kept constant. While the threshold value allowing species persistence is always 1 when using the un-scaled population growth rates $ \rho_0 $, it becomes species-specific when using $ \tilde \rho_0 $. Our study also overlooked seedling growth and survival because no data are available for individual trees below a certain diameter. The ensemble of processes, from seed to seedling survival, describes the recruitment niche, which defines the requirements that allow a seed to germinate and establish \citep{Valdez2019}. Different studies that focus upon a single species \citep[\textit{Acer saccharum}]{Solarik2016}, or along either a longitudinal gradient \citep{CLARK2011}, or a latitudinal gradient \citep{Boisvert-Marsh2019}, provide good evidences of the role that climate plays in shaping the recruitment niche. More specifically, seed production, rather than tree growth and mortality, was found to be the most responsive to spring temperature and summer drought \citep[11 sites in the Appalachians and Piedmont of North Carolina]{CLARK2011}. Soil properties and pathogens were found to constrain the regeneration of trees even more strongly than did climate \citep[Mont M\'{e}gantic, Qu\'{e}bec]{Brown2014}. These studies corroborate the plethora of mechanisms underlying population growth rates, and underscore our need for more data on the juvenile stage, at least to avoid extrapolating demographic rates that have been estimated mostly from adult trees ($ dbh \geqslant 100 $ mm) to saplings. A sensitivity analysis of $ \tilde \rho_0 $ with respect to the three vital rates is necessary to make further process and to distinguish which demographic parameter is the most important in the context of climate change, and for which species, at which stage, and at which part of the range.

\subsubsection{Dispersal}
We posed the hypothesis of local dispersal, using a Dirac distribution (\ie the dispersion does not appear in our equation) for the sake of tractability. Therefore, $ \tilde \rho_0 $ is a quantity derived only from demographic processes, neglecting potential effects of source and sink dynamics on distribution. The role of dispersal in tree distribution is controversial: certain studies have reported that long-distance dispersal determines the migration rate of trees and can explain some species shifts \citep[and references therein]{Nathan2002}, while other studies have argued that micro-refugia (small patches beyond the main distribution) play a major role in colonisation \citep[and references therein]{Feurdean2013}. These hypotheses could be investigated numerically with our model by developing an algorithm adapted to transport equations, such as \eqref{eq::dynamics_x}. However, retrieving analytical results would requires simpler radial growth and mortality functions.

\subsection{Looking forward}
Even though $ \rho_0 $ has useful theoretical properties (Supporting Information Appendix S1), it is too early to use it in assisted migration debates and forest management. Nevertheless, we note that some species exhibited low values of $ \rho_0 $ in the southern portion of their distribution, but increased moving northward, which might be a sign of future mismatches between their niches and their distributions. This is particularly true for \textit{Acer rubrum}, which is projected to have extinction debts in the south \citep{Talluto2017a}. Species that are poorly dispersed and highly persistent might be more subject to niche-distribution mismatches \citep[on shrubs, not trees, in the family Proteaceae in South Africa]{Pagel2020}. It is, therefore, important to discern which scale is the best to understand how tree distributions emerge from forest dynamics. \\

Tree demography is a multidimensional process resulting from individual's characteristics (\eg genetics, history, size), and from local conditions that are entangled with regional processes. One way to acknowledge our incomplete understanding or the high degree of variability in demography is to use stochasticity in modelling processes, at the cost of losing the analytical population growth rate. Integral Projection Models (IPMs) are stage- or size-structured models, like ours, and are promising tools to model forest dynamics \citep{Vindenes2012, Kunstler2019, Merow2014}. IPMs can integrate stochasticity into demography, which is a major feature as chance plays a significant role in lifetime reproductive success variance and, by extension, in $ R_0 $'s variance \citep{Snyder2016, Snyder2018}.

\section{Conclusion}
Our study stresses that using climate through demographic rates is not enough to explain species distributions, even though we accounted for forest structure and competition for light via a simplified method. Therefore, climatic niche should be used circumspectly, as the underlying processes of species occurrence remain unclear. We demonstrated that climate plays either a minor or an unpredictable role in tree demography. Hence, other factors such as stochastic extinction, dispersal limitation, sink populations or Allee Effects \citep[and references therein]{Holt2005} should be investigated to understand tree range dynamics. We showed that individual processes (according to the manner in which we estimated them) contributed very little to tree distributions as well. Therefore, tree dynamics cannot rely exclusively upon demographic rates that are determined by local spatial processes; they should also include phenomena that can be perceived mostly at the meta-population scale, such as fire dynamics. Our study underlines the need to mix scales and to use integrated population modelling \citep{Isaac2020}.

% \section{Acknowledgement}
This research was enabled in part by support provided by calcul Qu\'ebec (\url{http://www.calculquebec.ca/en/}) and Compute Canada (\url{www.computecanada.ca}). This research was funded by NSERC Strategic Grant. The authors are grateful to F. Guillaume Blanchet for helpful discussions, and to William Parsons for polishing the English language of the introduction.


\end{onehalfspace}

\begin{refcontext}[sorting=nyt]
\printbibliography
\end{refcontext}

\section{Data accessibility statement}
All scripts that were used in this study and the trace plots of the Markov chains can be found on Github (\url{https://github.com/amael-ls/code_R0niche}). Most of the figures were made using the package \textit{tikzDevice} \citep{tikzDevice}. The data that support the findings of this study are available from the Forest Inventory and Analysis (FIA, USDA Forest Servive), Ministère des Forêts, de la Faune et des Parcs du Québec (MFFPQ), the Ministry of Natural Resources and Forestry of Ontario (MNRFO), the Ministry of Natural Resources of New Brunswick, and the forest products company Domtar. Restrictions apply to the availability of these data, except for FIA, which were used under license for this study. FIA data are available from \url{https://www.fia.fs.fed.us/}. Data from MFFPQ, MNRFO, the Ministry of Natural Resources of New Brunswick, and Domtar can be requested from their platforms.


\newpage

\clearpage % ensure all floats are processed
\counterwithout{figure}{section}
\processdelayedfloats
\clearpage

\counterwithin{figure}{section}
\counterwithin{table}{section}
\appendix
\addtocontents{toc}{\setcounter{tocdepth}{3}}
\tableofcontents

\section{Derivation of the net reproduction rate $ R_0 $ and some of its properties} \label{app::calc_R0}
\begin{refsection}
In this appendix, we derive the net reproduction rate $ R_0(x, \s) $ for a given location $ x $ and a canopy height $ \s $. Finally, we prove that $ R_0(x, \s) $ is a decreasing function of $ \s $ which allows us to make our interpretation in the results section. We suppose that the dispersal kernel $ \K $ is a Dirac, in other words, seeds emitted from $ x $ remain in their source patch $ x $.

\subsection{Derivation of $ R_0 $ \label{calc_R0::sec::R0}}
We use the method of characteristics in equation \eqref{eq::dynamics_x}. This method is commonly used in transport equations, like \eqref{eq::dynamics_x} describing the advection of trees at a non-constant speed  $ G $ along the size axis. Characteristics allow us to follow individuals through their life, \ie they represent the trajectories of individuals in the time-size plan (fig. \ref{fig::chara} for an example). Our transport equation requires an initial population at time $ t = 0 $ that we denote by $ \phi(s) $. It is the density of trees of size $ s $ at the beginning.

\begin{figure}[h!]
	\includegraphics{illustrations/chara}
	\caption{Characteristics for a constant competition $ \s_c = 2 $, a climatic constant $ \xi_{x, i} = 1 $, and for coefficients $ \beta_0 = \ln(3) $, $ \beta_1 = 1 $ and $ \beta_2 = -1 $. These curves are parameterised by $ \theta $ and allow us to follow a cohort with relative time $ \theta $}
	\label{fig::chara}
\end{figure}

Let the canopy height $ \s $ be known with a constant value $ \s_c $, and, let $ s $ and $ t $ be functions of a parameter $ \theta $. By applying the chain rule, we get:
\[
	\frac{d N\big(s(\theta), t(\theta) \big)}{d \theta} = \left. \frac{\partial N}{\partial s} \right|_{(s(\theta), t(\theta))} \frac{ds}{d \theta} +
			\left. \frac{\partial N}{\partial t} \right|_{(s(\theta), t(\theta))} \frac{dt}{d \theta}
\]
We therefore set the characteristic equations to:
% \begin{align}
% 	\frac{ds}{d \theta} &= G(s, \s_c) \label{eq::chara_s} \\
% 		&= \xi_{x, i}(s, \s_c) \exp\left[ -\frac{1}{2} \left( \frac{\ln(s/\phi_{opt})}{\sigma} \right)^2 \right] \nonumber \\
% 	\frac{dt}{d \theta} &= 1 \label{eq::chara_t}
% \end{align}
\begin{align}
	\frac{ds}{d \theta} &= G(s, \s_c) \label{eq::chara_s} \\
		&= \xi_{x, i} e^{\beta_0 \mathbb{1}_{[\s_c, \infty[}(s) + \beta_1 s + \beta_2 s^2} \nonumber \\
	\frac{dt}{d \theta} &= 1 \label{eq::chara_t}
\end{align}
where $ \xi_{x, i} $ is a species-specific ($ i $ index) constant depending of the climate at the location $ x $, and $ \mathbb{1}_A $ is the indicator function (\ie $ \mathbb{1}_A (s) $ equals 1 if $ s \in A $ and 0 otherwise). Hence, equation \eqref{eq::dynamics_x} is along the characteristics:
\begin{equation}
	\frac{dN}{d\theta} = - \left\{ \frac{\partial G}{\partial s}(s, \s, x) + \mu(s, \s, x) \right\} N(\theta) \label{eq::ODE_N}
\end{equation}
\begin{nota}[Abusing notations]
	I am abusing the notations $ N(\theta) $ and $ N\big( s(\theta), t(\theta) \big) $ to make equations easier to read. It would have been more correct to write
	\[
		\widetilde{N}(\theta) = N \big(s(\theta), t(\theta) \big)
	\]
\end{nota}

For the sake of readability, I drop the $ x $ in $ G $ and $ \mu $, and write $ \exp $ for the exponential function instead of $ e^x $. The solution of \eqref{eq::ODE_N} is:
\begin{equation} \label{eq::sol_ODE_N}
	N(\theta_1) = N(\theta = 0) \exp \left[-\int_0^{\theta_1} \mu(s, \s_c) + \frac{\partial G}{\partial s}(s, \s_c) \, d\theta \right]
\end{equation}
where the boundary condition $ N(\theta = 0) = N(s_{\theta = 0}, t_{\theta = 0}) $, and the coordinates $ (s(\theta_1), t(\theta_1)) $ are to be determined. We denoted $ s_{\theta = 0} $ to specify it is the origin of the characteristic but still distinguish it from $ s_0 $, the size of newborns. The time coordinate, solution of \eqref{eq::chara_t}, is denoted by $ T $:
\[
	T(\theta) = \theta + t_{\theta = 0}
\]
however, the $ i $-state coordinate $ s $ cannot be expressed in terms of elementary functions. In what follows, we assume $ \beta_2 < 0 $ which is true for all the species we parameterised. Thus, the integral
\begin{align*}
	\int  | G(s, \s_c) | \, ds &\leqslant \int e^{|\beta_0|} e^{\beta_1 s + \beta_2 s^2} \, ds \\
	&\leqslant e^{|\beta_0| - \frac{\beta_1^2}{4 \beta_2^2}} \int e^{\beta_2(s + \sfrac{\beta_1}{2\beta_2})^2} \, ds
\end{align*}
exists and equation \eqref{eq::chara_s} has a solution that we denote by $ S(\theta_1, s_{\theta = 0}, \s_c) $. This solution $ S(\theta_1, s_{\theta = 0}, \s_c) $ is the coordinate of the $ i $-state at $ \theta_1 $ along the characteristic originated in $ s_{\theta = 0} $ at $ t_{\theta = 0} $.

Let us assume that $ S $ admits an inverse, in other words, there exists a function $ \tau $ such that:
\begin{align*}
	\tau \big( S(\theta_1, s_{\theta = 0}, \s_c), s_{\theta = 0}, \s_c \big) &= \theta_1 \\
	S \big(\tau(s_1, s_{\theta = 0}, \s_c), s_{\theta = 0}, \s_c \big) &= s_1
\end{align*}
In other words, $ \tau(s_1, s_0, \s_c) $ is the time it requires to grow from $ s_0 $ to $ s_1 $ under competition $ \s_c $. Although, equation \eqref{eq::dynamics_x} cannot be solved analytically (we would need an explicit solution of $ S $ and $ \tau $), the net reproduction rate is still tractable. Indeed, equation \eqref{eq::sol_ODE_N} can be rewritten:
\begin{align*}
	N \big( s(\theta_1), t(\theta_1) \big) &= N( s_{\theta = 0}, t_{\theta = 0}) \exp \left[-\int_0^{\theta_1} \mu(s, \s_c) + \frac{\partial G}{\partial s}(s, \s_c) \, d\theta \right] \\
	&= N( s_{\theta = 0}, t_{\theta = 0}) \exp \left[-\int_{s_{\theta = 0}}^{s(\theta_1)} \left( \mu(s, \s_c) + \frac{\partial G}{\partial s}(s, \s_c) \right) \frac{d \theta}{ds} \, ds \right] \\
	&= N( s_{\theta = 0}, t_{\theta = 0}) \exp \left[-\int_{s_{\theta = 0}}^{s(\theta_1)} \left( \mu(s, \s_c) + \frac{\partial G}{\partial s}(s, \s_c) \right) \frac{1}{G(s, \s_c)} \, ds \right] \\
	&= N( s_{\theta = 0}, t_{\theta = 0}) \frac{G(s_{\theta = 0}, \s_c)}{G \big( s(\theta_1), \s_c \big)} \exp \left[-\int_{s_{\theta = 0}}^{s(\theta_1)} \frac{\mu(s, \s_c)}{G(s, \s_c)} \, ds \right] \\
\end{align*}

If a characteristic emerged at $ t_{\theta = 0} \leqslant 0 $, then it concerns the initial population (at $ t = 0 $) denoted by $ \phi(s) $:
\begin{equation}\label{eq::sol_init}
	N(S(t, s, \s_c), t) = \phi(s) \frac{G(s, \s_c)}{G\big( S(t, s, \s_c), \s_c \big)} \exp \left[ -\int_{s}^{S(t, s, \s_c)} \frac{\mu(\sigma, \s_c)}{G(\sigma, \s_c)} \, d\sigma \right]
\end{equation}
where $ S(t, s, \s_c) $ is the size of the individual at time $ t $ given at time $ 0 $ they were of size $ s $. In figure \ref{fig::chara}, the characteristics that emerged before $ t = 0 $ are the first, second and third curves.

For characteristics that emerged after time 0 (fourth and fifth curves on fig \ref{fig::chara}), we denote their birth coordinates $ (s_{\theta = 0}, t_{\theta = 0}) $ by $ (0, t_b) $; $ t_b > 0 $ stands for the time of birth.
\begin{equation}\label{eq::sol_later}
	N(s, t) = N \big( 0, t - \tau(s, 0, \s_c) \big) \frac{G(0, \s_c)}{G(s, \s_c)} \exp \left[ -\int_{0}^{s} \frac{\mu(\sigma, \s_c)}{G(\sigma, \s_c)} \, d\sigma \right]
\end{equation}

The initial population (equation \eqref{eq::sol_init}) goes extinct as $ t $ goes by. Therefore, only equation \eqref{eq::sol_later} is of interest \citep{DeRoos1997}. We now use the boundary condition \eqref{eq::recruitment_x} (once again, we drop the spatial variable $ x $ since the dispersal kernel $ \K $ is a Dirac):
\begin{align*}
	N(0, t) G(0, \s_c) &= \int_{0}^{\infty} F(s, \s_c) N(s, t) \, ds \\
		&= \int_{0}^{\infty} F(s, \s_c) N \big( 0, t - \tau(s, 0, \s_c) \big) \frac{G(0, \s_c)}{G(s, \s_c)} \exp \left[ -\int_{0}^{s} \frac{\mu(\sigma, \s_c)}{G(\sigma, \s_c)} \, d\sigma \right] \, ds
\end{align*}
Let
\[
	B(t, \s_c) = N(0, t) G(0, \s_c)
\]
the population birth rate (\ie the quantity of seedlings created at time $ t $ by a population undergoing a competition $ \s_c $). We can substitute $ B $ into the previous equation:
\begin{equation} \label{eq::popBirth}
	B(t, \s_c) = \int_{0}^{\infty} B \big(t - \tau(s, 0, \s_c), \s_c \big) \frac{F(s, \s_c)}{G(s, \s_c)} \exp \left[ -\int_{0}^{s} \frac{\mu(\sigma, \s_c)}{G(\sigma, \s_c)} \, d\sigma \right] \, ds
\end{equation}
If $ \tau $ were a known function (which is not the case here), equation \eqref{eq::popBirth} would relate the population birth rate to its past. The quantity
\[
	\frac{1}{G(s, \s_c)} B \big(t - \tau(s, 0, \s_c), \s_c \big) \exp \left[ -\int_{0}^{s} \frac{\mu(\sigma, \s_c)}{G(\sigma, \s_c)} \, d\sigma \right]
\]
is just the density of individuals born $ t - \tau $ times unit ago that survived up to a size in $ [s, s + ds] $ at time $ t $, and $ B $ is the `input flow' per unit time.

Equation \eqref{eq::popBirth} admits an equilibrium only for particular combinations of parameters $ G $, $ \mu $ and $ F $ and since there is no density dependence here ($ \s $ is set to a known value $ \s_c $), then $ B $ ultimately grows or decline exponentially \citep{DeRoos1997}. We therefore substitute the trial solution
\begin{equation} \label{eq::B_trial}
	B(t, \s_c) = B_0e^{\lambda t}
\end{equation}
into \eqref{eq::popBirth} to get:
\begin{align*}
	B_0 e^{\lambda t} &= \int_{0}^{\infty} B_0 e^{\lambda t - \lambda \tau(s, 0, \s_c)} \frac{F(s, \s_c)}{G(s, \s_c)} \exp \left[ -\int_{0}^{s} \frac{\mu(\sigma, \s_c)}{G(\sigma, \s_c)} \, d\sigma \right] \, ds \\
	&= B_0 e^{\lambda t} \int_{0}^{\infty} e^{- \lambda \tau(s, 0, \s_c)} \frac{F(s, \s_c)}{G(s, \s_c)} \exp \left[ -\int_{0}^{s} \frac{\mu(\sigma, \s_c)}{G(\sigma, \s_c)} \, d\sigma \right] \, ds \\
\end{align*}
Thus, we get:
\begin{equation} \label{eq::eigen}
	1 = \int_{0}^{\infty} e^{- \lambda \tau(s, 0, \s_c)} \frac{F(s, \s_c)}{G(s, \s_c)} \exp \left[ -\int_{0}^{s} \frac{\mu(\sigma, \s_c)}{G(\sigma, \s_c)} \, d\sigma \right] \, ds
\end{equation}
When $ \lambda $ is setted to 0, the right hand side of \eqref{eq::eigen} becomes:
\begin{equation} \label{eq::R0_app}
	\int_{0}^{\infty} \frac{F(s, \s_c)}{G(s, \s_c)} \exp \left[ -\int_{0}^{s} \frac{\mu(\sigma, \s_c)}{G(\sigma, \s_c)} \, d\sigma \right] \, ds
\end{equation}
It represents the expected number of seedlings produced by an individual through its lifespan, which is by definition the net reproduction rate $ R_0(\s_c) $ we were looking for (equation \eqref{eq::R0sol}). Given only canopy individuals can reproduce, $ \forall s < \s_c $, $ F(s, \s_c) = 0 $ and $ \forall s \geqslant \s_c $, $ F(s, \s_c) > 0 $ and therefore the lower limit of the integrals can be changed to $ \s_c $ in equation \eqref{eq::R0_app}:
\[
	R_0 = \exp \left[- \int_{0}^{\s_c} \frac{\mu(\sigma, \s_c)}{G(\sigma, \s_c)} \, d\sigma \right] \times \int_{\s_c}^{\infty} \frac{F(s, \s_c)}{G(s, \s_c)} \exp \left[ - \int_{\s_c}^{s} \frac{\mu(\sigma, \s_c)}{G(\sigma, \s_c)} \, d\sigma \right] \, ds
\]
\begin{rem}[On $ \lambda $]
	If $ \lambda = 0 $ is truly solution of \eqref{eq::eigen}, then $ R_0(\s_c) = 1 $ and the population is stable. This is consistent with \eqref{eq::B_trial} since in this case the population birth rate would be constant.
\end{rem}

The constant $ \lambda $ is called the intrinsic growth rate of the population. Let us now prove that $ \lambda > 0 $ is equivalent to a net reproduction rate $ R_0(\s_c) > 1 $. Denote by $ f $ the function:
\[
	f(\lambda) = \int_{0}^{\infty} e^{- \lambda \tau(s, 0, \s_c)} \frac{F(s, \s_c)}{G(s, \s_c)} \exp \left[ -\int_{0}^{s} \frac{\mu(\sigma, \s_c)}{G(\sigma, \s_c)} \, d\sigma \right] \, ds
\]
We have:
\begin{align*}
	f'(\lambda) &= -\lambda f(\lambda) \\
	f''(\lambda) &= +\lambda^2 f(\lambda)
\end{align*}
Given $ f $ is a positive function (fecundity and growth are positive functions), then $ f $ is a convex decreasing function on $ \R^{+} $ and its maximum is reached at $ \lambda = 0 $. Hence, if $ R_0 > 1 $:
\[
	f(\lambda) = 1
\]
admits a unique solution on $ \R^{+} $. Thus
\[
	R_0 > 1 \Leftrightarrow \lambda > 0
\]

\subsection{Proofs of the three assertions \label{app::calc_R0::sec::3asser}}
We now prove the three following assertions:
\begin{enumerate}[label=(\textit{\roman*})]
	\item $ R_0(\s_c) $ is a decreasing function
	\item $ R_0 $ is an increasing function of the average understorey growth $ \bar{G} $ and a decreasing function of the average mortality $ \bar{\mu} $
	\item $ R_0 $ is an increasing function of the average fecundity $ \bar{F} $
\end{enumerate}

\subsubsection{First assertion}
The easiest argument to prove $ R_0 $ is a decreasing function of $ \s $ is to see that all the integrand's parts are always positive. Let:
\[
	f(s_1) = \int_{s_1}^{\infty} \frac{F}{G} e^{-\int_{s_1}^s \frac{\mu}{G} \, d\sigma} \, ds
\]
For any $ s_1 \leqslant s_2 $ we have:
\begin{align*}
	R_0(s_1) - R_0(s_2) &= e^{-\int_{0}^{s_1} \frac{\mu}{G} \, ds} \left[ \left( 1 - e^{-\int_{s_1}^{s_2} \frac{\mu}{G} \, ds} \right) f(s_2) + \int_{s_1}^{s_2} \frac{\mu}{G} \, ds \right] \\
		&\geqslant 0
\end{align*}
Therefore, $ R_0 $ is a decreasing function of $ \s $.

\subsubsection{Second assertion}
The average understorey growth for a canopy height $ \s $ is defined by
\[
	\bar{G} = \frac{1}{\s} \int_0^{\s} G(s, \s) \, ds
\]
Let $ G_1 $ and $ G_2 $ two growth functions. Since the overstorey growth is not changed, we just study the ratio:
\begin{equation} \label{eq::ratio_G}
	\frac{e^{\int_0^{\s} \frac{\mu}{G_1} \, ds}}{\int_0^{\s}e^{\frac{\mu}{G_2} \, ds}}
\end{equation}
We want to find a sufficient condition on $ G_1 $ and $ G_2 $ to get this ratio larger than 1, which means the net reproduction rate $ R_0 $ increases. Equation \eqref{eq::ratio_G} can be be rewritten:
\[
	e^{\int_0^{\s} \frac{\mu (G_1 - G_2)}{G_1 G_2} \, ds} \geqslant 1 \quad \Leftrightarrow \quad \int_0^{\s} \frac{\mu (G_1 - G_2)}{G_1 G_2} \, ds \geqslant 0
\]
Given $ \mu $ and $ G $ are positive functions, we can lower bound the integrand:
\begin{equation} \label{eq::minor}
	\frac{\mu (G_1 - G_2)}{G_1 G_2} \geqslant \frac{\max(\mu)}{\min(G_1 G_2)} \mathds{1}_{G_1 < G_2} (s) \times (G_1 - G_2) + \frac{\min(\mu)}{\max(G_1 G_2)} \mathds{1}_{G_1 \geqslant G_2}(s) \times (G_1 - G_2)
\end{equation}
Define
\[
	\begin{matrix}
		k_1 = \frac{\max(\mu)}{\min(G_1 G_2)} & &
			\mathscr{D}_1 = \{ s | G_1(s) < G_2(s) \} \\
		k_2 = \frac{\min(\mu)}{\max(G_1 G_2)} & &
		 	\mathscr{D}_2 = \{ s | G_1(s) \geqslant G_2(s) \}\\
	\end{matrix}
\]
equation \eqref{eq::minor} is:
\begin{align*}
	\int_0^{\s} \frac{\mu (G_1 - G_2)}{G_1 G_2} &\geqslant
		k_1 \int_{\mathscr{D}_1} G_1  \, ds + k_2 \int_{\mathscr{D}_2} G_1 \, ds +
		k_1 \int_{\mathscr{D}_1} G_2  \, ds + k_2 \int_{\mathscr{D}_2} G_2 \, ds \\
	&\geqslant \min (k_1, k_2) \left[\int_{\mathscr{D}_1 \cup \mathscr{D}_2} G_1 \, ds + \int_{\mathscr{D}_1 \cup \mathscr{D}_2} G_2 \, ds \right]
\end{align*}
Therefore, given $ \mathscr{D}_1 \cup \mathscr{D}_2 = [0, \s] $, a sufficient condition to have the ratio defined by equation \eqref{eq::ratio_G} larger than $ 1 $ is:
\begin{equation}
	\int_0^{\s} G_1 \, ds \geqslant \int_0^{\s} G_2 \, ds,
\end{equation}
that is to say:
\[
	\bar{G}_1 \geqslant \bar{G}_2
\]
Thus, when the understorey growth increases in average, so does the net reproduction rate $ R_0 $. A similar reasoning on $ \mu $ would prove that increasing the understorey mortality induce a depleted $ R_0 $. This conclude the proof of the second assertion.

\subsubsection{Third assertion}
In this article, we used the fecundity of \citet{Purves2008} which is proportional to sun-exposed crown area $ \A $:
\[
	F(s, \s) = f \A(s, \s)
\]
where $ f $ is the new recruits per unit time per crown area.
\begin{align*}
	\frac{\partial R_0}{\partial f} &= e^{-\int_0^{\s} \frac{\mu}{G} \, ds} \int_{\s}^{\infty} \frac{A}{G} e^{-\int_{\s}^{s} \frac{\mu}{G} \, d \sigma} \, ds \\
		&\geqslant 0
\end{align*}
Thus, increasing fecundity augments $ R_0 $, which proves the third assertion.

\printbibliography[heading=subbibliography]
\end{refsection}

\section{Proofs of the relations between \citet{Purves2009} and our study} \label{app::purves2009}
\begin{refsection}
In this section, we first provide the assumptions concerning the demography in \citet{Purves2009}, and match our notations with \citeauthor{Purves2009}'s article in the table \ref{tab::notations_purves2009}. Finally, we derive \citet{Purves2009}'s formul\ae{} from our $ R_0 $ equation \eqref{eq::R0sol}.
\begin{table}
	\centering
	\caption{Link between our notations and \citet{Purves2009}'s notations. Note that \citeauthor{Purves2009} uses the flat-top version of the perfect-plasticity approximation, which implies the fecundity function to be proportional to the trunk area.}
	\label{tab::notations_purves2009}
	\begin{tabular}{@{}rll@{}}
	\toprule
	\textbf{Meaning} & \textbf{Our notation} & \textbf{\citet{Purves2009}} \\
	Species index & $ j $, but mostly dropped & $ j $ \\
	Space & $ x $ & $ R $ (not to be confound with $ R_0 $) \\
	Time & $ t $ & $ \tau $ \\
	Threshold diameter & $ \s, \, \s_c $ & $ D^{*} $ \\
	Net reprod. rate & $ R_0 (x, \s_c) $ & $ R_{0, j, R} $ \\
	Indiv. growth rate overstorey & $ G(s, \s, x), \, s \geqslant \s $ & $ G_{L, j, R} $ \\
	Indiv. growth rate understorey & $ G(s, \s, x), \, s < \s $ & $ G_{D, j, R} $ \\
	Mortality rate overstorey & $ \mu(s, \s, x), \, s \geqslant \s $ & $ \mu_{L, j, R} = \sfrac{1}{\rho_{L, j, R}} $ \\
	Mortality rate understorey & $ \mu(s, \s, x), \, s < \s $ & $ \mu_{D, j, R} = \sfrac{1}{\rho_{D, j, R}} $ \\
	Fecundity function & $ \F \A(s, \s) $ & $ F_{j, R}^{\text{capita}} \pi \sfrac{s^2}{4} $ \\
   \bottomrule
	\end{tabular}
\end{table}

\subsection{Additional assumptions to our model from \citet{Purves2009}}
We sum-up here some useful assumptions from \citet{Purves2009}. The assumptions are better described in \citet[and yes, it is in the paper of 2008]{Purves2008}
\begin{assum}[Flat-top crown] \label{assum::flat-top}
	The crown is a flat-top disc expressed at the top of the tree, which implies that the area of the crown $ \A(s, \s) $ is independent of its second argument.
\end{assum}

\begin{assum}[Demographic rates]
	The individual growth and mortality rates are step functions with the discontinuity at $ \s $:
	\[
		G_{j, R}(s, \s) =
		\begin{cases}
			G_{L, j, R} & \text{if } s \geqslant \s \\
			G_{D, j, R} & \text{otherwise}
		\end{cases}
	\]
	\[
	\mu_{j, R}(s, \s) =
		\begin{cases}
			\mu_{L, j, R} & \text{if } s \geqslant \s \\
			\mu_{D, j, R} & \text{otherwise}
		\end{cases}
	\]
	where $ G_{L, j, R}, \, G_{D, j, R}, \, \mu_{L, j, R}, \, \mu_{D, j, R} $ are estimated constants.
\end{assum}

\begin{assum}[Fecundity function]
	Due to the flat-top assumption \ref{assum::flat-top}, the fecundity is simply proportional to the cross-section of the trunk area at breast height:
	\[
		F_{j, R} (s, \s) =
		\begin{cases}
			\frac{1}{10^4} F_{j, R}^{\text{capita}} \pi s^2 & \text{if } s \geqslant \s \\
			0 & \text{otherwise}
		\end{cases}
	\]
	The factor $ \frac{1}{10^4} $ corrects for the units of $ F_{j, R}^{\text{capita}} $ (basal area per year, dividing by $ 10^4 $ makes the basal area dimensionless).
\end{assum}

\begin{assum}[Mortality versus individual growth] \label{assum::negligible}
	The mortality rate is negligible compare to both 1 and the individual growth rate:
	\begin{align}
		\mu_{L, j, R} &\ll 1 \\
		\mu_{L, j, R} &\ll G_{L, j, R}
	\end{align}
\end{assum}

Note that in his supporting information, \citet{Purves2009} integrates with respect to time rather than size. In the case trees are always in the overstorey, it is strictly equivalent:
\[
	\begin{cases}
		\frac{ds}{d \tau} = G_{L, j, R} \\
		s(0) = D^{*}
	\end{cases}
\]
which has the solution:
\[
	s(\tau) = \tau G_{L, j, R} + D^{*}
\]
This solution is for instance found in equation S11 \citep{Purves2009}.

\subsection{$ R_0 $ equation \citep[p. 1479]{Purves2009}}
Equation \eqref{eq::R0sol} is equivalent to equation S10.2 \citep[also p. 1479 in his article]{Purves2009} when $ \s_c = 0 $.
\begin{proof}
	Let $ \s_c = 0 $.
	\begin{align*}
		R_0 (x, 0) &= \overbrace{e^{-\int_0^{0}\frac{\mu(s, \s_c, x)}{G(s, \s_c, x)} \, ds}}^{= 1} \int_{0}^{\infty} \frac{1}{10^4} \frac{F_{j, R}^{\text{capita}} \pi (\sfrac{s}{2})^2}{G_{L, j, R}} e^{-\int_{0}^{s} \frac{\mu_{L, j, R}}{G_{L, j, R}} \, d\sigma} \, ds \\
			&= \frac{1}{10^4} \frac{\pi}{4} \frac{F_{j, R}^{\text{capita}}}{G_{L, j, R}} \int_{0}^{\infty} s^2 e^{-\frac{\mu_{L, j, R}}{G_{L, j, R}} s} \, ds \\
			&= \frac{1}{10^4} \frac{\pi}{2} F_{j, R}^{\text{capita}} \frac{G_{L, j, R}^2}{\mu_{L, j, R}^3} \\
			&= \frac{1}{10^4} \frac{\pi}{2} F_{j, R}^{\text{capita}} G_{L, j, R}^2 \rho_{L, j, R}^3
	\end{align*}
\end{proof}

\subsection{Proportion of the trees that make up to the canopy}
Let $ P^{\text{canop}} $ be the proportion of individuals that survive up to the canopy. The quantity $ P^{\text{canop}} $ is derived in \citet{Purves2009}'s supporting information, a little before equation S11; we can obtain the same result with the first part of equation \eqref{eq::R0sol}.
\begin{proof}
	\begin{align*}
		P^{\text{canop}} &= e^{-\int_{0}^{D^{*}} \frac{\mu}{G} \, ds} \\
			&= e^{-\frac{\mu_{D, j, R}}{G_{D, j, R}} D^{*}} \\
			&= e^{-\frac{D^{*}}{G_{D, j, R}\rho_{D, j, R}}}
	\end{align*}
\end{proof}

\subsection{$ D^{*} $ at equilibrium \citep[equation S13.2]{Purves2009}}
Let $ \hat D^{*} $ be the threshold diameter when the (monospecific) population is at equilibrium. We derive equation S13.2 \citep{Purves2009} from our equation \eqref{eq::R0sol} using \citeauthor{Purves2009}'s notations:
\begin{align*}
	R^{\text{canop}} &= e^{-\int_{0}^{D^{*}} \frac{\mu_{D, j, R}}{G_{D, j, R}} \, ds} \int_{D^{*}}^{\infty} \frac{1}{10^4} \frac{F_{j, R}^{\text{capita}} \pi (\sfrac{s}{2})^2}{G_{L, j, R}} e^{-\int_{D^{*}}^{s} \frac{\mu_{L, j, R}}{G_{L, j, R}} \, d\sigma} \, ds \\
		&= e^{\left(\frac{\mu_{L, j, R}}{G_{L, j, R}} - \frac{\mu_{D, j, R}}{G_{D, j, R}} \right) D^{*}} \frac{1}{10^4} \frac{\pi}{4 G_{L, j, R}} F_{j, R}^{\text{capita}} \int_{D^{*}}^{\infty} s^2 e^{-\frac{\mu_{L, j, R}}{G_{L, j, R}} s} \\
		&= \frac{1}{10^4} \frac{\pi}{4} F_{j, R}^{\text{capita}} e^{- \frac{\mu_{D, j, R}}{G_{D, j, R}} D^{*}} \frac{1}{\mu_{L, j, R}^3} (2 G_{L, j, R}^2 + 2 G_{L, j, R} \mu_{L, j, R} D^{*} + \mu_{L, j, R}^2 {D^{*}}^2)
\end{align*}
Using assumption \ref{assum::negligible}, we get a simplified version of $ R^{\text{canop}} $:
\begin{equation} \label{eq::RcanopApprox}
	R^{\text{canop}} \approx \frac{1}{10^4} \frac{\pi}{2} F_{j, R}^{\text{capita}} e^{- \frac{\mu_{D, j, R}}{G_{D, j, R}} D^{*}} \frac{1}{\mu_{D, j, R}^3} G_{L, j, R}^2
\end{equation}
At equilibrium, each individual replace itself once in average. Therefore, $ R^{\text{canop}} = 1 $. Solving equation \eqref{eq::RcanopApprox} at equilibrium for $ \hat D^{*} $, we get:
\[
	\hat D^{*} = G_{D, j, R} \rho_{D, j, R} \ln \left( \frac{1}{10^4} \frac{\pi}{2} F_{j, R}^{\text{capita}} G_{L, j, R}^2 \rho_{L, j, R}^3 \right) \text{, where } \frac{1}{\mu_{L, j, R}^3} = \rho_{L, j, R}^3,
\]
which is equation S13.2 \citep{Purves2009}.

% \subsection{Does assumptions \ref{assum::negligible} holds in our case?}
% In this subsection, we set the species $ j $ to \textit{Acer rubrum} (as in \citet[page 15 supporting information]{Purves2009}), and set arbitrarily the associated demographic rates:
% \begin{align*}
% 	G_{D, j, R} &= \int_{0}^{} G(s, \s_c, x) \, ds \\
% 	G_{L, j, R} &= \int_{}^{\infty} G(s, \s_c, x) \, ds \\
% 	\mu_{D, j, R} &= \int_{0}^{} \mu(s, \s_c, x) \, ds \\
% 	\mu_{L, j, R} &= \int_{}^{\infty} \mu(s, \s_c, x) \, ds
% \end{align*}
% where $ x = (77.93 \, W, 39.07 \, N) $, the closest point to the centroid of \textit{Acer rubrum}'s data where I have data. The value $ xxx $ is the diameter of a $ 12.5 \, m $ height \textit{Acer rubrum}.
\printbibliography[heading=subbibliography]
\end{refsection}

\section{Database} \label{app::database}
\subsection{Description of the data}
\begin{refsection}
We used an abbreviation for each species, that we list in Tab. \ref{tab::species}. For each species, we list their spatial range in Tab. \ref{tab::spaceRange}, and we plot the ranges of temperature and precipitation used in the individual growth and mortality functions (Fig. \ref{fig::speciesClimRange}).  We also join the 19 bioclimatic variables used in the random forest and the demography models selection (Tab. \ref{tab::bioclim}). Finally, we mapped all the data we used, from 1963 to 2010 (Fig \ref{fig::mapDatabase}).

\begin{table}[h!]
\centering
\caption{Species list, and the abbreviation used in the other tables \label{tab::species}. The last three columns are the maximal dbh, the maximal age, and the page of \citet{Burns1990, Burns1990a} at which we found the informations.}
\label{tab::database}
\begin{tabular}{rlllll}
	\toprule
	\textbf{Species} & \textbf{Scientific name} & \textbf{Vernacular name} & $ \text{\textbf{dbh}}_{\bm{\max}} $ & $ \text{\textbf{age}}_{\bm{\max}} $ & \textbf{page} \\
	\midrule
	ABI-BAL & Abies balsamea & Balsam fir & 460 & 200 & 33 \\
	ACE-RUB & Acer rubrum & Red maple & 760 & 125 & 170 \\
	ACE-SAC & Acer saccharum & Sugar maple & 910 & 350 & 202 \\
	BET-ALL & Betula alleghaniensis & Yellow birch & 760 & 300 & 302 \\
	BET-PAP & Betula papyrifera & White birch & 300 & 150 & 348 \\
	FAG-GRA & Fagus grandifolia & American beech & 510 & 366 & 660 \\
	PIC-GLA & Picea glauca & White spruce & 900 & 275 & 410 \\
	PIC-MAR & picea mariana & Black spruce & 230 & 250 & 451 \\
	PIC-RUB & picea rubens & Red spruce & 610 & 400 & 499 \\
	PIN-BAN & Pinus banksiana & Jack pine & 250 &  80 & 566 \\
	PIN-STR & Pinus strobus & Eastern white pine & 1020 & 200 & 982 \\
	POP-TRE & Populus tremuloides & Quaking aspen & 300 & 200 & 1093 \\
	THU-OCC & Thuja occidentalis & Northern white cedar & 600 & 400 & 1197 \\
	TSU-CAN & Tsuga canadensis & Eastern hemlock & 1020 & 400 & 1246 \\
	\bottomrule
\end{tabular}
\end{table}

\begin{table}[ht]
\centering
\caption{Number of individual measurements in the database, and geographical extent for each species (before cropping, see Fig. \ref{fig::mapDatabase} for the crop extent). There are \num{3816854} individual measurements in total. We used the projection system WGS 84 (EPSG:4326) for this table.}
\label{tab::spaceRange}
\begin{tabular}{lrrrrr}
	\toprule
	~ & ~ & \multicolumn{2}{c}{\textbf{Longitude}} & \multicolumn{2}{c}{\textbf{latitude}} \\
	\cmidrule(lr){3-4} \cmidrule(lr){5-6}
	\bfseries{Species} & \bfseries{$ \# $ measures} & \bfseries{min} & \bfseries{max} & \bfseries{min} & \bfseries{max} \\
	\midrule
		ABI-BAL & \num{822265} & -96.07 & -57.31 & 39.04 & 52.90 \\
		ACE-RUB & \num{469195} & -96.47 & -63.86 & 25.91 & 49.49 \\
		ACE-SAC & \num{321776} & -97.65 & -63.96 & 30.66 & 49.44 \\
		BET-ALL & \num{104249} & -95.52 & -63.87 & 31.48 & 49.38 \\
		BET-PAP & \num{291701} & -154.01 & -57.99 & 37.92 & 61.42 \\
		FAG-GRA & \num{96990} & -95.04 & -64.59 & 30.36 & 51.29 \\
		PIC-GLA & \num{104342} & -151.78 & -58.19 & 37.70 & 61.44 \\
		PIC-MAR & \num{656273} & -151.40 & -57.31 & 41.70 & 61.13 \\
		PIC-RUB & \num{126063} & -84.35 & -62.61 & 35.31 & 51.43 \\
		PIN-BAN & \num{204463} & -99.09 & -64.49 & 38.95 & 52.57 \\
		PIN-STR & \num{90002} & -96.94 & -61.95 & 31.49 & 51.50 \\
		POP-TRE & \num{310514} & -151.15 & -59.58 & 32.43 & 61.42 \\
		THU-OCC & \num{153013} & -95.79 & -64.22 & 35.20 & 50.82 \\
		TSU-CAN & \num{66008} & -91.93 & -64.70 & 31.60 & 49.57 \\
	\bottomrule
\end{tabular}
\end{table}

\begin{figure}[htb]
    \centering
	%% First row
	\begin{subfigure}{0.98\textwidth}
		\input{graphs/annual_mean_temperature-min_temperature_of_coldest_month_sp_range}
		\caption{Annual mean temperature $ T_a $ (left), and minimal temperature $ T_m $(right) range for each species. The vertical line is the average among species, and the dark circles are the species-specific mean.}
		\label{fig::annual_mean_temperature}
	\end{subfigure}
	\medskip
	%% Second row
	\begin{subfigure}{0.98\textwidth}
		\input{graphs/annual_precipitation-precipitation_of_driest_quarter_sp_range}
		\caption{Annual precipitation $ P_a $ (left), and minimal precipitation $ P_m $(right) range for each species. The vertical line is the average among species, and the dark circles are the species-specific mean.}
		\label{fig::annual_precipitation}
	\end{subfigure}
\caption{Species-specific (\subref{fig::annual_mean_temperature}) temperature ranges and (\subref{fig::annual_precipitation}) precipitation ranges. The definitions of $ T_a, \, T_m, \, P_a, \, P_m $ are in table \ref{tab::bioclim}.}
\label{fig::speciesClimRange}
\end{figure}

\begin{table}[h!]
\centering
\caption{Bioclimatic variables list \citep{McKenney2011}. The symbols used in the Supporting information \ref{app::glmm} to describe the formulae are defined here. \label{tab::bioclim}}
\begin{tabular}{p{6.75cm}p{9cm}}
	\toprule
	\textbf{Name} & \textbf{Description} \\
	\midrule
	annual mean temperature $ T_a $ & avg of mean monthly temperatures \\
	mean diurnal range $ r_T $ & avg of monthly temperature ranges \\
	isothermality $ I $ & variable 2 divided by variable 7 \\
	temperature seasonality $ T_s $ & standard deviation of monthly-mean temperature estimates expressed as a percent of their mean \\
	max temperature of warmest month $ T_M $ & highest monthly maximum temperature \\
	min temperature of coldest month $ T_m $ & lowest monthly minimum temperature \\
	temperature annual range $ T_r $ & Variable 5 minus variable 6 \\
	mean temperature of wettest quarter $ T_w $ & avg temperature during 3 wettest months \\
	mean temperature of driest quarter $ T_d $ & avg temperature during 3 driest months \\
	mean temperature of warmest quarter $ T_h $ & avg temperature during 3 warmest months \\
	mean temperature of coldest quarter $ T_c $ & avg temperature during 3 coldest months \\
	annual precipitation $ P_a $ & sum of monthly precipitation values \\
	precipitation of wettest month $ P_M $ & precipitation of the wettest month \\
	precipitation of driest month $ P_m $ & precipitation of the driest month \\
	precipitation seasonality $ P_s $ & standard deviation of the monthly precipitation estimates expressed as a percent of their mean \\
	precipitation of wettest quarter$ P_w $  & total precipitation of 3 wettest months \\
	precipitation of driest quarter $ P_d $ & total precipitation of 3 driest months \\
	precipitation of warmest quarter $ P_h $ & total precipitation of 3 warmest months \\
	precipitation of coldest quarter $ P_c $ & total precipitation of 3 coldest months \\
	\bottomrule
\end{tabular}
\end{table}

\begin{figure}[htb]
    \centering
	\includegraphics[scale=0.5]{graphs/mapDatabase}
	\caption{Map of the data we used, ranging from 1963 to 2010. Each point represents a location where there is at least one record. The orange rectangle is the bounding box of the data. We used it to crop all the maps concerning $ R_0 $. For aesthetic reasons and familiarity, the projection for this map is EPSG:4269. For all the other maps, the projection is EPSG:2163}
\label{fig::mapDatabase}
\end{figure}

\subsection{Variability in radial growth and mortality}
For each species, we separated the latitudes within three regions (Fig. \ref{fig::defineRegion}). The radial growth is described by three box plots per species with the following quantiles: $ 2.5 \% $, $ 25 \% $, the median (, $ 50 \% $), $ 75 \% $, and $ 97.5 \% $ (Figs. \ref{fig::growthVar1-3}--\ref{fig::growthVar13-14}). For the mortality, we computed species-specific proportions of death events within each region (Figs. \ref{fig::mortalityVar1-3}--\ref{fig::mortalityVar13-14}).

\begin{figure}
	\centering
	\input{appendices/selectRegion}
	\caption{We defined the three regions for the Figs. \ref{fig::growthVar1-3}--\ref{fig::mortalityVar13-14} as follow: for species-specific data, (\textit{i}) compute the centroid of the data, (\textit{ii}) find the middle point between the maximum latitude of the data and the centroid to define the northern region, and (\textit{iii}) same with the minimum latitude of the data to define the southern region. Everything else is the middle region. \label{fig::defineRegion}}
\end{figure}

\begin{figure}
	\centering
	\input{appendices/growthVar1-3}
	\caption{Variance of the growth data. The three regions were determined following the method describe by the figure \ref{fig::defineRegion}. The box plots represent the $ 2.5 \%$ quantile, the $ 25 \% $, the median ($ 50 \% $), the $ 75 \% $ quantile, and the $ 97.5 \% $ quantile. The dots are outliers. \label{fig::growthVar1-3}}
\end{figure}

\begin{figure}
	\centering
	\input{appendices/growthVar4-6}
	\caption{Variance of the growth data. The three regions were determined following the method describe by the figure \ref{fig::defineRegion}. The box plots represent the $ 2.5 \%$ quantile, the $ 25 \% $, the median ($ 50 \% $), the $ 75 \% $ quantile, and the $ 97.5 \% $ quantile. The dots are outliers. \label{fig::growthVar4-6}}
\end{figure}

\begin{figure}
	\centering
	\input{appendices/growthVar7-9}
	\caption{Variance of the growth data. The three regions were determined following the method describe by the figure \ref{fig::defineRegion}. The box plots represent the $ 2.5 \%$ quantile, the $ 25 \% $, the median ($ 50 \% $), the $ 75 \% $ quantile, and the $ 97.5 \% $ quantile. The dots are outliers. \label{fig::growthVar7-9}}
\end{figure}

\begin{figure}
	\centering
	\input{appendices/growthVar10-12}
	\caption{Variance of the growth data. The three regions were determined following the method describe by the figure \ref{fig::defineRegion}. The box plots represent the $ 2.5 \%$ quantile, the $ 25 \% $, the median ($ 50 \% $), the $ 75 \% $ quantile, and the $ 97.5 \% $ quantile. The dots are outliers. \label{fig::growthVar10-12}}
\end{figure}

\begin{figure}
	\centering
	\input{appendices/growthVar13-14}
	\caption{Variance of the growth data. The three regions were determined following the method describe by the figure \ref{fig::defineRegion}. The box plots represent the $ 2.5 \%$ quantile, the $ 25 \% $, the median ($ 50 \% $), the $ 75 \% $ quantile, and the $ 97.5 \% $ quantile. The dots are outliers. \label{fig::growthVar13-14}}
\end{figure}

\begin{figure}
	\centering
	\input{appendices/mortalityVar1-3}
	\caption{Variance of the mortality data. The three regions were determined following the method describe by the figure \ref{fig::defineRegion}. \label{fig::mortalityVar1-3}}
\end{figure}

\begin{figure}
	\centering
	\input{appendices/mortalityVar4-6}
	\caption{Variance of the mortality data. The three regions were determined following the method describe by the figure \ref{fig::defineRegion}. \label{fig::mortalityVa4-6}}
\end{figure}

\begin{figure}
	\centering
	\input{appendices/mortalityVar7-9}
	\caption{Variance of the mortality data. The three regions were determined following the method describe by the figure \ref{fig::defineRegion}. \label{fig::mortalityVa7-9}}
\end{figure}

\begin{figure}
	\centering
	\input{appendices/mortalityVar10-12}
	\caption{Variance of the mortality data. The three regions were determined following the method describe by the figure \ref{fig::defineRegion}. \label{fig::mortalityVar10-12}}
\end{figure}

\begin{figure}
	\centering
	\input{appendices/mortalityVar13-14}
	\caption{Variance of the mortality data. The three regions were determined following the method describe by the figure \ref{fig::defineRegion}. \label{fig::mortalityVar13-14}}
\end{figure}


\printbibliography[heading=subbibliography]
\end{refsection}

\section{Supplementary information of growth ($ G $), and mortality ($ \mu $) parameterisation} \label{app::glmm}
\begin{refsection}
\subsection{Selection of the individual tree growth model}
For the growth model selection, we applied a top-down four-step strategy \citep[Chapter 5]{Zuur2009}:
\begin{enumerate}
	\item Create a list of models with different random effect structures, but same fixed-effect structure
	\item Select the best model, which we call the beyond optimal model, via Restricted Maximum Likelihood (ReML)
	\item Once the optimal random structure is determined, we then run nested fixed-effect models (nested within the optimal beyond model), and compare them using Maximum Likelihood (and not ReML)
	\item The final model is then run with ReML
\end{enumerate}

We created four models, all containing the same fixed-structure (equation \eqref{eq::fixedStructureBeyond_OM} written in a `R-style', temp stands for temperature, and precip for precipitation), and differ only on the random structure. Each model corresponds to a hypothesis on the correlation structure within individuals (Tab. \ref{tab::randomStructure}). The model 1 assumes individuals from the same plot to be correlated (whatever the year of measurement), while model 2 add a temporal correlation within the plot. In model 3, we hypothesised trees to be spatially correlated (the plot), temporally correlated at large spatial scale (the year of measurement), and lastly, time-related at local spatial scale (the year within the plot). Eventually, model 4 tests whether or not the correlation structure is temporal, locally spatial dependent.

\begin{equation} \label{eq::fixedStructureBeyond_OM}
\begin{split}
\big\{cano & py\_status +{} dbh + dbh^2) \big\} \times{} \\
	\big\{ & annual\_mean\_temp + annual\_mean\_temp^2 +{} \\
	& temp\_annual\_range + temp\_annual\_range^2 +{} \\
	& mean\_temp\_of\_wettest\_quarter + mean\_temp\_of\_wettest\_quarter^2 +{} \\
	& mean\_temp\_of\_driest\_quarter + mean\_temp\_of\_driest\_quarter^2 +{} \\
	& mean\_temp\_of\_warmest\_quarter + mean\_temp\_of\_warmest\_quarter^2 +{} \\
	& mean\_temp\_of\_coldest\_quarter + mean\_temp\_of\_coldest\_quarter^2 +{} \\
	& annual\_precip + annual\_precip^2 +{} \\
	& precip\_of\_wettest\_quarter + precip\_of\_wettest\_quarter^2 +{} \\
	& precip\_of\_driest\_quarter + precip\_of\_driest\_quarter^2 +{} \\
	& precip\_of\_warmest\_quarter + precip\_of\_warmest\_quarter^2 +{} \\
	& precip\_of\_coldest\_quarter + precip\_of\_coldest\_quarter^2
			\big\}
\end{split}
\end{equation}

\begin{table}[h!]
\centering
\caption{Notations: $ x $ designates the plot effect, and $ t $ the year grouping effect. The spatial correlation groups individual by plots. The time correlation is divided into two parts: one that is spatially dependent, that is to say only individuals belonging to the same plot are time-related, and one that is spatially independent to represent a temporal correlation at large spatial scale.}
\label{tab::randomStructure}
\begin{tabular}{@{}rlp{8cm}@{}}
	\toprule
	\textbf{Model} & \textbf{hypothesis} & \textbf{Random structure} \\
	\midrule
	Model 1 & $ x $ & Spatial correlations \\
	Model 2 & $ x + x:t $ & Spatial correlations and temporal spatially-dependent correlations \\
	Model 3 & $ x + t + x:t = x + x \backslash t $ & As model 2, with a time correlation independent of space \\
	Model 4 & $ x:t $ & Spatially dependent time correlation \\
	\bottomrule
\end{tabular}
\end{table}

We ran these models for each species, we found the third model to be best for all. This ends the first and second steps of our strategy. For the third part, we first compared 16 models that have similar size effect (dbh and canopy status), and differ only by the climatic variables (see below, from equation \eqref{eq::model1} to equation \eqref{eq::model16}). We controlled for the maximum variance inflation factor (VIF, we set a limit of 20), and calculated for each model, their relative average rank (equation \eqref{eq::relMean}, and Tab \ref{tab::climSelection}).
\begin{equation} \label{eq::relMean}
	\text{rank}_{\text{rel}}^{(i)} = \dfrac{\sum_{j = 1}^{n_i} p_j^{(i)}}{n_i}
\end{equation}
Where $ i $ designates the $ i^{\text{th}} $ model, $ n_i $ the number of time this model appears (up to the number of species, or less if the VIF was above 20 for some species), and $ p_j^{(i)} $ the position of the $ i^{\text{th}} $ model, for species $ j $.

We found that in average, the annual mean temperature and annual precipitations, were the best climatic explanatory variables under the constraint of keeping a VIF lower than 20 (Tab. \ref{tab::climSelection}). Finally, for all the species, removing climatic or size variables downgrades the model. Hence, our choice to keep size and climate effects (same outcome than \textit{Acer saccharum} Tab. \ref{tab::acsa_fixeff}).

\begin{table}[h!]
\centering
\caption{Selection of the climate variables. The models are sorted by increasing rank of $ R^2 $, \ie from the best to the worst. Models' equation are written below. \textsuperscript{*} Selected model in bold. \dag models with a maximum VIF below 20. For the VIF, the mean and max are taken among the 14 species for each model.}
\label{tab::climSelection}
\begin{tabular}{@{}rcccc@{}}
	\toprule
	\textbf{Equation} & \multicolumn{2}{c}{\textbf{Average rank}} & \multicolumn{2}{c}{\textbf{Max VIF}} \\
		\cmidrule(lr){2-3} \cmidrule(lr){4-5}
		& $ R^2 $ & $ AIC_c $ & mean & max \\
	\midrule
	\ref{eq::model7} & 1.00 & 1.00 & 877.66 & 4526.05 \\
	\ref{eq::model10} & 3.43 & 3.79 & 17.09 & 82.86 \\
	\ref{eq::model3} & 4.64 & 4.64 & 17.33 & 28.95 \\
	\ref{eq::model1} & 4.71 & 4.00 & 11.01 & 24.54 \\
	\ref{eq::model4} & 5.00 & 4.57 & 24.82 & 42.31 \\
	\ref{eq::model14} & 7.14 & 6.29 & 7.00 & 20.25 \\
	\ref{eq::model16} & 8.36 & 9.29 & 56.42 & 278.20 \\
	\ref{eq::model2}\textsuperscript{*} \dag \ref{eq::model2} & \textbf{9.79} & \textbf{9.57} & \textbf{6.47} & \textbf{19.48} \\
	\ref{eq::model8} \dag & 10.00 & 9.07 & 6.15 & 19.76 \\
	\ref{eq::model11} \dag & 10.21 & 10.57 & 6.00 & 19.81 \\
	\ref{eq::model6} & 10.21 & 9.71 & 27.12 & 37.33 \\
	\ref{eq::model5} \dag & 10.36 & 10.07 & 11.02 & 19.41 \\
	\ref{eq::model9} \dag & 10.43 & 12.79 & 5.94 & 16.60 \\
	\ref{eq::model13} & 11.21 & 12.14 & 6.13 & 21.94 \\
	\ref{eq::model15} & 13.86 & 13.36 & 38.30 & 223.09 \\
	\ref{eq::model12} \dag & 15.64 & 15.14 & 4.41 & 5.72 \\
	\bottomrule
\end{tabular}
\end{table}

\begin{table}
	\centering
	\caption{Summary of the different growth models for \textit{Acer saccharum} to select the fixed effect structure of growth. We used Maximum Likelihood (ML) method to be able to compare the AICc \citep{AICcmodavg}; the selected model was then fit using Restricted Maximum Likelihood (ReML). $ T_a $, $ P_a $, and $ cs $ denote the annual temperature and precipitation, and the canopy status respectively (more details in table \ref{tab::bioclim}). \dag climatic and dbh variables have quadratic terms. $ \flat $ canopy status and climate interact. $ \sharp $ climate and dbh interact. The ratio $ \varphi $ is defined by equation \eqref{eq::ratio}. \label{tab::acsa_fixeff}}
	\begin{tabular}{@{}rclll@{}}
	\toprule
	\multicolumn{1}{c}{\textbf{Model}} & \multicolumn{1}{c}{\textbf{Equation}} & \multicolumn{1}{c}{$ \bm{\varphi} $} & \multicolumn{1}{c}{$ \bm{R^2_m} $} & \multicolumn{1}{c}{$ \bm{R^2_c} $} \\
	\midrule
		Selected model & Eq. \ref{eq::model2} & 0.00 & 0.15 & 0.44 \\
		$ T_a, P_a, cs, dbh $ \dag $ \flat \sharp $ & Eq. \ref{eq::model25} & 379.63 & 0.15 & 0.44 \\
		$ T_a, P_a, cs, dbh $ \dag $ \flat $ & Eq. \ref{eq::model24} & 478.58 & 0.14 & 0.43 \\
		$ T_a, P_a, cs, dbh $ \dag & Eq. \ref{eq::model23} & 545.07 & 0.14 & 0.43 \\
		$ dbh $ \dag & Eq. \ref{eq::model22} & 1695.19 & 0.14 & 0.45 \\
		$ dbh $ & Eq. \ref{eq::model21} & 6902.63 & 0.09 & 0.39 \\
		$ T_a, P_a, cs $ \dag $ \flat $ & Eq. \ref{eq::model20} & 7230.74 & 0.09 & 0.36 \\
		$ cs $ & Eq. \ref{eq::model19} & 7705.75 & 0.08 & 0.37 \\
		$ T_a, P_a $ \dag & Eq. \ref{eq::model18} & 16821.08 & 0.01 & 0.30 \\
		$ T_a, P_a $ & Eq. \ref{eq::model17} & 16889.54 & 0.01 & 0.30 \\
   \bottomrule
	\end{tabular}
\end{table}

We provide here the 16 models used to select the best climatic explanatory variables. They are written in a `R-style', and can be ran from the file glmm\_ReML=FALSE.R:

\begin{g}{align}
	\begin{split} \label{eq::model1}
		\text{growth} \sim 1 +{} &(1 | x) + (1 | t) + (1 | x:t) +{} \\
			& (cs + dbh + dbh^2) * (T_a + T_a^2 + T_w + T_w^2 + P_a + P_a^2 + P_h + P_h^2)
	\end{split}
	\\[2ex]
	\begin{split} \label{eq::model2}
		\text{growth} \sim 1 +{} &(1 | x) + (1 | t) + (1 | x:t) +{} \\
			& (cs + dbh + dbh^2) * (T_a + T_a^2 + P_a + P_a^2)
	\end{split}
	\\[2ex]
	\begin{split} \label{eq::model3}
		\text{growth} \sim 1 +{} &(1 | x) + (1 | t) + (1 | x:t) +{} \\
			& (cs + dbh + dbh^2) * (T_a + T_a^2 + T_w + T_w^2 + P_a + P_a^2 + P_c + P_c^2)
	\end{split}
	\\[2ex]
	\begin{split} \label{eq::model4}
		\text{growth} \sim 1 +{} &(1 | x) + (1 | t) + (1 | x:t) +{} \\
			& (cs + dbh + dbh^2) * (T_a + T_a^2 + T_d + T_d^2 + P_a + P_a^2 + P_c + P_c^2)
	\end{split}
	\\[2ex]
	\begin{split} \label{eq::model5}
		\text{growth} \sim 1 +{} &(1 | x) + (1 | t) + (1 | x:t) +{} \\
			& (cs + dbh + dbh^2) * (T_d + T_d^2 + T_w + T_w^2 + P_w + P_w^2 + P_c + P_c^2)
	\end{split}
	\\[2ex]
	\begin{split} \label{eq::model6}
		\text{growth} \sim 1 +{} &(1 | x) + (1 | t) + (1 | x:t) +{} \\
			& (cs + dbh + dbh^2) * (T_d + T_d^2 + P_a + P_a^2 + P_h + P_h^2 + P_c + P_c^2)
	\end{split}
	\\[2ex]
	\begin{split} \label{eq::model7}
		\text{growth} \sim 1 +{} &(1 | x) + (1 | t) + (1 | x:t) +{} \\
			& (cs + dbh + dbh^2) * (T_a + T_a^2 + r_T + r_T^2 + T_w + T_w^2 +T_d + T_d^2 +{} \\
			& T_h + T_h^2 + T_c + T_c^2 + P_a + P_a^2 + P_w + P_w^2 + P_d + P_d^2 + P_h + P_h^2 +{} \\
			& P_c + P_c^2)
	\end{split}
	\\[2ex]
	\begin{split} \label{eq::model8}
		\text{growth} \sim 1 +{} &(1 | x) + (1 | t) + (1 | x:t) +{} \\
			& (cs + dbh + dbh^2) * (T_h + T_h^2 + P_d + P_d^2)
	\end{split}
	\\[2ex]
	\begin{split} \label{eq::model9}
		\text{growth} \sim 1 +{} &(1 | x) + (1 | t) + (1 | x:t) +{} \\
			& (cs + dbh + dbh^2) * (T_c + T_c^2 + P_w + P_w^2)
	\end{split}
	\\[2ex]
	\begin{split} \label{eq::model10}
		\text{growth} \sim 1 +{} &(1 | x) + (1 | t) + (1 | x:t) +{} \\
			& (cs + dbh + dbh^2) * (T_h + T_h^2 + T_c + T_c^2 + P_w + P_w^2 + P_d + P_d^2)
	\end{split}
	\\[2ex]
	\begin{split} \label{eq::model11}
		\text{growth} \sim 1 +{} &(1 | x) + (1 | t) + (1 | x:t) +{} \\
			& (cs + dbh + dbh^2) * (T_h + T_h^2 + P_h + P_h^2)
	\end{split}
	\\[2ex]
	\begin{split} \label{eq::model12}
		\text{growth} \sim 1 +{} &(1 | x) + (1 | t) + (1 | x:t) +{} \\
			& (cs + dbh + dbh^2) * (T_w + T_w^2 + P_h + P_h^2)
	\end{split}
	\\[2ex]
	\begin{split} \label{eq::model13}
		\text{growth} \sim 1 +{} &(1 | x) + (1 | t) + (1 | x:t) +{} \\
			& (cs + dbh + dbh^2) * (T_h + T_h^2 + P_w + P_w^2)
	\end{split}
	\\[2ex]
	\begin{split} \label{eq::model14}
		\text{growth} \sim 1 +{} &(1 | x) + (1 | t) + (1 | x:t) +{} \\
			& (cs + dbh + dbh^2) * (T_h + T_h^2 + r_T + r_T^2 + P_h + P_h^2)
	\end{split}
	\\[2ex]
	\begin{split} \label{eq::model15}
		\text{growth} \sim 1 +{} &(1 | x) + (1 | t) + (1 | x:t) +{} \\
			& (cs + dbh + dbh^2) * (T_w + T_w^2 + P_h + P_h^2)
	\end{split}
	\\[2ex]
	\begin{split} \label{eq::model16}
		\text{growth} \sim 1 +{} &(1 | x) + (1 | t) + (1 | x:t) +{} \\
			& (cs + dbh + dbh^2) * (T_w + T_w^2 + P_w + P_w^2)
	\end{split}
\end{g}
where $ cs $ is the canopy status (boolean, true if the individual is in the understorey, false otherwise), and $ dbh $ the diameter at breast height.

The following list of equations is to evaluate the impact of climate and size on growth (which is our second objective).
\begin{g}{align}
	\begin{split} \label{eq::model17}
		\text{growth} \sim 1 +{} &(1 | x) + (1 | t) + (1 | x:t) +{} \\
			& T_a + P_a
	\end{split}
	\\[2ex]
	\begin{split} \label{eq::model18}
		\text{growth} \sim 1 +{} &(1 | x) + (1 | t) + (1 | x:t) +{} \\
			& T_a + T_a^2 + P_a + P_a^2
	\end{split}
	\\[2ex]
	\begin{split} \label{eq::model19}
		\text{growth} \sim 1 +{} &(1 | x) + (1 | t) + (1 | x:t) +{} \\
			& cs
	\end{split}
	\\[2ex]
	\begin{split} \label{eq::model20}
		\text{growth} \sim 1 +{} &(1 | x) + (1 | t) + (1 | x:t) +{} \\
			& cs * (T_a + T_a^2 + P_a + P_a^2)
	\end{split}
	\\[2ex]
	\begin{split} \label{eq::model21}
		\text{growth} \sim 1 +{} &(1 | x) + (1 | t) + (1 | x:t) +{} \\
			& dbh
	\end{split}
	\\[2ex]
	\begin{split} \label{eq::model22}
		\text{growth} \sim 1 +{} &(1 | x) + (1 | t) + (1 | x:t) +{} \\
			& dbh + dbh^2
	\end{split}
	\\[2ex]
	\begin{split} \label{eq::model23}
		\text{growth} \sim 1 +{} &(1 | x) + (1 | t) + (1 | x:t) +{} \\
			& cs + T_a + T_a^2 + P_a + P_a^2 + dbh + dbh^2
	\end{split}
	\\[2ex]
	\begin{split} \label{eq::model24}
		\text{growth} \sim 1 +{} &(1 | x) + (1 | t) + (1 | x:t) +{} \\
			& cs * (T_a + T_a^2 + P_a + P_a^2) + dbh + dbh^2
	\end{split}
	\\[2ex]
	\begin{split} \label{eq::model25}
		\text{growth} \sim 1 +{} &(1 | x) + (1 | t) + (1 | x:t) +{} \\
			& cs * (T_a + T_a^2 + P_a + P_a^2) + (P_a + P_a^2) * (dbh + dbh^2)
	\end{split}
\end{g}

\subsection{Selection of the individual tree mortality model}
For the mortality model, we first select the climatic variables, and then compare sub-models to assess the impact of climate and competition.

\begin{table}[h!]
\centering
\caption{Selection of the climate variables for mortality. The models are sorted by increasing positions, \ie from the best to the worst. Models' equation are written below. \textsuperscript{*}Selected model in bold.}
\label{tab::climSelection_mu}
\begin{tabular}{@{}rc@{}}
	\toprule
	\textbf{Equation} & \textbf{Average position} \\
	\midrule
		\ref{eq::model_mu7}\textsuperscript{*} & \textbf{2.50} \\
		\ref{eq::model_mu1} & 2.71 \\
		\ref{eq::model_mu6} & 3.57 \\
		\ref{eq::model_mu2} & 4.43 \\
		\ref{eq::model_mu5} & 4.71 \\
		\ref{eq::model_mu3} & 5.00 \\
		\ref{eq::model_mu4} & 5.07 \\
	\bottomrule
\end{tabular}
\end{table}

\begin{table}
	\centering
	\caption{Summary of the different mortality models for \textit{Acer saccharum} to select the fixed effect structure of mortality. \dag climatic and dbh variables have quadratic terms. $ \flat $ canopy status and climate interact}
	\label{tab::acsa_fixeff_mu}
	\begin{tabular}{@{}rcl@{}}
	\toprule
	\multicolumn{1}{c}{\textbf{Model}} & \multicolumn{1}{c}{\textbf{Equation}} & \multicolumn{1}{c}{$ \bm{\Delta \text{WAIC}} $} \\
	\midrule
		Selected model & \ref{eq::model_mu7} & 0.00 \\
		$ dbh $, $ T_m $, $ P_d $, $ cs $ \dag & \ref{eq::model_mu14} & 124.38 \\
		$ T_m $, $ P_d $, $ cs $ \dag $ \flat $ & \ref{eq::model_mu11} & 261.97 \\
		$ dbh $ \dag & \ref{eq::model_mu13} & 436.52 \\
		$ cs $ & \ref{eq::model_mu10} & 532.21 \\
		$ dbh $ & \ref{eq::model_mu12} & 709.34 \\
		$ T_m $, $ P_d $ \dag & \ref{eq::model_mu9} & 911.00 \\
		$ T_m $, $ P_d $ & \ref{eq::model_mu8} & 931.57 \\
   \bottomrule
	\end{tabular}
\end{table}

We provide here the 7 models used to select the best climatic explanatory variables. They are written in a `R-style', and can be ran from the file rstanarm.R (use the file selectClimaticVariables.R):
\begin{m}{align}
	\begin{split} \label{eq::model_mu1}
		\text{deltaState} \sim 1 +{} & \text{offset}\big( \log(\Delta t) \big) +{} \\
			& cs*(T_a + T_a^2 + P_a + P_a^2) + dbh + dbh^2
	\end{split}
	\\[2ex]
	\begin{split} \label{eq::model_mu2}
		\text{deltaState} \sim 1 +{} & \text{offset}\big( \log(\Delta t) \big) +{} \\
			& cs*(T_d + T_d^2 + P_d + P_d^2) + dbh + dbh^2
	\end{split}
	\\[2ex]
	\begin{split} \label{eq::model_mu3}
		\text{deltaState} \sim 1 +{} & \text{offset}\big( \log(\Delta t) \big) +{} \\
			& cs*(T_w + T_w^2 + P_w + P_w^2) + dbh + dbh^2
	\end{split}
	\\[2ex]
	\begin{split} \label{eq::model_mu4}
		\text{deltaState} \sim 1 +{} & \text{offset}\big( \log(\Delta t) \big) +{} \\
			& cs*(T_d + T_d^2 + P_m + P_m^2) + dbh + dbh^2
	\end{split}
	\\[2ex]
	\begin{split} \label{eq::model_mu5}
		\text{deltaState} \sim 1 +{} & \text{offset}\big( \log(\Delta t) \big) +{} \\
			& cs*(T_d + T_d^2 + P_a + P_a^2) + dbh + dbh^2
	\end{split}
	\\[2ex]
	\begin{split} \label{eq::model_mu6}
		\text{deltaState} \sim 1 +{} & \text{offset}\big( \log(\Delta t) \big) +{} \\
			& cs*(T_m + T_m^2 + P_m + P_m^2) + dbh + dbh^2
	\end{split}
	\\[2ex]
	\begin{split} \label{eq::model_mu7}
		\text{deltaState} \sim 1 +{} & \text{offset}\big( \log(\Delta t) \big) +{} \\
			& cs*(T_m + T_m^2 + P_d + P_d^2) + dbh + dbh^2
	\end{split}
\end{m}

The following list of equations is to evaluate the impact of climate and size on mortality (which is our second objective). They are in the file submodels.R, and are run from rstanarm.R with the option `submodels':
\begin{m}{align}
	\begin{split} \label{eq::model_mu8}
		\text{deltaState} \sim 1 +{} & \text{offset}\big( \log(\Delta t) \big) +{} \\
			& T_m + P_d
	\end{split}
	\\[2ex]
	\begin{split} \label{eq::model_mu9}
		\text{deltaState} \sim 1 +{} & \text{offset}\big( \log(\Delta t) \big) +{} \\
			& T_m + T_m^2 + P_d + P_d^2
	\end{split}
	\\[2ex]
	\begin{split} \label{eq::model_mu10}
		\text{deltaState} \sim 1 +{} & \text{offset}\big( \log(\Delta t) \big) +{} \\
			& cs
	\end{split}
	\\[2ex]
	\begin{split} \label{eq::model_mu11}
		\text{deltaState} \sim 1 +{} & \text{offset}\big( \log(\Delta t) \big) +{} \\
			& cs*(T_m + T_m^2 + P_d + P_d^2)
	\end{split}
	\\[2ex]
	\begin{split} \label{eq::model_mu12}
		\text{deltaState} \sim 1 +{} & \text{offset}\big( \log(\Delta t) \big) +{} \\
			& dbh
	\end{split}
	\\[2ex]
	\begin{split} \label{eq::model_mu13}
		\text{deltaState} \sim 1 +{} & \text{offset}\big( \log(\Delta t) \big) +{} \\
			& dbh + dbh^2
	\end{split}
	\\[2ex]
	\begin{split} \label{eq::model_mu14}
		\text{deltaState} \sim 1 +{} & \text{offset}\big( \log(\Delta t) \big) +{} \\
			& cs + T_m + T_m^2 + P_d + P_d^2 + dbh + dbh^2
	\end{split}
\end{m}

\subsection{Complementary log-log function}
\subsubsection{How to account for $ \Delta t $?}
Here we show that the complementary log-log function $ g $ can take into account the time interval between to surveys. Let $ \mu $ be a probability, and $ \eta $ a linear predictor:
\begin{align*}
	g(\mu) &= \eta \\
		&= \beta_0 + \sum_{i = 1}^{n} \beta_i x_i
\end{align*}
By definition of the complementary log-log function, we have:
\[
	g(\mu) = \log \big( -log(1 - \mu) \big)
\]
and its reciprocal:
\begin{align*}
	g^{-1} \big( g(\mu) \big) &= g^{-1}(\eta) \\
		&= \mu \\
	g^{-1}(\eta) &= 1 - \exp \big[ -\exp(\eta) \big]
\end{align*}
Using the properties of the exponential, we get for any $ \Delta t > 0 $:
\begin{align*}
	g^{-1} \big( \eta + \log(\Delta t) \big) &= 1 - \exp \Big[ -\exp \big(\eta + \log(\Delta t)\big) \Big] \\
		&= 1 - \exp \big[ - \Delta t \exp (\eta) \big] \\
		&= 1 - \Big( \exp \big[ -\exp (\eta) \big] \Big)^{\Delta t}
\end{align*}
On the other side, we can prove that:
\[
	\exp \big[ -\exp (\eta) \big] = 1 - \mu
\]
Therefore we get:
\begin{equation}
	1 - \Big( \exp \big[ -\exp (\eta) \big] \Big)^{\Delta t} = 1 - (1 - \mu)^{\Delta t}
\end{equation}
which is the property we were looking for to consider time intervals between surveys. If $ \mu $ is the annual probability of death, then, $ (1 - \mu)^{\Delta t} $ is the probability of surviving $ \Delta t $ years and $ 1 - (1 - \mu)^{\Delta t} $ is the probability of dying.

\subsubsection{Why a positive slope in front of a squared term provides an optimal condition?}
To get an optimal climatic or size (dbh) condition, we need the mortality to first decrease, and then increase, that is to say, its derivative must first be negative and then positive. Let $ x $ be a variable (dbh, temperature, or precipitation), and $ \eta(x) = \beta_0 + \beta_1 x + \beta_2 x^2 $ the linear predictor:
\[
	\mu(x) = 1 - \exp \Big[ -\exp \big(\eta(x) \big) \Big]
\]
Therefore:
\[
	\mu'(x) = \eta'(x) \exp \Big[ \eta(x) -\exp \big(\eta(x) \big) \Big]
\]
The sign of $ \mu' $ depends exclusively on the sign of $ \eta' $, which is in our case:
\begin{equation}\label{eq::sign_eta}
	\eta'(x) = \beta_1 x + \beta_2 x^2
\end{equation}
It is easy to see from equation \eqref{eq::sign_eta} that for $ \beta_2 > 0 $, we have:
\[
	\begin{cases}
		\eta'(x) \leqslant 0 & x \leqslant -\frac{\beta_1}{2\beta_2} \\
		\eta'(x) \geqslant 0 & x \geqslant -\frac{\beta_1}{2\beta_2}
	\end{cases}
\]
which is what we want (incidentally, the optimal condition is reached for $ x = -\sfrac{\beta_1}{2\beta_2}$).

\subsection{Shade tolerance information}
The 14 parameterised species are listed in table \ref{tab::species}, with their vernacular name, scientific name and species code. We report the estimated canopy status effect from growth and mortality estimations (Tab. \ref{tab::cs}), and the shade tolerance informations \citet{Burns1990, Burns1990a} used to make Fig. \ref{fig::groups} and \ref{fig::groups_mu}.

\begin{table}[h!]
\centering
\caption{Estimated values of canopy status (cs) effect for growth ($ G $), and mortality ($ \mu $). These values are used in the Figs \ref{fig::groups} and \ref{fig::groups_mu}. A shade tolerance level---High(H), Medium (M) or Low (L)---is associated to each species according to \citet{Burns1990, Burns1990a}, and the page column specifies where the information has been found. \label{tab::cs}}
\begin{tabular}{@{}rcccc@{}}
	\toprule
	\bfseries{species} & \bfseries{cs} ($ G $) & \bfseries{cs} ($ \mu $) & \bfseries{Tolerance level} & \bfseries{Page} \\
	\midrule
	ABI-BAL & 0.24 & -0.56 & H & 37 \\
	ACE-RUB & 0.24 & -0.42 & M & 72 \\
	ACE-SAC & 0.23 & -0.53 & H & 204 \\
	BET-ALL & 0.28 & -0.38 & M & 302 \\
	BET-PAP & 0.41 & -0.95 & L & 349 \\
	FAG-GRA & -0.01 & 0.24 & H & 661 \\
	PIC-GLA & 0.46 & -1.09 & M & 415 \\
	PIC-MAR & 0.26 & -0.77 & M & 454 \\
	PIC-RUB & 0.33 & -0.79 & H & 504 \\
	PIN-BAN & 0.27 & -1.81 & L & 569 \\
	PIN-STR & 0.41 & -0.54 & M & 987-988 \\
	POP-TRE & 0.40 & -1.05 & L & 1095 \\
	THU-OCC & 0.13 & -0.26 & H & 1199 \\
	TSU-CAN & 0.12 & 0.26 & H & 1247 \\
	\bottomrule
\end{tabular}
\end{table}

\subsection{Growth function and uncertainties}
We plot here the growth of the 14 species in function of dbh.
\begin{figure}[htb] %% -- first part
	\centering
	%% First row
	\begin{subfigure}{0.48\textwidth}
		\input{graphs/18032-ABI-BAL_G_dbh}
		\caption{\textit{Abies balsamea}}
		\label{fig::abibal_G_dbh}
	\end{subfigure}
	\hfill
	\begin{subfigure}{0.48\textwidth}
		\input{graphs/28731-ACE-SAC_G_dbh}
		\caption{\textit{Acer saccharum}}
		\label{fig::acesac_G_dbh}
	\end{subfigure}
	\medskip
	%% Second row
	\begin{subfigure}{0.48\textwidth}
		\input{graphs/28728-ACE-RUB_G_dbh}
		\caption{\textit{Acer rubrum}}
		\label{fig::acerub_G_dbh}
	\end{subfigure}
	\hfill
	\begin{subfigure}{0.48\textwidth}
		\input{graphs/19481-BET-ALL_G_dbh}
		\caption{\textit{Betula alleghaniensis}}
		\label{fig::betal_G_dbh}
	\end{subfigure}
\end{figure}

\begin{figure}[htb] \ContinuedFloat %% -- Second part
	\centering
	%% First row
	\begin{subfigure}{0.48\textwidth}
		\input{graphs/19489-BET-PAP_G_dbh}
		\caption{\textit{Betula papyrifera}}
		\label{fig::betap_G_dbh}
	\end{subfigure}
	\hfil
	\begin{subfigure}{0.48\textwidth}
		\input{graphs/19462-FAG-GRA_G_dbh}
		\caption{\textit{Fagus grandifolia}}
		\label{fig::faggran_G_dbh}
	\end{subfigure}
	\medskip
	%% Second row
	\begin{subfigure}{0.48\textwidth}
		\input{graphs/183295-PIC-GLA_G_dbh}
		\caption{\textit{Picea glauca}}
		\label{fig::picgla_G_dbh}
	\end{subfigure}
	\hfil
	\begin{subfigure}{0.48\textwidth}
		\input{graphs/183302-PIC-MAR_G_dbh}
		\caption{\textit{Picea mariana}}
		\label{fig::picmar_G_dbh}
	\end{subfigure}
\end{figure}

\begin{figure}[htb] \ContinuedFloat %% -- Third part
	\centering
	%% First row
	\begin{subfigure}{0.48\textwidth}
		\input{graphs/18034-PIC-RUB_G_dbh}
		\caption{\textit{Picea rubens}}
		\label{fig::picrub_G_dbh}
	\end{subfigure}
	\hfil
	\begin{subfigure}{0.48\textwidth}
		\input{graphs/183319-PIN-BAN_G_dbh}
		\caption{\textit{Pinus banksiana}}
		\label{fig::pinban_G_dbh}
	\end{subfigure}
	\medskip
	%% Second row
	\begin{subfigure}{0.48\textwidth}
		\input{graphs/183385-PIN-STR_G_dbh}
		\caption{\textit{Pinus strobus}}
		\label{fig::pinstr_G_dbh}
	\end{subfigure}
	\hfil
	\begin{subfigure}{0.48\textwidth}
		\input{graphs/195773-POP-TRE_G_dbh}
		\caption{\textit{Populus tremuloides}}
		\label{fig::poptre_G_dbh}
	\end{subfigure}
\end{figure}

\begin{figure}[htb] \ContinuedFloat %% -- Fourth part
	\centering
	%% First row
	\begin{subfigure}{0.48\textwidth}
		\input{graphs/505490-THU-OCC_G_dbh}
		\caption{\textit{Thuja occidentalis}}
		\label{fig::thuocc_G_dbh}
	\end{subfigure}
	\hfil
	\begin{subfigure}{0.48\textwidth}
		\input{graphs/183397-TSU-CAN_G_dbh}
		\caption{\textit{Tsuga canadensis}}
		\label{fig::tsucan_G_dbh}
	\end{subfigure}
	\caption{Species-specific growth as a function of dbh, with the climate set to the species-specific average. The orange line is for individuals in the overstorey, and the blue lines for individuals in the understorey. The shaded areas delimited by the dotted lines are the $ 95 \% $ confidence intervals. The double sided arrow near the x-axis is the species-specific range of parameterisation of dbh.}
	\label{fig::12speciesG_dbh}
\end{figure}

\subsection{Extrapolations of mortality function}
We plot here the mortality of the 13 remaining species in function of dbh. Note that not all the species have the expected u-shape, that is commonly seen in mortality functions \citep{Lines2010}. The u-shape can only be obtained with a negative dbh coefficient, while the hump shape is associated to a positive coefficient.

\begin{figure}[htb] %% -- first part
	\centering
	%% First row
	\begin{subfigure}{0.48\textwidth}
		\input{graphs/18032-ABI-BAL_M_dbh}
		\caption{\textit{Abies balsamea}}
		\label{fig::abibal_M_dbh}
	\end{subfigure}
	\hfill
	\begin{subfigure}{0.48\textwidth}
		\input{graphs/28731-ACE-SAC_M_dbh}
		\caption{\textit{Acer saccharum}}
		\label{fig::acesac_M_dbh}
	\end{subfigure}
	\medskip
	%% Second row
	\begin{subfigure}{0.48\textwidth}
		\input{graphs/28728-ACE-RUB_M_dbh}
		\caption{\textit{Acer rubrum}}
		\label{fig::acerub_M_dbh}
	\end{subfigure}
	\hfil
	\begin{subfigure}{0.48\textwidth}
		\input{graphs/19481-BET-ALL_M_dbh}
		\caption{\textit{Betula alleghaniensis}}
		\label{fig::betal_M_dbh}
	\end{subfigure}
\end{figure}

\begin{figure}[htb] \ContinuedFloat %% -- Second part
	\centering
	%% First row
	\begin{subfigure}{0.48\textwidth}
		\input{graphs/19489-BET-PAP_M_dbh}
		\caption{\textit{Betula papyrifera}}
		\label{fig::betap_M_dbh}
	\end{subfigure}
	\hfil
	\begin{subfigure}{0.48\textwidth}
		\input{graphs/19462-FAG-GRA_M_dbh}
		\caption{\textit{Fagus grandifolia}}
		\label{fig::faggran_M_dbh}
	\end{subfigure}
	\medskip
	%% Second row
	\begin{subfigure}{0.48\textwidth}
		\input{graphs/183295-PIC-GLA_M_dbh}
		\caption{\textit{Picea glauca}}
		\label{fig::picgla_M_dbh}
	\end{subfigure}
	\hfil
	\begin{subfigure}{0.48\textwidth}
		\input{graphs/183302-PIC-MAR_M_dbh}
		\caption{\textit{Picea mariana}}
		\label{fig::picmar_M_dbh}
	\end{subfigure}
\end{figure}

\begin{figure}[htb] \ContinuedFloat %% -- Third part
	\centering
	%% First row
	\begin{subfigure}{0.48\textwidth}
		\input{graphs/18034-PIC-RUB_M_dbh}
		\caption{\textit{Picea rubens}}
		\label{fig::picrub_M_dbh}
	\end{subfigure}
	\hfil
	\begin{subfigure}{0.48\textwidth}
		\input{graphs/183319-PIN-BAN_M_dbh}
		\caption{\textit{Pinus banksiana}}
		\label{fig::pinban_M_dbh}
	\end{subfigure}
	\medskip
	%% Second row
	\begin{subfigure}{0.48\textwidth}
		\input{graphs/183385-PIN-STR_M_dbh}
		\caption{\textit{Pinus strobus}}
		\label{fig::pinstr_M_dbh}
	\end{subfigure}
	\hfil
	\begin{subfigure}{0.48\textwidth}
		\input{graphs/195773-POP-TRE_M_dbh}
		\caption{\textit{Populus tremuloides}}
		\label{fig::poptre_M_dbh}
	\end{subfigure}
\end{figure}

\begin{figure}[htb] \ContinuedFloat %% -- Third part
	\centering
	%% First row
	\begin{subfigure}{0.48\textwidth}
		\input{graphs/505490-THU-OCC_M_dbh}
		\caption{\textit{Thuja occidentalis}}
		\label{fig::thuocc_M_dbh}
	\end{subfigure}
	\hfil
	\begin{subfigure}{0.48\textwidth}
		\input{graphs/183397-TSU-CAN_M_dbh}
		\caption{\textit{Tsuga canadensis}}
		\label{fig::tsucan_M_dbh}
	\end{subfigure}
	\caption{Species-specific mortality as a function of dbh, with the climate set to the species-specific average. The orange line is for individuals in the overstorey, and the blue lines for individuals in the understorey. The shaded areas delimited by the dotted lines are the $ 95 \% $ confidence intervals. The double sided arrow near the x-axis is the species-specific range of parameterisation of dbh.}
	\label{fig::12speciesM_dbh}
\end{figure}

\printbibliography[heading=subbibliography]
\end{refsection}


\documentclass[letterpaper, 12pt]{article}

%%%%%%%%%%%%%%%%%%%	PACKAGES	%%%%%%%%%%%%%%%%%%%
%% Font
\usepackage{fontspec}
\usepackage{marvosym}

%% Languages
\usepackage{polyglossia}
	\setdefaultlanguage[variant=british]{english}

%% Table of content
\addto\captionsenglish{ % Replace "english" with the language used in babel
	\renewcommand{\contentsname}{Supporting Information Appendix S4 contents}
}

%% Marges, space...
\usepackage[top=2.5cm, bottom=2.5cm, left=2.5cm, right=2.5cm]{geometry}

%% Graphics
\usepackage[luatex]{graphicx}

\usepackage{tikz}
	\usetikzlibrary{arrows, plotmarks, decorations.markings}
	\tikzstyle{arrow} = [->,>=stealth,thick,rounded corners=4pt,line width=1pt]
	\usetikzlibrary{shadows}
	\usetikzlibrary{shadings}
	\usetikzlibrary{positioning} % relative coodinate
	\usetikzlibrary{tikzmark, calc} % calc, to calculate coordinate
	\usetikzlibrary{decorations.pathmorphing} % to snake an arrow
	\usetikzlibrary{shapes.arrows}
	\usetikzlibrary{patterns}
	\tikzset{
		invisible/.style={opacity=0},
		visible on/.style={alt={#1{}{invisible}}},
		alt/.code args={<#1>#2#3}{%
		\alt<#1>{\pgfkeysalso{#2}}{\pgfkeysalso{#3}} % \pgfkeysalso doesn't change the path
		},
	} % end tikzset. Code from http://tex.stackexchange.com/questions/136143/tikz-animated-figure-in-beamer

\makeatletter
\renewcommand{\thefigure}{S\thesection.\@arabic\c@figure}
\renewcommand{\thetable}{S\thesection.\@arabic\c@table}
\renewcommand{\theequation}{S\thesection.\arabic{equation}}
\makeatother

%% mathematics
\usepackage{amsthm}
\usepackage{amsmath}
\usepackage{amssymb}
\usepackage{bbold}
\usepackage{dsfont}
\usepackage{mathrsfs}
\usepackage{bm}
\usepackage{xfrac}

\usepackage{thmbox} % cf after for "theorem" definitions

\usepackage{gensymb}
\newcommand {\s}{{s}^{*}}

%%%%%%%%%%%%%%%%%   SET COUNTER		%%%%%%%%%%%%%%%%%
\setcounter{section}{3}

\begin{document}
\tableofcontents
\listoftables
\listoffigures

\newpage
\section{Confindence intervals} \label{app::confInt}
\subsection{Growth}

\begin{figure}[h]
	\centering
	\input{confIntPlot/page_1_growth}
	\caption[Confidence intervals for growth 1]{Confidence intervals for growth, $ \s $ is the growth response to light. \label{fig::confInt_g_1}}
\end{figure}

\begin{figure}
	\centering
	\input{confIntPlot/page_2_growth}
	\caption[Confidence intervals for growth 2]{Confidence intervals for growth. \label{fig::confInt_g_2}}
\end{figure}

\begin{figure}
	\centering
	\input{confIntPlot/page_3_growth}
	\caption[Confidence intervals for growth 3]{Confidence intervals for growth, $ \s $ is the growth response to light. The colon `$ : $' denotes an interaction between two variables. \label{fig::confInt_g_3}}
\end{figure}

\begin{figure}
	\centering
	\input{confIntPlot/page_4_growth}
	\caption[Confidence intervals for growth 4]{Confidence intervals for growth. The colon `$ : $' denotes an interaction between two variables. \label{fig::confInt_g_4}}
\end{figure}

\begin{figure}
	\centering
	\input{confIntPlot/page_5_growth}
	\caption[Confidence intervals for growth 5]{Confidence intervals for growth. The colon `$ : $' denotes an interaction between two variables. \label{fig::confInt_g_5}}
\end{figure}

\clearpage
\subsection{Mortality}
\begin{figure}[h]
	\centering
	\input{confIntPlot/page_1_mortality}
	\caption[Confidence intervals for mortality 1]{Confidence intervals for mortality, $ \s $ is the mortality response to light. \label{fig::confInt_m_1}}
\end{figure}

\begin{figure}
	\centering
	\input{confIntPlot/page_2_mortality}
	\caption[Confidence intervals for mortality 2]{Confidence intervals for mortality. \label{fig::confInt_m_2}}
\end{figure}

\begin{figure}
	\centering
	\input{confIntPlot/page_3_mortality}
	\caption[Confidence intervals for mortality 3]{Confidence intervals for mortality, $ \s $ is the mortality response to light. The colon `$ : $' denotes an interaction between two variables. \label{fig::confInt_m_3}}
\end{figure}

\begin{figure}[h]
	\centering
	\input{rhat_distrib}
	\caption[Distribution of $ \hat{r} $]{Distribution of $ \hat{r} $ for all the parameters (including group effects). The threshold 1.05 is a commonly chosen limit to pinpoints a convergence problem from a chain. At convergence, the $ \hat{r} $ should be 1. The trace plots of the chains and posterior distributions of the parameters are available in Supporting Information Figures S7. \label{fig::rhat_conv}}
\end{figure}

\end{document}

\section{Maps} \label{app::maps}
\begin{refsection}
\subsection{With competition (canopy height $ \s = 10 $ m)}
\begin{table}[ht]
\centering
\caption{Azimuth (truncated) of the averaged gradients of $ \tilde \rho_0 $ for the northern and southern regions with competition (canopy height $ \s = 10 $ m). These values generated the Figs. \ref{fig::grad_north} and \ref{fig::grad_south} \label{tab::azimuth}}
\begin{tabular}{rcccc}
	\toprule
		~ & \multicolumn{2}{c}{\textbf{Northern region}} & \multicolumn{2}{c}{\textbf{Southern region}} \\
		\cmidrule(lr){2-3} \cmidrule(lr){4-5}
		\textbf{Species} & \textbf{Direction} & \textbf{Azimuth} & \textbf{Direction} & \textbf{Azimuth} \\
	\midrule
		ABI-BAL & South-East & 139 & South-East & 150 \\
		ACE-RUB & North-West & 313 & North-West & 340 \\
		ACE-SAC & North-West & 325 & North-West & 302 \\
		BET-ALL & North-West & 338 & South-East & 144 \\
		BET-PAP & South-West & 192 & South-West & 233 \\
		FAG-GRA & North-West & 312 & South-East & 159 \\
		PIC-GLA & North-West & 325 & North-West & 331 \\
		PIC-MAR & South-East & 159 & South-West & 206 \\
		PIC-RUB & South-East & 144 & South-East & 137 \\
		PIN-BAN & South-East & 93 & South-East & 150 \\
		PIN-STR & North-West & 333 & North-West & 332 \\
		POP-TRE & South-East & 164 & South-West & 243 \\
		THU-OCC & North-West & 309 & North-West & 319 \\
		TSU-CAN & North-West & 322 & North-West & 327 \\
	\bottomrule
\end{tabular}
\end{table}

\begin{figure}
	\centering
	\input{graphs/azimuth_north}
	\caption{Species-specific averaged direction of the gradient for the northern region, with a canopy height $ \s = 10 $ m. The azimuths can be found in Tab. \ref{tab::azimuth} \label{fig::grad_north}}
\end{figure}

\begin{figure}
	\centering
	\input{graphs/azimuth_south}
	\caption{Species-specific averaged direction of the gradient for the southern region, with a canopy height $ \s = 10 $ m. The azimuths can be found in Tab. \ref{tab::azimuth} \label{fig::grad_south}}
\end{figure}

%%%% Abies balsamea
\begin{figure}
	\centering
	\begin{tikzpicture}[colorbar arrow/.style={
			shape=single arrow,
			double arrow head extend=0.125cm,
			shape border rotate=90,
			minimum height=8cm,
			shading=#1
		}]
		\node[inner sep=0pt] (abibal) at (0,0)
		    {\includegraphics[scale=0.3]{graphs/18032-ABI-BAL_R0_h=10m_scaled}};
		\node [colorbar arrow=RmapShading] at (6,0) {};
		\node at (6.5,-3.9) {$ 0 $};
		\node at (6.5,3.9) {$ 1 $};
		\node at (6,4.5) {$ \tilde \rho_0 $};
	\end{tikzpicture}
	\caption{Map of \textit{Abies balsamea}'s performance ($ \tilde \rho_0 $) with a canopy height of $ 10 $ m (which corresponds to a species-specific diameter $ \s = 133 $ mm), and climate data averaged from $ 2006 $ to $ 2010 $. The distribution area is from \citet{Little1971} in eastern Canada and USA (Fig. \ref{fig::mapDatabase} for the bounding box of the data on the Northern American continent). Red colours indicate $ \tilde \rho_0 $ values close to 1, and decrease up to 0 through blue and dark colours. The green square represents the centroid of the (cropped) distribution, and the red arrows are the average direction of increase of $ \tilde \rho_0 $ for the northern and southern region to the centroid. The white arrows are local changes of $ \tilde \rho_0 $. Extreme values of $ \rho_0 $ (\ie the non-scaled equivalent of $ \tilde \rho_0 $) can be found in Tab. \ref{tab::R0_min_max} with the proportion of unsuitable patches ($ \rho_0 < 1$). \label{fig::abibal}}
\end{figure}

%%%% Acer rubrum
\begin{figure}
	\centering
	\begin{tikzpicture}[colorbar arrow/.style={
			shape=single arrow,
			double arrow head extend=0.125cm,
			shape border rotate=90,
			minimum height=8cm,
			shading=#1
		}]
		\node[inner sep=0pt] (acerub) at (0,0)
		    {\includegraphics[scale=0.3]{graphs/28728-ACE-RUB_R0_h=10m_scaled}};
		\node [colorbar arrow=RmapShading] at (6,0) {};
		\node at (6.5,-3.9) {$ 0 $};
		\node at (6.5,3.9) {$ 1 $};
		\node at (6,4.5) {$ \tilde \rho_0 $};
	\end{tikzpicture}
	\caption{Map of \textit{Acer rubrum}'s performance ($ \tilde \rho_0 $) with a canopy height of $ 10 $ m (which corresponds to a species-specific diameter $ \s = 85 $ mm), and climate data averaged from $ 2006 $ to $ 2010 $. The distribution area is from \citet{Little1971} in eastern Canada and USA (Fig. \ref{fig::mapDatabase} for the bounding box of the data on the Northern American continent). Red colours indicate $ \tilde \rho_0 $ values close to 1, and decrease up to 0 through blue and dark colours. The green square represents the centroid of the (cropped) distribution, and the red arrows are the average direction of increase of $ \tilde \rho_0 $ for the northern and southern region to the centroid. The white arrows are local changes of $ \tilde \rho_0 $. Extreme values of $ \rho_0 $ (\ie the non-scaled equivalent of $ \tilde \rho_0 $) can be found in Tab. \ref{tab::R0_min_max} with the proportion of unsuitable patches ($ \rho_0 < 1$). \label{fig::acerub}}
\end{figure}

%%%% Acer saccharum
\begin{figure}
	\centering
	\begin{tikzpicture}[colorbar arrow/.style={
			shape=single arrow,
			double arrow head extend=0.125cm,
			shape border rotate=90,
			minimum height=8cm,
			shading=#1
		}]
		\node[inner sep=0pt] (acsa) at (0,0)
		    {\includegraphics[scale=0.3]{graphs/28731-ACE-SAC_R0_h=10m_scaled.jpg}};
		\node [colorbar arrow=RmapShading] at (6,0) {};
		\node at (6.5,-3.9) {$ 0 $};
		\node at (6.5,3.9) {$ 1 $};
		\node at (6,4.5) {$ \tilde \rho_0 $};
	\end{tikzpicture}
	\caption{Map of \textit{Acer saccharum}'s performance ($ \tilde \rho_0 $) with a canopy height of $ 10 $ m (which corresponds to a species-specific diameter $ \s = 83 $ mm), and climate data averaged from $ 2006 $ to $ 2010 $. The distribution area is from \citet{Little1971} in eastern Canada and USA (Fig. \ref{fig::mapDatabase} for the bounding box of the data on the Northern American continent). Red colours indicate $ \tilde \rho_0 $ values close to 1, and decrease up to 0 through blue and dark colours. The green square represents the centroid of the (cropped) distribution, and the red arrows are the average direction of increase of $ \tilde \rho_0 $ for the northern and southern region to the centroid. The white arrows are local changes of $ \tilde \rho_0 $. Extreme values of $ \rho_0 $ (\ie the non-scaled equivalent of $ \tilde \rho_0 $) can be found in Tab. \ref{tab::R0_min_max} with the proportion of unsuitable patches ($ \rho_0 < 1$). \label{fig::acesac}}
\end{figure}

%%%% Betula alleghaniensis
\begin{figure}
	\centering
	\begin{tikzpicture}[colorbar arrow/.style={
			shape=single arrow,
			double arrow head extend=0.125cm,
			shape border rotate=90,
			minimum height=8cm,
			shading=#1
		}]
		\node[inner sep=0pt] (betal) at (0,0)
		    {\includegraphics[scale=0.3]{graphs/19481-BET-ALL_R0_h=10m_scaled}};
		\node [colorbar arrow=RmapShading] at (6,0) {};
		\node at (6.5,-3.9) {$ 0 $};
		\node at (6.5,3.9) {$ 1 $};
		\node at (6,4.5) {$ \tilde \rho_0 $};
	\end{tikzpicture}
	\caption{Map of \textit{Betula alleghaniensis}'s performance ($ \tilde \rho_0 $) with a canopy height of $ 10 $ m (which corresponds to a species-specific diameter $ \s = 100 $ mm), and climate data averaged from $ 2006 $ to $ 2010 $. The distribution area is from \citet{Little1971} in eastern Canada and USA (Fig. \ref{fig::mapDatabase} for the bounding box of the data on the Northern American continent). Red colours indicate $ \tilde \rho_0 $ values close to 1, and decrease up to 0 through blue and dark colours. The green square represents the centroid of the (cropped) distribution, and the red arrows are the average direction of increase of $ \tilde \rho_0 $ for the northern and southern region to the centroid. The white arrows are local changes of $ \tilde \rho_0 $. Extreme values of $ \rho_0 $ (\ie the non-scaled equivalent of $ \tilde \rho_0 $) can be found in Tab. \ref{tab::R0_min_max} with the proportion of unsuitable patches ($ \rho_0 < 1$). \label{fig::betal}}
\end{figure}

%%%% Betula papyrifera
\begin{figure}
	\centering
	\begin{tikzpicture}[colorbar arrow/.style={
			shape=single arrow,
			double arrow head extend=0.125cm,
			shape border rotate=90,
			minimum height=8cm,
			shading=#1
		}]
		\node[inner sep=0pt] (betap) at (0,0)
		    {\includegraphics[scale=0.3]{graphs/19489-BET-PAP_R0_h=10m_scaled}};
		\node [colorbar arrow=RmapShading] at (6,0) {};
		\node at (6.5,-3.9) {$ 0 $};
		\node at (6.5,3.9) {$ 1 $};
		\node at (6,4.5) {$ \tilde \rho_0 $};
	\end{tikzpicture}
	\caption{Map of \textit{Betula papyrifera}'s performance ($ \tilde \rho_0 $) with a canopy height of $ 10 $ m (which corresponds to a species-specific diameter $ \s = 98 $ mm), and climate data averaged from $ 2006 $ to $ 2010 $. The distribution area is from \citet{Little1971} in eastern Canada and USA (Fig. \ref{fig::mapDatabase} for the bounding box of the data on the Northern American continent). Red colours indicate $ \tilde \rho_0 $ values close to 1, and decrease up to 0 through blue and dark colours. The green square represents the centroid of the (cropped) distribution, and the red arrows are the average direction of increase of $ \tilde \rho_0 $ for the northern and southern region to the centroid. The white arrows are local changes of $ \tilde \rho_0 $. Extreme values of $ \rho_0 $ (\ie the non-scaled equivalent of $ \tilde \rho_0 $) can be found in Tab. \ref{tab::R0_min_max} with the proportion of unsuitable patches ($ \rho_0 < 1$). \label{fig::betap}}
\end{figure}

%%%% Fagus grandifolia
\begin{figure}
	\centering
	\begin{tikzpicture}[colorbar arrow/.style={
			shape=single arrow,
			double arrow head extend=0.125cm,
			shape border rotate=90,
			minimum height=8cm,
			shading=#1
		}]
		\node[inner sep=0pt] (faggran) at (0,0)
		    {\includegraphics[scale=0.3]{graphs/19462-FAG-GRA_R0_h=10m_scaled}};
		\node [colorbar arrow=RmapShading] at (6,0) {};
		\node at (6.5,-3.9) {$ 0 $};
		\node at (6.5,3.9) {$ 1 $};
		\node at (6,4.5) {$ \tilde \rho_0 $};
	\end{tikzpicture}
	\caption{Map of \textit{Fagus grandifolia}'s performance ($ \tilde \rho_0 $) with a canopy height of $ 10 $ m (which corresponds to a species-specific diameter $ \s = 103 $ mm), and climate data averaged from $ 2006 $ to $ 2010 $. The distribution area is from \citet{Little1971} in eastern Canada and USA (Fig. \ref{fig::mapDatabase} for the bounding box of the data on the Northern American continent). Red colours indicate $ \tilde \rho_0 $ values close to 1, and decrease up to 0 through blue and dark colours. The green square represents the centroid of the (cropped) distribution, and the red arrows are the average direction of increase of $ \tilde \rho_0 $ for the northern and southern region to the centroid. The white arrows are local changes of $ \tilde \rho_0 $. Extreme values of $ \rho_0 $ (\ie the non-scaled equivalent of $ \tilde \rho_0 $) can be found in Tab. \ref{tab::R0_min_max} with the proportion of unsuitable patches ($ \rho_0 < 1$). \label{fig::faggran}}
\end{figure}

%%%% Picea glauca
\begin{figure}
	\centering
	\begin{tikzpicture}[colorbar arrow/.style={
			shape=single arrow,
			double arrow head extend=0.125cm,
			shape border rotate=90,
			minimum height=8cm,
			shading=#1
		}]
		\node[inner sep=0pt] (picgla) at (0,0)
		    {\includegraphics[scale=0.3]{graphs/183295-PIC-GLA_R0_h=10m_scaled}};
		\node [colorbar arrow=RmapShading] at (6,0) {};
		\node at (6.5,-3.9) {$ 0 $};
		\node at (6.5,3.9) {$ 1 $};
		\node at (6,4.5) {$ \tilde \rho_0 $};
	\end{tikzpicture}
	\caption{Map of \textit{Picea glauca}'s performance ($ \tilde \rho_0 $) with a canopy height of $ 10 $ m (which corresponds to a species-specific diameter $ \s = 148 $ mm), and climate data averaged from $ 2006 $ to $ 2010 $. The distribution area is from \citet{Little1971} in eastern Canada and USA (Fig. \ref{fig::mapDatabase} for the bounding box of the data on the Northern American continent). Red colours indicate $ \tilde \rho_0 $ values close to 1, and decrease up to 0 through blue and dark colours. The green square represents the centroid of the (cropped) distribution, and the red arrows are the average direction of increase of $ \tilde \rho_0 $ for the northern and southern region to the centroid. The white arrows are local changes of $ \tilde \rho_0 $. Extreme values of $ \rho_0 $ (\ie the non-scaled equivalent of $ \tilde \rho_0 $) can be found in Tab. \ref{tab::R0_min_max} with the proportion of unsuitable patches ($ \rho_0 < 1$). \label{fig::picgla}}
\end{figure}

%%%% Picea mariana
\begin{figure}
	\centering
	\begin{tikzpicture}[colorbar arrow/.style={
			shape=single arrow,
			double arrow head extend=0.125cm,
			shape border rotate=90,
			minimum height=8cm,
			shading=#1
		}]
		\node[inner sep=0pt] (picmar) at (0,0)
		    {\includegraphics[scale=0.3]{graphs/183302-PIC-MAR_R0_h=10m_scaled}};
		\node [colorbar arrow=RmapShading] at (6,0) {};
		\node at (6.5,-3.9) {$ 0 $};
		\node at (6.5,3.9) {$ 1 $};
		\node at (6,4.5) {$ \tilde \rho_0 $};
	\end{tikzpicture}
	\caption{Map of \textit{Picea mariana}'s performance ($ \tilde \rho_0 $) with a canopy height of $ 10 $ m (which corresponds to a species-specific diameter $ \s = 127 $ mm), and climate data averaged from $ 2006 $ to $ 2010 $. The distribution area is from \citet{Little1971} in eastern Canada and USA (Fig. \ref{fig::mapDatabase} for the bounding box of the data on the Northern American continent). Red colours indicate $ \tilde \rho_0 $ values close to 1, and decrease up to 0 through blue and dark colours. The green square represents the centroid of the (cropped) distribution, and the red arrows are the average direction of increase of $ \tilde \rho_0 $ for the northern and southern region to the centroid. The white arrows are local changes of $ \tilde \rho_0 $. Extreme values of $ \rho_0 $ (\ie the non-scaled equivalent of $ \tilde \rho_0 $) can be found in Tab. \ref{tab::R0_min_max} with the proportion of unsuitable patches ($ \rho_0 < 1$). \label{fig::picmar}}
\end{figure}

%%%% Picea rubens
\begin{figure}
	\centering
	\begin{tikzpicture}[colorbar arrow/.style={
			shape=single arrow,
			double arrow head extend=0.125cm,
			shape border rotate=90,
			minimum height=8cm,
			shading=#1
		}]
		\node[inner sep=0pt] (picrub) at (0,0)
		    {\includegraphics[scale=0.3]{graphs/18034-PIC-RUB_R0_h=10m_scaled}};
		\node [colorbar arrow=RmapShading] at (6,0) {};
		\node at (6.5,-3.9) {$ 0 $};
		\node at (6.5,3.9) {$ 1 $};
		\node at (6,4.5) {$ \tilde \rho_0 $};
	\end{tikzpicture}
	\caption{Map of \textit{Picea rubens}'s performance ($ \tilde \rho_0 $) with a canopy height of $ 10 $ m (which corresponds to a species-specific diameter $ \s = 142 $ mm), and climate data averaged from $ 2006 $ to $ 2010 $. The distribution area is from \citet{Little1971} in eastern Canada and USA (Fig. \ref{fig::mapDatabase} for the bounding box of the data on the Northern American continent). Red colours indicate $ \tilde \rho_0 $ values close to 1, and decrease up to 0 through blue and dark colours. The green square represents the centroid of the (cropped) distribution, and the red arrows are the average direction of increase of $ \tilde \rho_0 $ for the northern and southern region to the centroid. The white arrows are local changes of $ \tilde \rho_0 $. Extreme values of $ \rho_0 $ (\ie the non-scaled equivalent of $ \tilde \rho_0 $) can be found in Tab. \ref{tab::R0_min_max} with the proportion of unsuitable patches ($ \rho_0 < 1$). \label{fig::picrub}}
\end{figure}

%%%% Pinus banksiana
\begin{figure}
	\centering
	\begin{tikzpicture}[colorbar arrow/.style={
			shape=single arrow,
			double arrow head extend=0.125cm,
			shape border rotate=90,
			minimum height=8cm,
			shading=#1
		}]
		\node[inner sep=0pt] (pinban) at (0,0)
		    {\includegraphics[scale=0.3]{graphs/183319-PIN-BAN_R0_h=10m_scaled}};
		\node [colorbar arrow=RmapShading] at (6,0) {};
		\node at (6.5,-3.9) {$ 0 $};
		\node at (6.5,3.9) {$ 1 $};
		\node at (6,4.5) {$ \tilde \rho_0 $};
	\end{tikzpicture}
	\caption{Map of \textit{Pinus banksiana}'s performance ($ \tilde \rho_0 $) with a canopy height of $ 10 $ m (which corresponds to a species-specific diameter $ \s = 138 $ mm), and climate data averaged from $ 2006 $ to $ 2010 $. The distribution area is from \citet{Little1971} in eastern Canada and USA (Fig. \ref{fig::mapDatabase} for the bounding box of the data on the Northern American continent). Red colours indicate $ \tilde \rho_0 $ values close to 1, and decrease up to 0 through blue and dark colours. The green square represents the centroid of the (cropped) distribution, and the red arrows are the average direction of increase of $ \tilde \rho_0 $ for the northern and southern region to the centroid. The white arrows are local changes of $ \tilde \rho_0 $. Extreme values of $ \rho_0 $ (\ie the non-scaled equivalent of $ \tilde \rho_0 $) can be found in Tab. \ref{tab::R0_min_max} with the proportion of unsuitable patches ($ \rho_0 < 1$). \label{fig::pinban}}
\end{figure}

%%%% Pinus strobus
\begin{figure}
	\centering
	\begin{tikzpicture}[colorbar arrow/.style={
			shape=single arrow,
			double arrow head extend=0.125cm,
			shape border rotate=90,
			minimum height=8cm,
			shading=#1
		}]
		\node[inner sep=0pt] (pinstr) at (0,0)
		    {\includegraphics[scale=0.3]{graphs/183385-PIN-STR_R0_h=10m_scaled}};
		\node [colorbar arrow=RmapShading] at (6,0) {};
		\node at (6.5,-3.9) {$ 0 $};
		\node at (6.5,3.9) {$ 1 $};
		\node at (6,4.5) {$ \tilde \rho_0 $};
	\end{tikzpicture}
	\caption{Map of \textit{Pinus strobus}'s performance ($ \tilde \rho_0 $) with a canopy height of $ 10 $ m (which corresponds to a species-specific diameter $ \s = 139 $ mm), and climate data averaged from $ 2006 $ to $ 2010 $. The distribution area is from \citet{Little1971} in eastern Canada and USA (Fig. \ref{fig::mapDatabase} for the bounding box of the data on the Northern American continent). Red colours indicate $ \tilde \rho_0 $ values close to 1, and decrease up to 0 through blue and dark colours. The green square represents the centroid of the (cropped) distribution, and the red arrows are the average direction of increase of $ \tilde \rho_0 $ for the northern and southern region to the centroid. The white arrows are local changes of $ \tilde \rho_0 $. Extreme values of $ \rho_0 $ (\ie the non-scaled equivalent of $ \tilde \rho_0 $) can be found in Tab. \ref{tab::R0_min_max} with the proportion of unsuitable patches ($ \rho_0 < 1$). \label{fig::pinstr}}
\end{figure}

%%%% Populus tremuloides
\begin{figure}
	\centering
	\begin{tikzpicture}[colorbar arrow/.style={
			shape=single arrow,
			double arrow head extend=0.125cm,
			shape border rotate=90,
			minimum height=8cm,
			shading=#1
		}]
		\node[inner sep=0pt] (poptre) at (0,0)
		    {\includegraphics[scale=0.3]{graphs/195773-POP-TRE_R0_h=10m_scaled}};
		\node [colorbar arrow=RmapShading] at (6,0) {};
		\node at (6.5,-3.9) {$ 0 $};
		\node at (6.5,3.9) {$ 1 $};
		\node at (6,4.5) {$ \tilde \rho_0 $};
	\end{tikzpicture}
	\caption{Map of \textit{Populus tremuloides}'s performance ($ \tilde \rho_0 $) with a canopy height of $ 10 $ m (which corresponds to a species-specific diameter $ \s = 117 $ mm), and climate data averaged from $ 2006 $ to $ 2010 $. The distribution area is from \citet{Little1971} in eastern Canada and USA (Fig. \ref{fig::mapDatabase} for the bounding box of the data on the Northern American continent). Red colours indicate $ \tilde \rho_0 $ values close to 1, and decrease up to 0 through blue and dark colours. The green square represents the centroid of the (cropped) distribution, and the red arrows are the average direction of increase of $ \tilde \rho_0 $ for the northern and southern region to the centroid. The white arrows are local changes of $ \tilde \rho_0 $. Extreme values of $ \rho_0 $ (\ie the non-scaled equivalent of $ \tilde \rho_0 $) can be found in Tab. \ref{tab::R0_min_max} with the proportion of unsuitable patches ($ \rho_0 < 1$). \label{fig::poptre}}
\end{figure}

%%%% Thuja occidentalis
\begin{figure}
	\centering
	\begin{tikzpicture}[colorbar arrow/.style={
			shape=single arrow,
			double arrow head extend=0.125cm,
			shape border rotate=90,
			minimum height=8cm,
			shading=#1
		}]
		\node[inner sep=0pt] (thuocc) at (0,0)
		    {\includegraphics[scale=0.3]{graphs/505490-THU-OCC_R0_h=10m_scaled}};
		\node [colorbar arrow=RmapShading] at (6,0) {};
		\node at (6.5,-3.9) {$ 0 $};
		\node at (6.5,3.9) {$ 1 $};
		\node at (6,4.5) {$ \tilde \rho_0 $};
	\end{tikzpicture}
	\caption{Map of \textit{Thuja occidentalis}'s performance ($ \tilde \rho_0 $) with a canopy height of $ 10 $ m (which corresponds to a species-specific diameter $ \s = 175 $ mm), and climate data averaged from $ 2006 $ to $ 2010 $. The distribution area is from \citet{Little1971} in eastern Canada and USA (Fig. \ref{fig::mapDatabase} for the bounding box of the data on the Northern American continent). Red colours indicate $ \tilde \rho_0 $ values close to 1, and decrease up to 0 through blue and dark colours. The green square represents the centroid of the (cropped) distribution, and the red arrows are the average direction of increase of $ \tilde \rho_0 $ for the northern and southern region to the centroid. The white arrows are local changes of $ \tilde \rho_0 $. Extreme values of $ \rho_0 $ (\ie the non-scaled equivalent of $ \tilde \rho_0 $) can be found in Tab. \ref{tab::R0_min_max} with the proportion of unsuitable patches ($ \rho_0 < 1$). \label{fig::thuocc}}
\end{figure}

%%%% Tsuga canadensis
\begin{figure}
	\centering
	\begin{tikzpicture}[colorbar arrow/.style={
			shape=single arrow,
			double arrow head extend=0.125cm,
			shape border rotate=90,
			minimum height=8cm,
			shading=#1
		}]
		\node[inner sep=0pt] (tsucan) at (0,0)
		    {\includegraphics[scale=0.3]{graphs/183397-TSU-CAN_R0_h=10m_scaled}};
		\node [colorbar arrow=RmapShading] at (6,0) {};
		\node at (6.5,-3.9) {$ 0 $};
		\node at (6.5,3.9) {$ 1 $};
		\node at (6,4.5) {$ \tilde \rho_0 $};
	\end{tikzpicture}
	\caption{Map of \textit{Tsuga canadensis}'s performance ($ \tilde \rho_0 $) with a canopy height of $ 10 $ m (which corresponds to a species-specific diameter $ \s = 149 $ mm), and climate data averaged from $ 2006 $ to $ 2010 $. The distribution area is from \citet{Little1971} in eastern Canada and USA (Fig. \ref{fig::mapDatabase} for the bounding box of the data on the Northern American continent). Red colours indicate $ \tilde \rho_0 $ values close to 1, and decrease up to 0 through blue and dark colours. The green square represents the centroid of the (cropped) distribution, and the red arrows are the average direction of increase of $ \tilde \rho_0 $ for the northern and southern region to the centroid. The white arrows are local changes of $ \tilde \rho_0 $. Extreme values of $ \rho_0 $ (\ie the non-scaled equivalent of $ \tilde \rho_0 $) can be found in Tab. \ref{tab::R0_min_max} with the proportion of unsuitable patches ($ \rho_0 < 1$). \label{fig::tsucan}}
\end{figure}

\subsection{Without competition (canopy height $ \s = 0 $ m)}
\begin{table}[ht]
\centering
\caption{Azimuth (truncated) of the averaged gradients of $ \tilde \rho_0 $ for the northern and southern regions without competition (canopy height $ \s = 0 $ m). These values generated the Figs. \ref{fig::grad_north_0} and \ref{fig::grad_south_0} \label{tab::azimuth_0}}
\begin{tabular}{rcccc}
	\toprule
	~ & \multicolumn{2}{c}{\textbf{Northern region}} & \multicolumn{2}{c}{\textbf{Southern region}} \\
	\cmidrule(lr){2-3} \cmidrule(lr){4-5}
	\textbf{Species} & \textbf{Direction} & \textbf{Azimuth} & \textbf{Direction} & \textbf{Azimuth} \\
	\midrule
		ABI-BAL & South-East & 124 & South-East & 137 \\
		ACE-RUB & North-West & 324 & North-West & 342 \\
		ACE-SAC & North-West & 332 & North-West & 323 \\
		BET-ALL & North-West & 338 & South-East & 144 \\
		BET-PAP & South-West & 181 & North-West & 334 \\
		FAG-GRA & North-West & 324 & South-East & 157 \\
		PIC-GLA & North-West & 316 & North-West & 331 \\
		PIC-MAR & South-East & 172 & North-West & 347 \\
		PIC-RUB & South-East & 177 & South-East & 146 \\
		PIN-BAN & South-East & 176 & South-East & 163 \\
		PIN-STR & North-West & 331 & North-West & 331 \\
		POP-TRE & North-West & 308 & North-West & 354 \\
		THU-OCC & South-West & 197 & North-West & 326 \\
		TSU-CAN & North-West & 324 & North-West & 325 \\
	\bottomrule
\end{tabular}
\end{table}

\begin{figure}
	\centering
	\input{graphs/azimuth_north_0}
	\caption{Species-specific averaged direction of the gradient for the northern region, without competition (canopy height $ \s = 0 $ m). The azimuths can be found in Tab. \ref{tab::azimuth_0} \label{fig::grad_north_0}}
\end{figure}

\begin{figure}
	\centering
	\input{graphs/azimuth_south_0}
	\caption{Species-specific averaged direction of the gradient for the southern region, without competition (canopy height $ \s = 0 $ m). The azimuths can be found in Tab. \ref{tab::azimuth_0} \label{fig::grad_south_0}}
\end{figure}

%%%% Abies balsamea
\begin{figure}
	\centering
	\begin{tikzpicture}[colorbar arrow/.style={
			shape=single arrow,
			double arrow head extend=0.125cm,
			shape border rotate=90,
			minimum height=8cm,
			shading=#1
		}]
		\node[inner sep=0pt] (abibal) at (0,0)
		    {\includegraphics[scale=0.3]{graphs/18032-ABI-BAL_R0_h=0m_scaled}};
		\node [colorbar arrow=RmapShading] at (6,0) {};
		\node at (6.5,-3.9) {$ 0 $};
		\node at (6.5,3.9) {$ 1 $};
		\node at (6,4.5) {$ \tilde \rho_0 $};
	\end{tikzpicture}
	\caption{Map of \textit{Abies balsamea}'s performance ($ \tilde \rho_0 $) without competition (canopy height = $ 0 $ m), and climate data averaged from $ 2006 $ to $ 2010 $. The distribution area is from \citet{Little1971} in eastern Canada and USA (Fig. \ref{fig::mapDatabase} for the bounding box of the data on the Northern American continent). Red colours indicate $ \tilde \rho_0 $ values close to 1, and decrease up to 0 through blue and dark colours. The green square represents the centroid of the (cropped) distribution, and the red arrows are the average direction of increase of $ \tilde \rho_0 $ for the northern and southern region to the centroid. The white arrows are local changes of $ \tilde \rho_0 $. Extreme values of $ \rho_0 $ (\ie the non-scaled equivalent of $ \tilde \rho_0 $) can be found in Tab. \ref{tab::R0_min_max} with the proportion of unsuitable patches ($ \rho_0 < 1$). \label{fig::abibal_0}}
\end{figure}

%%%% Acer rubrum
\begin{figure}
	\centering
	\begin{tikzpicture}[colorbar arrow/.style={
			shape=single arrow,
			double arrow head extend=0.125cm,
			shape border rotate=90,
			minimum height=8cm,
			shading=#1
		}]
		\node[inner sep=0pt] (acerub) at (0,0)
		    {\includegraphics[scale=0.3]{graphs/28728-ACE-RUB_R0_h=0m_scaled}};
		\node [colorbar arrow=RmapShading] at (6,0) {};
		\node at (6.5,-3.9) {$ 0 $};
		\node at (6.5,3.9) {$ 1 $};
		\node at (6,4.5) {$ \tilde \rho_0 $};
	\end{tikzpicture}
	\caption{Map of \textit{Acer rubrum}'s performance ($ \tilde \rho_0 $) without competition (canopy height = $ 0 $ m), and climate data averaged from $ 2006 $ to $ 2010 $. The distribution area is from \citet{Little1971} in eastern Canada and USA (Fig. \ref{fig::mapDatabase} for the bounding box of the data on the Northern American continent). Red colours indicate $ \tilde \rho_0 $ values close to 1, and decrease up to 0 through blue and dark colours. The green square represents the centroid of the (cropped) distribution, and the red arrows are the average direction of increase of $ \tilde \rho_0 $ for the northern and southern region to the centroid. The white arrows are local changes of $ \tilde \rho_0 $. Extreme values of $ \rho_0 $ (\ie the non-scaled equivalent of $ \tilde \rho_0 $) can be found in Tab. \ref{tab::R0_min_max} with the proportion of unsuitable patches ($ \rho_0 < 1$). \label{fig::acerub_0}}
\end{figure}

%%%% Acer saccharum
\begin{figure}
	\centering
	\begin{tikzpicture}[colorbar arrow/.style={
			shape=single arrow,
			double arrow head extend=0.125cm,
			shape border rotate=90,
			minimum height=8cm,
			shading=#1
		}]
		\node[inner sep=0pt] (acerub) at (0,0)
		    {\includegraphics[scale=0.3]{graphs/28731-ACE-SAC_R0_h=0m_scaled}};
		\node [colorbar arrow=RmapShading] at (6,0) {};
		\node at (6.5,-3.9) {$ 0 $};
		\node at (6.5,3.9) {$ 1 $};
		\node at (6,4.5) {$ \tilde \rho_0 $};
	\end{tikzpicture}
	\caption{Map of \textit{Acer saccharum}'s performance ($ \tilde \rho_0 $) without competition (canopy height = $ 0 $ m), and climate data averaged from $ 2006 $ to $ 2010 $. The distribution area is from \citet{Little1971} in eastern Canada and USA (Fig. \ref{fig::mapDatabase} for the bounding box of the data on the Northern American continent). Red colours indicate $ \tilde \rho_0 $ values close to 1, and decrease up to 0 through blue and dark colours. The green square represents the centroid of the (cropped) distribution, and the red arrows are the average direction of increase of $ \tilde \rho_0 $ for the northern and southern region to the centroid. The white arrows are local changes of $ \tilde \rho_0 $. Extreme values of $ \rho_0 $ (\ie the non-scaled equivalent of $ \tilde \rho_0 $) can be found in Tab. \ref{tab::R0_min_max} with the proportion of unsuitable patches ($ \rho_0 < 1$). \label{fig::acesac_0}}
\end{figure}

%%%% Betula alleghaniensis
\begin{figure}
	\centering
	\begin{tikzpicture}[colorbar arrow/.style={
			shape=single arrow,
			double arrow head extend=0.125cm,
			shape border rotate=90,
			minimum height=8cm,
			shading=#1
		}]
		\node[inner sep=0pt] (betal) at (0,0)
		    {\includegraphics[scale=0.3]{graphs/19481-BET-ALL_R0_h=0m_scaled}};
		\node [colorbar arrow=RmapShading] at (6,0) {};
		\node at (6.5,-3.9) {$ 0 $};
		\node at (6.5,3.9) {$ 1 $};
		\node at (6,4.5) {$ \tilde \rho_0 $};
	\end{tikzpicture}
	\caption{Map of \textit{Betula alleghaniensis}'s performance ($ \tilde \rho_0 $) without competition (canopy height = $ 0 $ m), and climate data averaged from $ 2006 $ to $ 2010 $. The distribution area is from \citet{Little1971} in eastern Canada and USA (Fig. \ref{fig::mapDatabase} for the bounding box of the data on the Northern American continent). Red colours indicate $ \tilde \rho_0 $ values close to 1, and decrease up to 0 through blue and dark colours. The green square represents the centroid of the (cropped) distribution, and the red arrows are the average direction of increase of $ \tilde \rho_0 $ for the northern and southern region to the centroid. The white arrows are local changes of $ \tilde \rho_0 $. Extreme values of $ \rho_0 $ (\ie the non-scaled equivalent of $ \tilde \rho_0 $) can be found in Tab. \ref{tab::R0_min_max} with the proportion of unsuitable patches ($ \rho_0 < 1$). \label{fig::betal_0}}
\end{figure}

%%%% Betula papyrifera
\begin{figure}
	\centering
	\begin{tikzpicture}[colorbar arrow/.style={
			shape=single arrow,
			double arrow head extend=0.125cm,
			shape border rotate=90,
			minimum height=8cm,
			shading=#1
		}]
		\node[inner sep=0pt] (betap) at (0,0)
		    {\includegraphics[scale=0.3]{graphs/19489-BET-PAP_R0_h=0m_scaled}};
		\node [colorbar arrow=RmapShading] at (6,0) {};
		\node at (6.5,-3.9) {$ 0 $};
		\node at (6.5,3.9) {$ 1 $};
		\node at (6,4.5) {$ \tilde \rho_0 $};
	\end{tikzpicture}
	\caption{Map of \textit{Betula papyrifera}'s performance ($ \tilde \rho_0 $) without competition (canopy height = $ 0 $ m), and climate data averaged from $ 2006 $ to $ 2010 $. The distribution area is from \citet{Little1971} in eastern Canada and USA (Fig. \ref{fig::mapDatabase} for the bounding box of the data on the Northern American continent). Red colours indicate $ \tilde \rho_0 $ values close to 1, and decrease up to 0 through blue and dark colours. The green square represents the centroid of the (cropped) distribution, and the red arrows are the average direction of increase of $ \tilde \rho_0 $ for the northern and southern region to the centroid. The white arrows are local changes of $ \tilde \rho_0 $. Extreme values of $ \rho_0 $ (\ie the non-scaled equivalent of $ \tilde \rho_0 $) can be found in Tab. \ref{tab::R0_min_max} with the proportion of unsuitable patches ($ \rho_0 < 1$). \label{fig::betap_0}}
\end{figure}

%%%% Fagus grandifolia
\begin{figure}
	\centering
	\begin{tikzpicture}[colorbar arrow/.style={
			shape=single arrow,
			double arrow head extend=0.125cm,
			shape border rotate=90,
			minimum height=8cm,
			shading=#1
		}]
		\node[inner sep=0pt] (faggran) at (0,0)
		    {\includegraphics[scale=0.3]{graphs/19462-FAG-GRA_R0_h=0m_scaled}};
		\node [colorbar arrow=RmapShading] at (6,0) {};
		\node at (6.5,-3.9) {$ 0 $};
		\node at (6.5,3.9) {$ 1 $};
		\node at (6,4.5) {$ \tilde \rho_0 $};
	\end{tikzpicture}
	\caption{Map of \textit{Fagus grandifolia}'s performance ($ \tilde \rho_0 $) without competition (canopy height = $ 0 $ m), and climate data averaged from $ 2006 $ to $ 2010 $. The distribution area is from \citet{Little1971} in eastern Canada and USA (Fig. \ref{fig::mapDatabase} for the bounding box of the data on the Northern American continent). Red colours indicate $ \tilde \rho_0 $ values close to 1, and decrease up to 0 through blue and dark colours. The green square represents the centroid of the (cropped) distribution, and the red arrows are the average direction of increase of $ \tilde \rho_0 $ for the northern and southern region to the centroid. The white arrows are local changes of $ \tilde \rho_0 $. Extreme values of $ \rho_0 $ (\ie the non-scaled equivalent of $ \tilde \rho_0 $) can be found in Tab. \ref{tab::R0_min_max} with the proportion of unsuitable patches ($ \rho_0 < 1$). \label{fig::faggran_0}}
\end{figure}

%%%% Picea glauca
\begin{figure}
	\centering
	\begin{tikzpicture}[colorbar arrow/.style={
			shape=single arrow,
			double arrow head extend=0.125cm,
			shape border rotate=90,
			minimum height=8cm,
			shading=#1
		}]
		\node[inner sep=0pt] (picgla) at (0,0)
		    {\includegraphics[scale=0.3]{graphs/183295-PIC-GLA_R0_h=0m_scaled}};
		\node [colorbar arrow=RmapShading] at (6,0) {};
		\node at (6.5,-3.9) {$ 0 $};
		\node at (6.5,3.9) {$ 1 $};
		\node at (6,4.5) {$ \tilde \rho_0 $};
	\end{tikzpicture}
	\caption{Map of \textit{Picea glauca}'s performance ($ \tilde \rho_0 $) without competition (canopy height = $ 0 $ m), and climate data averaged from $ 2006 $ to $ 2010 $. The distribution area is from \citet{Little1971} in eastern Canada and USA (Fig. \ref{fig::mapDatabase} for the bounding box of the data on the Northern American continent). Red colours indicate $ \tilde \rho_0 $ values close to 1, and decrease up to 0 through blue and dark colours. The green square represents the centroid of the (cropped) distribution, and the red arrows are the average direction of increase of $ \tilde \rho_0 $ for the northern and southern region to the centroid. The white arrows are local changes of $ \tilde \rho_0 $. Extreme values of $ \rho_0 $ (\ie the non-scaled equivalent of $ \tilde \rho_0 $) can be found in Tab. \ref{tab::R0_min_max} with the proportion of unsuitable patches ($ \rho_0 < 1$). \label{fig::picgla_0}}
\end{figure}

%%%% Picea mariana
\begin{figure}
	\centering
	\begin{tikzpicture}[colorbar arrow/.style={
			shape=single arrow,
			double arrow head extend=0.125cm,
			shape border rotate=90,
			minimum height=8cm,
			shading=#1
		}]
		\node[inner sep=0pt] (picmar) at (0,0)
		    {\includegraphics[scale=0.3]{graphs/183302-PIC-MAR_R0_h=0m_scaled}};
		\node [colorbar arrow=RmapShading] at (6,0) {};
		\node at (6.5,-3.9) {$ 0 $};
		\node at (6.5,3.9) {$ 1 $};
		\node at (6,4.5) {$ \tilde \rho_0 $};
	\end{tikzpicture}
	\caption{Map of \textit{Picea mariana}'s performance ($ \tilde \rho_0 $) without competition (canopy height = $ 0 $ m), and climate data averaged from $ 2006 $ to $ 2010 $. The distribution area is from \citet{Little1971} in eastern Canada and USA (Fig. \ref{fig::mapDatabase} for the bounding box of the data on the Northern American continent). Red colours indicate $ \tilde \rho_0 $ values close to 1, and decrease up to 0 through blue and dark colours. The green square represents the centroid of the (cropped) distribution, and the red arrows are the average direction of increase of $ \tilde \rho_0 $ for the northern and southern region to the centroid. The white arrows are local changes of $ \tilde \rho_0 $. Extreme values of $ \rho_0 $ (\ie the non-scaled equivalent of $ \tilde \rho_0 $) can be found in Tab. \ref{tab::R0_min_max} with the proportion of unsuitable patches ($ \rho_0 < 1$). \label{fig::picmar_0}}
\end{figure}

%%%% Picea rubens
\begin{figure}
	\centering
	\begin{tikzpicture}[colorbar arrow/.style={
			shape=single arrow,
			double arrow head extend=0.125cm,
			shape border rotate=90,
			minimum height=8cm,
			shading=#1
		}]
		\node[inner sep=0pt] (picrub) at (0,0)
		    {\includegraphics[scale=0.3]{graphs/18034-PIC-RUB_R0_h=0m_scaled}};
		\node [colorbar arrow=RmapShading] at (6,0) {};
		\node at (6.5,-3.9) {$ 0 $};
		\node at (6.5,3.9) {$ 1 $};
		\node at (6,4.5) {$ \tilde \rho_0 $};
	\end{tikzpicture}
	\caption{Map of \textit{Picea rubens}'s performance ($ \tilde \rho_0 $) without competition (canopy height = $ 0 $ m), and climate data averaged from $ 2006 $ to $ 2010 $. The distribution area is from \citet{Little1971} in eastern Canada and USA (Fig. \ref{fig::mapDatabase} for the bounding box of the data on the Northern American continent). Red colours indicate $ \tilde \rho_0 $ values close to 1, and decrease up to 0 through blue and dark colours. The green square represents the centroid of the (cropped) distribution, and the red arrows are the average direction of increase of $ \tilde \rho_0 $ for the northern and southern region to the centroid. The white arrows are local changes of $ \tilde \rho_0 $. Extreme values of $ \rho_0 $ (\ie the non-scaled equivalent of $ \tilde \rho_0 $) can be found in Tab. \ref{tab::R0_min_max} with the proportion of unsuitable patches ($ \rho_0 < 1$). \label{fig::picrub_0}}
\end{figure}

%%%% Pinus banksiana
\begin{figure}
	\centering
	\begin{tikzpicture}[colorbar arrow/.style={
			shape=single arrow,
			double arrow head extend=0.125cm,
			shape border rotate=90,
			minimum height=8cm,
			shading=#1
		}]
		\node[inner sep=0pt] (pinban) at (0,0)
		    {\includegraphics[scale=0.3]{graphs/183319-PIN-BAN_R0_h=0m_scaled}};
		\node [colorbar arrow=RmapShading] at (6,0) {};
		\node at (6.5,-3.9) {$ 0 $};
		\node at (6.5,3.9) {$ 1 $};
		\node at (6,4.5) {$ \tilde \rho_0 $};
	\end{tikzpicture}
	\caption{Map of \textit{Pinus banksiana}'s performance ($ \tilde \rho_0 $) without competition (canopy height = $ 0 $ m), and climate data averaged from $ 2006 $ to $ 2010 $. The distribution area is from \citet{Little1971} in eastern Canada and USA (Fig. \ref{fig::mapDatabase} for the bounding box of the data on the Northern American continent). Red colours indicate $ \tilde \rho_0 $ values close to 1, and decrease up to 0 through blue and dark colours. The green square represents the centroid of the (cropped) distribution, and the red arrows are the average direction of increase of $ \tilde \rho_0 $ for the northern and southern region to the centroid. The white arrows are local changes of $ \tilde \rho_0 $. Extreme values of $ \rho_0 $ (\ie the non-scaled equivalent of $ \tilde \rho_0 $) can be found in Tab. \ref{tab::R0_min_max} with the proportion of unsuitable patches ($ \rho_0 < 1$). \label{fig::pinban_0}}
\end{figure}

%%%% Pinus strobus
\begin{figure}
	\centering
	\begin{tikzpicture}[colorbar arrow/.style={
			shape=single arrow,
			double arrow head extend=0.125cm,
			shape border rotate=90,
			minimum height=8cm,
			shading=#1
		}]
		\node[inner sep=0pt] (pinstr) at (0,0)
		    {\includegraphics[scale=0.3]{graphs/183385-PIN-STR_R0_h=0m_scaled}};
		\node [colorbar arrow=RmapShading] at (6,0) {};
		\node at (6.5,-3.9) {$ 0 $};
		\node at (6.5,3.9) {$ 1 $};
		\node at (6,4.5) {$ \tilde \rho_0 $};
	\end{tikzpicture}
	\caption{Map of \textit{Pinus strobus}'s performance ($ \tilde \rho_0 $) without competition (canopy height = $ 0 $ m), and climate data averaged from $ 2006 $ to $ 2010 $. The distribution area is from \citet{Little1971} in eastern Canada and USA (Fig. \ref{fig::mapDatabase} for the bounding box of the data on the Northern American continent). Red colours indicate $ \tilde \rho_0 $ values close to 1, and decrease up to 0 through blue and dark colours. The green square represents the centroid of the (cropped) distribution, and the red arrows are the average direction of increase of $ \tilde \rho_0 $ for the northern and southern region to the centroid. The white arrows are local changes of $ \tilde \rho_0 $. Extreme values of $ \rho_0 $ (\ie the non-scaled equivalent of $ \tilde \rho_0 $) can be found in Tab. \ref{tab::R0_min_max} with the proportion of unsuitable patches ($ \rho_0 < 1$). \label{fig::pinstr_0}}
\end{figure}

%%%% Populus tremuloides
\begin{figure}
	\centering
	\begin{tikzpicture}[colorbar arrow/.style={
			shape=single arrow,
			double arrow head extend=0.125cm,
			shape border rotate=90,
			minimum height=8cm,
			shading=#1
		}]
		\node[inner sep=0pt] (poptre) at (0,0)
		    {\includegraphics[scale=0.3]{graphs/195773-POP-TRE_R0_h=0m_scaled}};
		\node [colorbar arrow=RmapShading] at (6,0) {};
		\node at (6.5,-3.9) {$ 0 $};
		\node at (6.5,3.9) {$ 1 $};
		\node at (6,4.5) {$ \tilde \rho_0 $};
	\end{tikzpicture}
	\caption{Map of \textit{Populus tremuloides}'s performance ($ \tilde \rho_0 $) without competition (canopy height = $ 0 $ m), and climate data averaged from $ 2006 $ to $ 2010 $. The distribution area is from \citet{Little1971} in eastern Canada and USA (Fig. \ref{fig::mapDatabase} for the bounding box of the data on the Northern American continent). Red colours indicate $ \tilde \rho_0 $ values close to 1, and decrease up to 0 through blue and dark colours. The green square represents the centroid of the (cropped) distribution, and the red arrows are the average direction of increase of $ \tilde \rho_0 $ for the northern and southern region to the centroid. The white arrows are local changes of $ \tilde \rho_0 $. Extreme values of $ \rho_0 $ (\ie the non-scaled equivalent of $ \tilde \rho_0 $) can be found in Tab. \ref{tab::R0_min_max} with the proportion of unsuitable patches ($ \rho_0 < 1$). \label{fig::poptre_0}}
\end{figure}

%%%% Thuja occidentalis
\begin{figure}
	\centering
	\begin{tikzpicture}[colorbar arrow/.style={
			shape=single arrow,
			double arrow head extend=0.125cm,
			shape border rotate=90,
			minimum height=8cm,
			shading=#1
		}]
		\node[inner sep=0pt] (thuocc) at (0,0)
		    {\includegraphics[scale=0.3]{graphs/505490-THU-OCC_R0_h=0m_scaled}};
		\node [colorbar arrow=RmapShading] at (6,0) {};
		\node at (6.5,-3.9) {$ 0 $};
		\node at (6.5,3.9) {$ 1 $};
		\node at (6,4.5) {$ \tilde \rho_0 $};
	\end{tikzpicture}
	\caption{Map of \textit{Thuja occidentalis}'s performance ($ \tilde \rho_0 $) without competition (canopy height = $ 0 $ m), and climate data averaged from $ 2006 $ to $ 2010 $. The distribution area is from \citet{Little1971} in eastern Canada and USA (Fig. \ref{fig::mapDatabase} for the bounding box of the data on the Northern American continent). Red colours indicate $ \tilde \rho_0 $ values close to 1, and decrease up to 0 through blue and dark colours. The green square represents the centroid of the (cropped) distribution, and the red arrows are the average direction of increase of $ \tilde \rho_0 $ for the northern and southern region to the centroid. The white arrows are local changes of $ \tilde \rho_0 $. Extreme values of $ \rho_0 $ (\ie the non-scaled equivalent of $ \tilde \rho_0 $) can be found in Tab. \ref{tab::R0_min_max} with the proportion of unsuitable patches ($ \rho_0 < 1$). \label{fig::thuocc_0}}
\end{figure}

%%%% Tsuga canadensis
\begin{figure}
	\centering
	\begin{tikzpicture}[colorbar arrow/.style={
			shape=single arrow,
			double arrow head extend=0.125cm,
			shape border rotate=90,
			minimum height=8cm,
			shading=#1
		}]
		\node[inner sep=0pt] (tsucan) at (0,0)
		    {\includegraphics[scale=0.3]{graphs/183397-TSU-CAN_R0_h=0m_scaled}};
		\node [colorbar arrow=RmapShading] at (6,0) {};
		\node at (6.5,-3.9) {$ 0 $};
		\node at (6.5,3.9) {$ 1 $};
		\node at (6,4.5) {$ \tilde \rho_0 $};
	\end{tikzpicture}
	\caption{Map of \textit{Tsuga canadensis}'s performance ($ \tilde \rho_0 $) without competition (canopy height = $ 0 $ m), and climate data averaged from $ 2006 $ to $ 2010 $. The distribution area is from \citet{Little1971} in eastern Canada and USA (Fig. \ref{fig::mapDatabase} for the bounding box of the data on the Northern American continent). Red colours indicate $ \tilde \rho_0 $ values close to 1, and decrease up to 0 through blue and dark colours. The green square represents the centroid of the (cropped) distribution, and the red arrows are the average direction of increase of $ \tilde \rho_0 $ for the northern and southern region to the centroid. The white arrows are local changes of $ \tilde \rho_0 $. Extreme values of $ \rho_0 $ (\ie the non-scaled equivalent of $ \tilde \rho_0 $) can be found in Tab. \ref{tab::R0_min_max} with the proportion of unsuitable patches ($ \rho_0 < 1$). \label{fig::tsucan_0}}
\end{figure}


\begin{figure}[htb]
    \centering
	%% First row
	\begin{subfigure}{0.25\textwidth}
		\input{graphs/azimuth_ABI-BAL_0}
		\caption{\textit{Abies balsamea}}
		\label{fig::abibal_az_0}
	\end{subfigure}
	\hfil
	\begin{subfigure}{0.25\textwidth}
		\input{graphs/azimuth_ACE-RUB_0}
		\caption{\textit{Acer rubrum}}
		\label{fig::acerub_az_0}
	\end{subfigure}
	\hfil
	\begin{subfigure}{0.25\textwidth}
		\input{graphs/azimuth_ACE-SAC_0}
		\caption{\textit{Acer saccharum}}
		\label{fig::acesac_az_0}
	\end{subfigure}
	\medskip
	%% Second row
	\begin{subfigure}{0.25\textwidth}
		\input{graphs/azimuth_BET-ALL_0}
		\caption{\textit{Betula alleghaniensis}}
		\label{fig::betall_az_0}
	\end{subfigure}
	\hfil
	\begin{subfigure}{0.25\textwidth}
		\input{graphs/azimuth_BET-PAP_0}
		\caption{\textit{Betula papyrifera}}
		\label{fig::betpap_az_0}
	\end{subfigure}
	\hfil
	\begin{subfigure}{0.25\textwidth}
		\input{graphs/azimuth_FAG-GRA_0}
		\caption{\textit{Fagus grandifolia}}
		\label{fig::faggran_az_0}
	\end{subfigure}
	\medskip
	%% Third row
	\begin{subfigure}{0.25\textwidth}
		\input{graphs/azimuth_PIC-GLA_0}
		\caption{\textit{Picea glauca}}
		\label{fig::picgla_az_0}
	\end{subfigure}
	\hfil
	\begin{subfigure}{0.25\textwidth}
		\input{graphs/azimuth_PIC-MAR_0}
		\caption{\textit{Picea mariana}}
		\label{fig::picmar_az_0}
	\end{subfigure}
	\hfil
	\begin{subfigure}{0.25\textwidth}
		\input{graphs/azimuth_PIC-RUB_0}
		\caption{\textit{Picea rubens}}
		\label{fig::picrub_az_0}
	\end{subfigure}
	\medskip
	%% Forth row
	\begin{subfigure}{0.25\textwidth}
		\input{graphs/azimuth_PIN-BAN_0}
		\caption{\textit{Pinus banksiana}}
		\label{fig::pinban_az_0}
	\end{subfigure}
	\hfil
	\begin{subfigure}{0.25\textwidth}
		\input{graphs/azimuth_PIN-STR_0}
		\caption{\textit{Pinus strobus}}
		\label{fig::pinstr_az_0}
	\end{subfigure}
	\hfil
	\begin{subfigure}{0.25\textwidth}
		\input{graphs/azimuth_POP-TRE_0}
		\caption{\textit{Populus tremuloides}}
		\label{fig::poptre_az_0}
	\end{subfigure}
	\medskip
	%% Fifth row
	\begin{subfigure}{0.25\textwidth}
		\input{graphs/azimuth_THU-OCC_0}
		\caption{\textit{Thuja occidentalis}}
		\label{fig::thuocc_az_0}
	\end{subfigure}
	\hfil
	\begin{subfigure}{0.25\textwidth}
		\input{graphs/azimuth_TSU-CAN_0}
		\caption{\textit{Tsuga canadensis}}
		\label{fig::tsucan_az_0}
	\end{subfigure}
	\hfil
	\begin{subfigure}{0.25\textwidth}
		\input{graphs/legend_cols}
	\end{subfigure}
	\caption{Species-specific averaged direction of the gradients for the northern region (blue arrows) and southern region (orange arrows), without competition. The azimuths can be found in Tab. \ref{tab::azimuth_0}. \label{fig::grad_cols_0}}
\end{figure}

\subsection{Table of $ \rho_0 $ (\ie non-scaled)}
\begin{table}[ht]
\centering
\caption{Extreme values of $ \rho_0 $ (\ie the non-scaled equivalent of $ \tilde \rho_0 $) with and without competition. Proportion columns are the percentage of unsuitable patches within $ \Omega $. A location $ x $ is unsuitable if $ \rho_0(x, \s) < 1 $, where 1 is the threshold for a species to maintain itself. \label{tab::R0_min_max}}
\begin{tabular}{lcccccc}
	\toprule
		~ & \multicolumn{3}{c}{\textbf{With competition ($ \bm {\s = 10} $ m)}} & \multicolumn{3}{c}{\textbf{Without competition ($ \bm {\s = 0} $ m)}} \\
	\cmidrule(lr){2-4} \cmidrule(lr){5-7}
		\textbf{Species} & $ \bm{\max(\rho_0)} $ & $ \bm{\min(\rho_0)} $ & \textbf{Proportion} & $ \bm{\max(\rho_0)} $ & $ \bm{\min(\rho_0)} $ & \textbf{Proportion} \\
	\midrule
		ABI-BAL & 7.64 & 0.00 & 55.57 & 36.30 & 5.60 & 0.00 \\
		ACE-RUB & 12.19 & 0.00 & 23.83 & 47.52 & 0.58 & 0.20 \\
		ACE-SAC & 160.86 & 13.60 & 0.00 & 426.13 & 20.70 & 0.00 \\
		BET-ALL & 324.95 & 17.29 & 0.00 & 643.71 & 50.71 & 0.00 \\
		BET-PAP & 10.79 & 0.00 & 51.97 & 46.45 & 0.21 & 0.35 \\
		FAG-GRA & 131.75 & 4.79 & 0.00 & 282.56 & 26.38 & 0.00 \\
		PIC-GLA & 49.53 & 0.00 & 11.12 & 136.54 & 7.31 & 0.00 \\
		PIC-MAR & 1.29 & 0.00 & 96.12 & 19.94 & 3.20 & 0.00 \\
		PIC-RUB & 53.69 & 2.06 & 0.00 & 94.42 & 12.25 & 0.00 \\
		PIN-BAN & 2.59 & 0.00 & 99.16 & 19.19 & 1.85 & 0.00 \\
		PIN-STR & 97.72 & 0.26 & 0.41 & 252.96 & 2.46 & 0.00 \\
		POP-TRE & 9.39 & 0.00 & 66.21 & 47.63 & 1.34 & 0.00 \\
		THU-OCC & 51.16 & 0.02 & 0.01 & 122.97 & 6.48 & 0.00 \\
		TSU-CAN & 221.87 & 0.00 & 0.05 & 366.56 & 19.49 & 0.00 \\
	\bottomrule
\end{tabular}
\end{table}

\printbibliography[heading=subbibliography]
\end{refsection}


\documentclass[letterpaper, 12pt]{article}

%%%%%%%%%%%%%%%%%%%	PACKAGES	%%%%%%%%%%%%%%%%%%%
%% Package version
\listfiles % Then check the .log file

%% Font
\usepackage{fontspec}
\usepackage{marvosym}

%% Languages
\usepackage{polyglossia}
	\setdefaultlanguage[variant=british]{english}

%% Marges, space...
\usepackage[top=2.5cm, bottom=2.5cm, left=2.5cm, right=2.5cm]{geometry}
\usepackage{setspace} % [nodisplayskipstretch] pour option space in equation

\usepackage{indentfirst}

\usepackage[bottom]{footmisc}
\usepackage{footnote}

% https://tex.stackexchange.com/questions/279/how-do-i-ensure-that-figures-appear-in-the-section-theyre-associated-with
\usepackage[section]{placeins} % To place figures in the section it is declared

%% Graphics
\usepackage[luatex]{graphicx}
	\graphicspath{{../graphs/}}

\usepackage{caption}
\usepackage{subcaption}
	\captionsetup{subrefformat=parens}

\makeatletter 
\renewcommand{\thefigure}{S\thesection.\@arabic\c@figure}
\renewcommand{\thetable}{S\thesection.\@arabic\c@table}
\makeatother

\usepackage[dvipsnames, svgnames]{xcolor}

\usepackage{tikz}
	\usetikzlibrary{arrows, plotmarks, decorations.markings}
	\tikzstyle{arrow} = [->,>=stealth,thick,rounded corners=4pt,line width=1pt]
	\usetikzlibrary{shadows}
	\usetikzlibrary{shadings}
	\usetikzlibrary{positioning} % relative coodinate
	\usetikzlibrary{tikzmark, calc} % calc, to calculate coordinate
	\usetikzlibrary{decorations.pathmorphing} % to snake an arrow
	\usetikzlibrary{shapes.arrows}
	\usetikzlibrary{patterns}

%% Table of content
\addto\captionsenglish{ % Replace "english" with the language used in babel
	\renewcommand{\contentsname}{Supporting Information Appendix S6 contents}
}

%% Format section
\usepackage{titlesec}
\titlelabel{S\thetitle~}
\usepackage{titletoc}
\titlecontents{section}[0pt]{}{\bfseries S\thecontentslabel~}{\bfseries}{\hspace{1em plus 1fill}\contentspage}

\usepackage{enumitem}% http://ctan.org/pkg/enumitem

%% Links
\usepackage{url}
\usepackage[luatex, colorlinks=true, linkcolor=NavyBlue, urlcolor=MidnightBlue, citecolor=PineGreen]{hyperref}

%% Table
\usepackage{booktabs}

%% bibliography
\usepackage{csquotes}
\usepackage[style=apa, natbib=true, sorting=ynt]{biblatex}
\addbibresource{bib_randomForest.bib}

%% mathematics
\usepackage{amsthm}
\usepackage{amsmath}
	\allowdisplaybreaks % Autoriser découpe formules entres pages
\usepackage{amssymb}
\usepackage{bbold}
\usepackage{dsfont}
\usepackage{mathrsfs}
\usepackage{bm}
\usepackage{xfrac}
\usepackage{etoolbox} % For renumbering (cf below, counter for model)

\usepackage{thmbox} % cf after for "theorem" definitions

\usepackage{gensymb}

\usepackage{siunitx}

%%%%%%%%%%%%%%%%%   NEW COMMANDES	%%%%%%%%%%%%%%%%%
%% Text
\newcommand {\ie}{\textit{i.e., }}
\newcommand {\eg}{\textit{e.g., }}
\newcommand {\cf}{\textit{cf} }
\newcommand\bsc[1]{\textsc{\MakeLowercase{#1}}} % Only if there is no french babel
\newcommand {\thup}[1]{{#1}\textsuperscript{th}}

%% Math
\newcommand {\s}{{s}^{*}}
\newcommand {\sst}{\tilde{s}^{*}} % s* stable d'où le st
\newcommand {\N}{\tilde{N}}
\newcommand {\A}{\mathscr{A}}
\newcommand {\K}{\mathcal{K}}
\renewcommand{\S}{\mathscr{S}}
\newcommand{\R}{\mathds{R}}
\newcommand{\Prob}{\mathds{P}}
\newcommand{\F}{\mathcal{F}}

\DeclareMathOperator{\logit}{logit}

%%%%%%%%%%%%%%%%%   THEOREM STYLE	%%%%%%%%%%%%%%%%%
\newtheoremstyle{theo}{\topsep}{\topsep}{\itshape}{}{\bfseries}{.}{\newline}{\thmname{#1} \thmnumber{#2} \thmnote{~: \textit{#3}}}
\theoremstyle{theo}
\newtheorem{rem}{Remark}[section]
\newtheorem{defi}{Definition}[section]
\newtheorem{assum}{Assumption}[section]
\newtheorem{nota}{Notation}[section]

%%%%%%%%%%%%%%%%%   SET COUNTER		%%%%%%%%%%%%%%%%%
\setcounter{section}{5}

\begin{document}
\tableofcontents
\listoftables
\listoffigures
\newpage

\begin{refsection}
\begin{onehalfspace}

\section{Random forest and correlations} \label{app::randomForest}
The equation used in the random forest is:
\begin{equation}
	\begin{split}
	\text{presence species} \sim T_a +{} & r_T + I + T_s + T_M + T_m + T_r + T_w + T_d + T_h + T_c \\
		& P_a + P_M + P_m + P_s + P_w + P_d + P_h + P_c
	\end{split}
\end{equation}
Where the $ T $s denote temperatures, and the $ P $s the precipitations (all defined in Table S2.3). The number of trees is set to 2000, and the number of variables to select at each node is set to 12.

\subsection{Correlations between the random forest or presence/absence data versus $ \tilde \rho_0 $}
\begin{table}[!h]
\centering
\caption[Correlations $ \tilde \rho_0 $ and random forest]{Correlations between $ \tilde \rho_0 $ derived from our model, and the SDM (random forest). We trained the random forest with \num{61374} data, and evaluate its prediction accuracy using the $ R^2 $ from \citet{Tjur2009}. Correlations were calculated without competition (\ie $ \s = 0 $, correlation 0), and with a competition of 10 meters (\ie $ \s = 10 $, correlation 10). The last column is the correlation between $ \tilde \rho_0 $ and the distance to the closest edge of the distribution defined by \citet{Little1971}. \label{tab::R0correlSDM}}
\begin{tabular}{@{}rccc@{}}
	\toprule
	\textbf{Species} & \textbf{Correlation 0} & \textbf{Correlation 10} & $ \bm{R^2} $ \textbf{(Tjur)} \\
	\midrule
		ABI-BAL & -0.09 & 0.48 & 0.85 \\
		ACE-RUB & -0.49 & -0.50 & 0.78 \\
		ACE-SAC & -0.07 & -0.18 & 0.78 \\
		BET-ALL & -0.30 & -0.25 & 0.77 \\
		BET-PAP & 0.61 & 0.48 & 0.78 \\
		FAG-GRA & -0.37 & -0.37 & 0.74 \\
		PIC-GLA & 0.41 & 0.08 & 0.76 \\
		PIC-MAR & 0.52 & 0.27 & 0.82 \\
		PIC-RUB & 0.18 & -0.26 & 0.83 \\
		PIN-BAN & 0.15 & -0.10 & 0.77 \\
		PIN-STR & 0.09 & 0.02 & 0.74 \\
		POP-TRE & 0.31 & 0.45 & 0.79 \\
		THU-OCC & 0.33 & 0.24 & 0.74 \\
		TSU-CAN & 0.14 & 0.17 & 0.74 \\
	\bottomrule
\end{tabular}
\end{table}

\begin{table}[!h]
\centering
\caption[Correlations $ \tilde \rho_0 $ and presence/absence data]{Species-specific correlations between $ \tilde \rho_0 $ and the data of presence and absence. \label{tab::corr_R0_presAbsData}}
\begin{tabular}{rcc}
	\toprule
	\textbf{Species} & \textbf{Correlation 0} & \textbf{Correlation 10} \\
	\midrule
		ABI-BAL & -0.08 & 0.42 \\
		ACE-RUB & -0.40 & -0.41 \\
		ACE-SAC & -0.06 & -0.15 \\
		BET-ALL & -0.24 & -0.20 \\
		BET-PAP & 0.49 & 0.39 \\
		FAG-GRA & -0.28 & -0.28 \\
		PIC-GLA & 0.32 & 0.06 \\
		PIC-MAR & 0.44 & 0.23 \\
		PIC-RUB & 0.15 & -0.22 \\
		PIN-BAN & 0.12 & -0.08 \\
		PIN-STR & 0.07 & 0.01 \\
		POP-TRE & 0.25 & 0.37 \\
		THU-OCC & 0.25 & 0.18 \\
		TSU-CAN & 0.11 & 0.13 \\
 	\bottomrule
\end{tabular}
\end{table}

\begin{figure}[!h]
	\centering
	\input{./correl_rf_vs_pa}
	\caption[$ \text{Cor}(\tilde \rho_0, \text{pres/abs}) $ vs $ \text{Cor}(\tilde \rho_0, \text{SDM}) $]{Correlations between $ \tilde \rho_0 $ and presence/absence data (x-axis) or between $ \tilde \rho_0 $ and the random forest (y-axis). The correlations are either without competition (\MoveUp), or with competition (canopy height $ \s = 10 $ m, \CircSteel). The values are from Tabs. \ref{tab::R0correlSDM} and \ref{tab::corr_R0_presAbsData}. \label{fig::correl_rf_vs_presAbs}}
\end{figure}

\newpage
\subsection{Correlation between $ \tilde \rho_0 $ and distance to closest edge}
\begin{table}[ht]
\centering
\caption[$ \text{Cor}(\tilde \rho_0, \text{distance closest edge}) $]{Species-specific correlations between $ \tilde \rho_0 $ and the distance to the closest edge of $ \Omega $, with competition (canopy height $ \s = 10 $ m) and without (canopy height $ \s = 0 $ m). The correlations are computed over the whole species distribution $ \Omega $, over the points closer to a border belonging to the northern region, or closer to a border belonging to the southern region. The data are used in Fig. \ref{fig::3correls_dist}. \label{tab::R0correl_dist}}
\begin{tabular}{lcccccc}
	\toprule
	~ & \multicolumn{2}{c}{\textbf{Over} $ \bm \Omega $} & \multicolumn{2}{c}{\textbf{Northern region}} & \multicolumn{2}{c}{\textbf{Southern region}} \\
		\cmidrule(lr){2-3} \cmidrule(lr){4-5} \cmidrule(lr){6-7}
		\textbf{species} & \textbf{Cor 0} & \textbf{Cor 10} & \textbf{Cor 0} & \textbf{Cor 10} & \textbf{Cor 0} & \textbf{Cor 10} \\
	\midrule
		ABI-BAL & -0.18 & -0.33 & 0.16 & -0.01 & -0.29 & -0.42 \\
		ACE-RUB & -0.33 & -0.36 & -0.23 & -0.16 & 0.21 & 0.13 \\
		ACE-SAC & -0.21 & -0.11 & 0.12 & 0.15 & -0.04 & -0.06 \\
		BET-ALL & -0.16 & -0.11 & -0.10 & -0.12 & -0.34 & -0.22 \\
		BET-PAP & 0.26 & -0.09 & 0.22 & 0.27 & 0.37 & -0.06 \\
		FAG-GRA & -0.12 & -0.14 & -0.21 & -0.20 & -0.55 & -0.57 \\
		PIC-GLA & 0.33 & 0.29 & -0.08 & -0.08 & 0.47 & 0.45 \\
		PIC-MAR & 0.29 & -0.31 & 0.16 & -0.04 & 0.43 & -0.26 \\
		PIC-RUB & -0.32 & -0.44 & -0.10 & -0.37 & -0.51 & -0.51 \\
		PIN-BAN & -0.31 & -0.27 & 0.24 & -0.00 & -0.48 & -0.34 \\
		PIN-STR & 0.16 & 0.06 & 0.06 & 0.05 & 0.32 & 0.08 \\
		POP-TRE & 0.06 & 0.08 & 0.10 & 0.28 & -0.03 & 0.10 \\
		THU-OCC & 0.21 & 0.29 & -0.21 & 0.13 & 0.40 & 0.27 \\
		TSU-CAN & 0.13 & 0.09 & 0.16 & 0.11 & 0.43 & 0.43 \\
 	\bottomrule
\end{tabular}
\end{table}

\begin{figure}
	\centering
	\input{./3correlations_proj10}
	\caption[$ \text{Cor}(\tilde \rho_0, \text{distance closest edge}) $, $ s^{*} = 10 $ m]{Species-specific correlations between $ \tilde \rho_0 $ and the distance to the closest edge of $ \Omega $, with competition (canopy height $ \s = 10 $ m). The correlations are computed over the whole species distribution (\CircSteel), over the points closer to a border within the northern region (\MoveUp) or closer to a border within the southern region (\MoveDown). A negative correlation indicates that $ \rho_0 $ increases toward north (\MoveUp) or south (\MoveDown). The three colours correspond to the shade tolerance level (the darker, the more shade-tolerant). The values can be found in Tab \ref{tab::R0correl_dist}. \label{fig::3correls_dist}}
\end{figure}

\begin{figure}
	\centering
	\input{./3correlations_proj0}
	\caption[$ \text{Cor}(\tilde \rho_0, \text{distance closest edge}) $, no competition]{Species-specific correlations between $ \tilde \rho_0 $ and the distance to the closest edge of $ \Omega $, without competition (canopy height $ \s = 0 $ m). The correlations are computed over the whole species distribution (\CircSteel), over the points closer to a border within the northern region (\MoveUp) or closer to a border within the southern region (\MoveDown). A negative correlation indicates that $ \rho_0 $ increases toward north (\MoveUp) or south (\MoveDown). The three colours correspond to the shade tolerance level (the darker, the more shade-tolerant). The values can be found in Tab \ref{tab::R0correl_dist}. \label{fig::3correls_dist_0}}
\end{figure}

\subsection{Correlation between demography and probabilities of occurrence}
\begin{figure}
	\centering
	\input{./demog_Pocc1-4}
	\caption[$ P_{\text{occ}} $ vs vital rates, species 1-4]{Species-specific correlations between probabilities of occurrence and the vital rates (growth: \CircSteel, and mortality: $ \times $). We added the data of Fig. \ref{fig::3correls} (\MoveUp) which represent the correlation between $ P_{\text{occ}} $ and $ \tilde \rho_0 $. For both demographic rates, we set the dbh at either $ 15 $ cm or $ 60 $ cm (to represent saplings and adults, we expect the saplings to have stronger correlations \citep{Kunstler2019}), while $ \tilde \rho_0 $ integrates all the possible sizes. We used an averaged climate from 2006 to 2010 at the permanent plots (which were the training dataset of the random forest). Dark blue colour is for the rates with competition, and light blue for those without competition). \label{fig::demog_Pocc1-4}}
\end{figure}

\begin{figure}
	\centering
	\input{./demog_Pocc5-8}
	\caption[$ P_{\text{occ}} $ vs vital rates, species 5-8]{Species-specific correlations between probabilities of occurrence and the vital rates (growth: \CircSteel, and mortality: $ \times $). We added the data of Fig. \ref{fig::3correls} (\MoveUp) which represent the correlation between $ P_{\text{occ}} $ and $ \tilde \rho_0 $. For both demographic rates, we set the dbh at either $ 15 $ cm or $ 60 $ cm (to represent saplings and adults, we expect the saplings to have stronger correlations \citep{Kunstler2019}), while $ \tilde \rho_0 $ integrates all the possible sizes. We used an averaged climate from 2006 to 2010 at the permanent plots (which were the training dataset of the random forest). Dark blue colour is for the rates with competition, and light blue for those without competition). \label{fig::demog_Pocc5-8}}
\end{figure}

\begin{figure}
	\centering
	\input{./demog_Pocc9-12}
	\caption[$ P_{\text{occ}} $ vs vital rates, species 9-12]{Species-specific correlations between probabilities of occurrence and the vital rates (growth: \CircSteel, and mortality: $ \times $). We added the data of Fig. \ref{fig::3correls} (\MoveUp) which represent the correlation between $ P_{\text{occ}} $ and $ \tilde \rho_0 $. For both demographic rates, we set the dbh at either $ 15 $ cm or $ 60 $ cm (to represent saplings and adults, we expect the saplings to have stronger correlations \citep{Kunstler2019}), while $ \tilde \rho_0 $ integrates all the possible sizes. We used an averaged climate from 2006 to 2010 at the permanent plots (which were the training dataset of the random forest). Dark blue colour is for the rates with competition, and light blue for those without competition). \label{fig::demog_Pocc9-12}}
\end{figure}

\begin{figure}
	\centering
	\input{./demog_Pocc13-14}
	\caption[$ P_{\text{occ}} $ vs vital rates, species 13-14]{Species-specific correlations between probabilities of occurrence and the vital rates (growth: \CircSteel, and mortality: $ \times $). We added the data of Fig. \ref{fig::3correls} (\MoveUp) which represent the correlation between $ P_{\text{occ}} $ and $ \tilde \rho_0 $. For both demographic rates, we set the dbh at either $ 15 $ cm or $ 60 $ cm (to represent saplings and adults, we expect the saplings to have stronger correlations \citep{Kunstler2019}), while $ \tilde \rho_0 $ integrates all the possible sizes. We used an averaged climate from 2006 to 2010 at the permanent plots (which were the training dataset of the random forest). Dark blue colour is for the rates with competition, and light blue for those without competition). \label{fig::demog_Pocc13-14}}
\end{figure}

\end{onehalfspace}

\clearpage
\printbibliography[heading=subbibliography]
\end{refsection}

\end{document}


\end{document}
