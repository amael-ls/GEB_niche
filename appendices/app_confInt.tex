
\documentclass[letterpaper, 12pt]{article}

%%%%%%%%%%%%%%%%%%%	PACKAGES	%%%%%%%%%%%%%%%%%%%
%% Font
\usepackage{fontspec}
\usepackage{marvosym}

%% Languages
\usepackage{polyglossia}
	\setdefaultlanguage[variant=british]{english}

%% Table of content
\addto\captionsenglish{ % Replace "english" with the language used in babel
	\renewcommand{\contentsname}{Supporting Information Appendix S5 contents}
}

%% Marges, space...
\usepackage[top=2.5cm, bottom=2.5cm, left=2.5cm, right=2.5cm]{geometry}

%% Graphics
\usepackage[luatex]{graphicx}

\usepackage{tikz}
	\usetikzlibrary{arrows, plotmarks, decorations.markings}
	\tikzstyle{arrow} = [->,>=stealth,thick,rounded corners=4pt,line width=1pt]
	\usetikzlibrary{shadows}
	\usetikzlibrary{shadings}
	\usetikzlibrary{positioning} % relative coodinate
	\usetikzlibrary{tikzmark, calc} % calc, to calculate coordinate
	\usetikzlibrary{decorations.pathmorphing} % to snake an arrow
	\usetikzlibrary{shapes.arrows}
	\usetikzlibrary{patterns}
	\tikzset{
		invisible/.style={opacity=0},
		visible on/.style={alt={#1{}{invisible}}},
		alt/.code args={<#1>#2#3}{%
		\alt<#1>{\pgfkeysalso{#2}}{\pgfkeysalso{#3}} % \pgfkeysalso doesn't change the path
		},
	} % end tikzset. Code from http://tex.stackexchange.com/questions/136143/tikz-animated-figure-in-beamer

\makeatletter
\renewcommand{\thefigure}{S\thesection.\@arabic\c@figure}
\renewcommand{\thetable}{S\thesection.\@arabic\c@table}
\makeatother

%% mathematics
\usepackage{amsthm}
\usepackage{amsmath}
\usepackage{amssymb}
\usepackage{bbold}
\usepackage{dsfont}
\usepackage{mathrsfs}
\usepackage{bm}
\usepackage{xfrac}

\usepackage{thmbox} % cf after for "theorem" definitions

\usepackage{gensymb}
\newcommand {\s}{{s}^{*}}

%%%%%%%%%%%%%%%%%   SET COUNTER		%%%%%%%%%%%%%%%%%
\setcounter{section}{3}

\begin{document}
\tableofcontents

\section{Confindence intervals} \label{app::confInt}
\subsection{Growth}

\begin{figure}[h]
	\centering
	\input{confIntPlot/page_1_growth}
	\caption{Confidence intervals for growth, $ \s $ is the growth response to light. \label{fig::confInt_g_1}}
\end{figure}

\begin{figure}
	\centering
	\input{confIntPlot/page_2_growth}
	\caption{Confidence intervals for growth. \label{fig::confInt_g_2}}
\end{figure}

\begin{figure}
	\centering
	\input{confIntPlot/page_3_growth}
	\caption{Confidence intervals for growth, $ \s $ is the growth response to light. The colon `$ : $' denotes an interaction between two variables. \label{fig::confInt_g_3}}
\end{figure}

\begin{figure}
	\centering
	\input{confIntPlot/page_4_growth}
	\caption{Confidence intervals for growth. The colon `$ : $' denotes an interaction between two variables. \label{fig::confInt_g_4}}
\end{figure}

\begin{figure}
	\centering
	\input{confIntPlot/page_5_growth}
	\caption{Confidence intervals for growth. The colon `$ : $' denotes an interaction between two variables. \label{fig::confInt_g_5}}
\end{figure}

\clearpage
\subsection{Mortality}
\begin{figure}[h]
	\centering
	\input{confIntPlot/page_1_mortality}
	\caption{Confidence intervals for mortality, $ \s $ is the mortality response to light. \label{fig::confInt_m_1}}
\end{figure}

\begin{figure}
	\centering
	\input{confIntPlot/page_2_mortality}
	\caption{Confidence intervals for mortality. \label{fig::confInt_m_2}}
\end{figure}

\begin{figure}
	\centering
	\input{confIntPlot/page_3_mortality}
	\caption{Confidence intervals for mortality, $ \s $ is the mortality response to light. The colon `$ : $' denotes an interaction between two variables. \label{fig::confInt_m_3}}
\end{figure}

\begin{figure}[h]
	\centering
	\input{rhat_distrib}
	\caption{Distribution of $ \hat{r} $ for all the parameters (including group effects). The threshold 1.05 is a commonly chosen limit to pinpoints a convergence problem from a chain. At convergence, the $ \hat{r} $ should be 1. The trace plots of the chains and posterior distributions of the parameters are available in Supporting Information Figures S7. \label{fig::rhat_conv}}
\end{figure}

\end{document}
