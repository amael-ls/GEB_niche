
\documentclass[letterpaper, 12pt]{article}

%%%%%%%%%%%%%%%%%%%    PACKAGES    %%%%%%%%%%%%%%%%%%%
%% Package version
\listfiles % Then check the .log file

%% Font
\usepackage{fontspec}
\usepackage{marvosym}

%% Languages
\usepackage{polyglossia}
	\setdefaultlanguage[variant=british]{english}

%% Marges, space...
\usepackage[top=2.5cm, bottom=2.5cm, left=2.5cm, right=2.5cm]{geometry}
\usepackage{setspace} % [nodisplayskipstretch] pour option space in equation

\usepackage{indentfirst}

\usepackage[bottom]{footmisc}
\usepackage{footnote}

% https://tex.stackexchange.com/questions/279/how-do-i-ensure-that-figures-appear-in-the-section-theyre-associated-with
\usepackage[section]{placeins} % To place figures in the section it is declared

%% Graphics
\usepackage[luatex]{graphicx}
	\graphicspath{{graphs/}}

\makeatletter 
\renewcommand{\thefigure}{S\thesection.\@arabic\c@figure}
\makeatother

\usepackage[dvipsnames, svgnames]{xcolor}

%% Table of content
\addto\captionsenglish{ % Replace "english" with the language used in babel
	\renewcommand{\contentsname}{Supporting Information Appendix S1 contents}
}

\makeatletter 
\renewcommand{\thefigure}{S\thesection.\@arabic\c@figure}
\renewcommand{\thetable}{S\thesection.\@arabic\c@table}
\renewcommand{\theequation}{S\thesection.\arabic{equation}}
\makeatother

%% Format section
\usepackage{titlesec}
\titlelabel{S\thetitle~}
\usepackage{titletoc}
\titlecontents{section}[0pt]{}{\bfseries S\thecontentslabel~}{\bfseries}{\hspace{1em plus 1fill}\contentspage}


\usepackage{enumitem}% http://ctan.org/pkg/enumitem

%% Links
\usepackage{url}
\usepackage[luatex, colorlinks=true, linkcolor=NavyBlue, urlcolor=MidnightBlue, citecolor=PineGreen]{hyperref}

%% Table
\usepackage{booktabs}

%% bibliography
\usepackage{csquotes}
\usepackage[style=apa, natbib=true, sorting=ynt]{biblatex}
\addbibresource{bib_calc_R0.bib}

%% mathematics
\usepackage{amsthm}
\usepackage{amsmath}
	\allowdisplaybreaks % Autoriser découpe formules entres pages
\usepackage{amssymb}
\usepackage{bbold}
\usepackage{dsfont}
\usepackage{mathrsfs}
\usepackage{bm}
\usepackage{xfrac}
\usepackage{etoolbox} % For renumbering (cf below, counter for model)

\usepackage{thmbox} % cf after for "theorem" definitions

\usepackage{gensymb}

%%%%%%%%%%%%%%%%%   NEW COMMANDES    %%%%%%%%%%%%%%%%%
%% Text
\newcommand {\ie}{\textit{i.e., }}
\newcommand {\eg}{\textit{e.g., }}
\newcommand {\cf}{\textit{cf} }
\newcommand\bsc[1]{\textsc{\MakeLowercase{#1}}} % Only if there is no french babel
\newcommand {\thup}[1]{{#1}\textsuperscript{th}}

%% Math
\newcommand {\s}{{s}^{*}}
\newcommand {\sst}{\tilde{s}^{*}} % s* stable d'où le st
\newcommand {\N}{\tilde{N}}
\newcommand {\A}{\mathscr{A}}
\newcommand {\K}{\mathcal{K}}
\renewcommand{\S}{\mathscr{S}}
\newcommand{\R}{\mathds{R}}
\newcommand{\Prob}{\mathds{P}}
\newcommand{\F}{\mathcal{F}}

\DeclareMathOperator{\logit}{logit}

%%%%%%%%%%%%%%%%%   THEOREM STYLE    %%%%%%%%%%%%%%%%%
\newtheoremstyle{theo}{\topsep}{\topsep}{\itshape}{}{\bfseries}{.}{\newline}{\thmname{#1} \thmnumber{#2} \thmnote{~: \textit{#3}}}
\theoremstyle{theo}
\newtheorem{rem}{Remark}[section]
\newtheorem{defi}{Definition}[section]
\newtheorem{assum}{Assumption}[section]
\newtheorem{nota}{Notation}[section]

\begin{document}
\tableofcontents

\begin{refsection}
\begin{onehalfspace}

\section{Derivation of the net reproduction rate $ R_0 $ and some of its properties} \label{app::calc_R0}
In this appendix, we derive the net reproduction rate $ R_0(x, \s) $ for a given location $ x $ and a canopy height $ \s $. Then, we prove that $ R_0(x, \s) $ is a decreasing function of $ \s $ which allows us to make our interpretation in the results section. We suppose that the dispersal kernel $ \K $ is a Dirac, in other words, seeds emitted from $ x $ remain in their source patch $ x $. Finally, we compare our study to \citet{Purves2009}.

\subsection{Derivation of $ R_0 $}
\begin{figure}[h!]
	\centering
	\includegraphics[scale = 0.75]{../illustrations/chara}
	\caption{Characteristics for a constant competition $ \s_c = 2 $, a climatic constant $ \xi_{x, i} = 1 $, and for coefficients $ \beta_0 = \ln(3) $, $ \beta_1 = 1 $ and $ \beta_2 = -1 $. These curves are parameterised by $ \theta $ and allow us to follow a cohort with relative time $ \theta $}
	\label{fig::chara}
\end{figure}
We use the method of characteristics \citep{Olver2014b} which allows us to follow individuals through their life, \ie the characteristics represent the trajectories of individuals in the time-size plane (fig. \ref{fig::chara} for an example). The transport equation (equation 1 in the main text) requires an initial population at time $ t = 0 $, which we denote by $ \phi(s) $. It is the density of trees of size $ s $ at the beginning. Let the canopy height $ \s $ be known with a constant value $ \s_c $, and, let $ s $ and $ t $ be functions of a parameter $ \theta $. By applying the chain rule, we get:
\[
	\frac{d N\big(s(\theta), t(\theta) \big)}{d \theta} = \left. \frac{\partial N}{\partial s} \right|_{(s(\theta), t(\theta))} \frac{ds}{d \theta} +
			\left. \frac{\partial N}{\partial t} \right|_{(s(\theta), t(\theta))} \frac{dt}{d \theta}
\]
We therefore set the characteristic equations to:
\begin{align}
	\frac{ds}{d \theta} &= G(s, \s_c) \label{eq::chara_s} \\
		&= \xi_{x, i} e^{\beta_0 \mathbb{1}_{[\s_c, \infty[}(s) + \beta_1 s + \beta_2 s^2} \nonumber \\
	\frac{dt}{d \theta} &= 1 \label{eq::chara_t}
\end{align}
where $ \xi_{x, i} $ is a species-specific (due to the $ i $ index) constant depending of the climate at the location $ x $, and $ \mathbb{1}_A $ is the indicator function (\ie $ \mathbb{1}_A (s) $ equals 1 if $ s \in A $ and 0 otherwise). Hence, equation 1 (in the paper) is along the characteristics:
\begin{equation}
	\frac{dN}{d\theta} = - \left\{ \frac{\partial G}{\partial s}(s, \s, x) + \mu(s, \s, x) \right\} N(\theta) \label{eq::ODE_N}
\end{equation}
\begin{nota}[Abusing notations]
	We are abusing the notations $ N(\theta) $ and $ N\big( s(\theta), t(\theta) \big) $ to make equations easier to read. It would have been more correct to write
	\[
		\widetilde{N}(\theta) = N \big(s(\theta), t(\theta) \big)
	\]
\end{nota}

For the sake of readability, we drop the $ x $ in $ G $ and $ \mu $, and write $ \exp $ for the exponential function instead of $ e^x $. The solution of equation \eqref{eq::ODE_N} is:
\begin{equation} \label{eq::sol_ODE_N}
	N(\theta_1) = N(\theta = 0) \exp \left[-\int_0^{\theta_1} \mu(s, \s_c) + \frac{\partial G}{\partial s}(s, \s_c) \, d\theta \right]
\end{equation}
where the boundary condition $ N(\theta = 0) = N(s_{\theta = 0}, t_{\theta = 0}) $, and the coordinates $ (s(\theta_1), t(\theta_1)) $ are to be determined. We denoted $ s_{\theta = 0} $ to specify it is the origin of the characteristic but still distinguish it from $ s_0 $, the size of newborns. The time coordinate, solution of \eqref{eq::chara_t}, is denoted by $ T $:
\[
	T(\theta) = \theta + t_{\theta = 0}
\]
however, the $ i $-state coordinate $ s $ cannot be expressed in terms of elementary functions. In what follows, we assume $ \beta_2 < 0 $ which is true for all the species we parameterised. Thus, the integral
\begin{align*}
	\int  | G(s, \s_c) | \, ds &\leqslant \int e^{|\beta_0|} e^{\beta_1 s + \beta_2 s^2} \, ds \\
	&\leqslant e^{|\beta_0| - \frac{\beta_1^2}{4 \beta_2^2}} \int e^{\beta_2(s + \sfrac{\beta_1}{2\beta_2})^2} \, ds
\end{align*}
exists and equation \eqref{eq::chara_s} has a solution that we denote by $ S(\theta_1, s_{\theta = 0}, \s_c) $. This solution $ S(\theta_1, s_{\theta = 0}, \s_c) $ is the coordinate of the $ i $-state at $ \theta_1 $ along the characteristic originated in $ s_{\theta = 0} $ at $ t_{\theta = 0} $.

Let us assume that $ S $ admits an inverse, in other words, there exists a function $ \tau $ such that:
\begin{align*}
	\tau \big( S(\theta_1, s_{\theta = 0}, \s_c), s_{\theta = 0}, \s_c \big) &= \theta_1 \\
	S \big(\tau(s_1, s_{\theta = 0}, \s_c), s_{\theta = 0}, \s_c \big) &= s_1
\end{align*}
Then, by definition, $ \tau(s_1, s_0, \s_c) $ is the time it requires to grow from $ s_0 $ to $ s_1 $ under competition $ \s_c $. Although, equation 1 (in the paper) cannot be solved analytically (we would need an explicit solution of $ S $ and $ \tau $), the net reproduction rate is still tractable. Indeed, equation \eqref{eq::sol_ODE_N} can be rewritten:
\begin{align*}
	N \big( s(\theta_1), t(\theta_1) \big) &= N( s_{\theta = 0}, t_{\theta = 0}) \exp \left[-\int_0^{\theta_1} \mu(s, \s_c) + \frac{\partial G}{\partial s}(s, \s_c) \, d\theta \right] \\
	&= N( s_{\theta = 0}, t_{\theta = 0}) \exp \left[-\int_{s_{\theta = 0}}^{s(\theta_1)} \left( \mu(s, \s_c) + \frac{\partial G}{\partial s}(s, \s_c) \right) \frac{d \theta}{ds} \, ds \right] \\
	&= N( s_{\theta = 0}, t_{\theta = 0}) \exp \left[-\int_{s_{\theta = 0}}^{s(\theta_1)} \left( \mu(s, \s_c) + \frac{\partial G}{\partial s}(s, \s_c) \right) \frac{1}{G(s, \s_c)} \, ds \right] \\
	&= N( s_{\theta = 0}, t_{\theta = 0}) \frac{G(s_{\theta = 0}, \s_c)}{G \big( s(\theta_1), \s_c \big)} \exp \left[-\int_{s_{\theta = 0}}^{s(\theta_1)} \frac{\mu(s, \s_c)}{G(s, \s_c)} \, ds \right] \\
\end{align*}

If a characteristic emerged at $ t_{\theta = 0} \leqslant 0 $, then it concerns the initial population (at $ t = 0 $) denoted by $ \phi(s) $:
\begin{equation}\label{eq::sol_init}
	N(S(t, s, \s_c), t) = \phi(s) \frac{G(s, \s_c)}{G\big( S(t, s, \s_c), \s_c \big)} \exp \left[ -\int_{s}^{S(t, s, \s_c)} \frac{\mu(\sigma, \s_c)}{G(\sigma, \s_c)} \, d\sigma \right]
\end{equation}
where $ S(t, s, \s_c) $ is the size of the individual at time $ t $ given at time $ 0 $ they were of size $ s $. In figure \ref{fig::chara}, the characteristics that emerged before $ t = 0 $ are the first, second and third curves.

For characteristics that emerged after time 0 (fourth and fifth curves on fig \ref{fig::chara}), we denote their birth coordinates $ (s_{\theta = 0}, t_{\theta = 0}) $ by $ (0, t_b) $; $ t_b > 0 $ stands for the time of birth.
\begin{equation}\label{eq::sol_later}
	N(s, t) = N \big( 0, t - \tau(s, 0, \s_c) \big) \frac{G(0, \s_c)}{G(s, \s_c)} \exp \left[ -\int_{0}^{s} \frac{\mu(\sigma, \s_c)}{G(\sigma, \s_c)} \, d\sigma \right]
\end{equation}

The initial population (equation \eqref{eq::sol_init}) goes extinct as $ t $ goes by. Therefore, only equation \eqref{eq::sol_later} is of interest \citep{DeRoos1997}. We now use the boundary condition equation 2 (in the paper, once again, we drop the spatial variable $ x $ since the dispersal kernel $ \K $ is a Dirac):
\begin{align*}
	N(0, t) G(0, \s_c) &= \int_{0}^{\infty} F(s, \s_c) N(s, t) \, ds \\
		&= \int_{0}^{\infty} F(s, \s_c) N \big( 0, t - \tau(s, 0, \s_c) \big) \frac{G(0, \s_c)}{G(s, \s_c)} \exp \left[ -\int_{0}^{s} \frac{\mu(\sigma, \s_c)}{G(\sigma, \s_c)} \, d\sigma \right] \, ds
\end{align*}
Let
\[
	B(t, \s_c) = N(0, t) G(0, \s_c)
\]
the population birth rate (\ie the quantity of seedlings created at time $ t $ by a population undergoing a competition $ \s_c $). We can substitute $ B $ into the previous equation:
\begin{equation} \label{eq::popBirth}
	B(t, \s_c) = \int_{0}^{\infty} B \big(t - \tau(s, 0, \s_c), \s_c \big) \frac{F(s, \s_c)}{G(s, \s_c)} \exp \left[ -\int_{0}^{s} \frac{\mu(\sigma, \s_c)}{G(\sigma, \s_c)} \, d\sigma \right] \, ds
\end{equation}
If $ \tau $ were a known function (which is not the case here), equation \eqref{eq::popBirth} would relate the population birth rate to its past. The quantity
\[
	\frac{1}{G(s, \s_c)} B \big(t - \tau(s, 0, \s_c), \s_c \big) \exp \left[ -\int_{0}^{s} \frac{\mu(\sigma, \s_c)}{G(\sigma, \s_c)} \, d\sigma \right]
\]
is just the density of individuals born $ t - \tau $ times unit ago and that survived up to a size in $ [s, s + ds] $ at time $ t $, and $ B $ is the inflow per unit time.

Equation \eqref{eq::popBirth} admits an equilibrium only for particular combinations of parameters $ G $, $ \mu $ and $ F $ and since there is no density dependence here ($ \s $ is set to a known value $ \s_c $), then $ B $ ultimately grows or decline exponentially \citep{DeRoos1997}. We therefore substitute the trial solution
\begin{equation} \label{eq::B_trial}
	B(t, \s_c) = B_0e^{\lambda t}
\end{equation}
into \eqref{eq::popBirth} to get:
\begin{align*}
	B_0 e^{\lambda t} &= \int_{0}^{\infty} B_0 e^{\lambda t - \lambda \tau(s, 0, \s_c)} \frac{F(s, \s_c)}{G(s, \s_c)} \exp \left[ -\int_{0}^{s} \frac{\mu(\sigma, \s_c)}{G(\sigma, \s_c)} \, d\sigma \right] \, ds \\
	&= B_0 e^{\lambda t} \int_{0}^{\infty} e^{- \lambda \tau(s, 0, \s_c)} \frac{F(s, \s_c)}{G(s, \s_c)} \exp \left[ -\int_{0}^{s} \frac{\mu(\sigma, \s_c)}{G(\sigma, \s_c)} \, d\sigma \right] \, ds \\
\end{align*}
Thus, we get:
\begin{equation} \label{eq::eigen}
	1 = \int_{0}^{\infty} e^{- \lambda \tau(s, 0, \s_c)} \frac{F(s, \s_c)}{G(s, \s_c)} \exp \left[ -\int_{0}^{s} \frac{\mu(\sigma, \s_c)}{G(\sigma, \s_c)} \, d\sigma \right] \, ds
\end{equation}
When $ \lambda $ is set to 0, the right hand side of \eqref{eq::eigen} becomes:
\begin{equation} \label{eq::R0_app}
	\int_{0}^{\infty} \frac{F(s, \s_c)}{G(s, \s_c)} \exp \left[ -\int_{0}^{s} \frac{\mu(\sigma, \s_c)}{G(\sigma, \s_c)} \, d\sigma \right] \, ds
\end{equation}
It represents the expected number of seedlings produced by an individual through its lifespan, which is by definition the net reproduction rate $ R_0(\s_c) $ we were looking for (equation 4 in the paper). Given only canopy individuals can reproduce, $ \forall s < \s_c $, $ F(s, \s_c) = 0 $ and $ \forall s \geqslant \s_c $, $ F(s, \s_c) > 0 $, the lower limit of the integrals can be changed to $ \s_c $ in equation \eqref{eq::R0_app}:
\[
	R_0 = \exp \left[- \int_{0}^{\s_c} \frac{\mu(\sigma, \s_c)}{G(\sigma, \s_c)} \, d\sigma \right] \times \int_{\s_c}^{\infty} \frac{F(s, \s_c)}{G(s, \s_c)} \exp \left[ - \int_{\s_c}^{s} \frac{\mu(\sigma, \s_c)}{G(\sigma, \s_c)} \, d\sigma \right] \, ds
\]
\begin{rem}[On $ \lambda $]
	If $ \lambda = 0 $ is truly solution of \eqref{eq::eigen}, then $ R_0(\s_c) = 1 $ and the population is stable. This is consistent with \eqref{eq::B_trial} since in this case the population birth rate would be constant.
\end{rem}

The constant $ \lambda $ is called the intrinsic growth rate of the population. Let us now prove that $ \lambda > 0 $ is equivalent to a net reproduction rate $ R_0(\s_c) > 1 $. Denote by $ f $ the function:
\[
	f(\lambda) = \int_{0}^{\infty} e^{- \lambda \tau(s, 0, \s_c)} \frac{F(s, \s_c)}{G(s, \s_c)} \exp \left[ -\int_{0}^{s} \frac{\mu(\sigma, \s_c)}{G(\sigma, \s_c)} \, d\sigma \right] \, ds
\]
We have:
\begin{align*}
	f'(\lambda) &= -\lambda f(\lambda) \\
	f''(\lambda) &= +\lambda^2 f(\lambda)
\end{align*}
Given $ f $ is a positive function (fecundity and growth are positive functions), then $ f $ is a convex decreasing function on $ \R^{+} $ and its maximum is reached at $ \lambda = 0 $. Hence, if $ R_0 > 1 $:
\[
	f(\lambda) = 1
\]
admits a unique solution on $ \R^{+} $. Thus
\[
	R_0 > 1 \Leftrightarrow \lambda > 0
\]

\subsection{Proofs of the three assertions \label{app::calc_R0::sec::3asser}}
We now prove the three following assertions:
\begin{enumerate}[label=(\textit{\roman*})]
	\item $ R_0(\s_c) $ is a decreasing function
	\item $ R_0 $ is an increasing function of the average understorey growth $ \bar{G} $ and a decreasing function of the average mortality $ \bar{\mu} $
	\item $ R_0 $ is an increasing function of the average fecundity $ \bar{F} $
\end{enumerate}

\subsubsection{First assertion}
The easiest argument to prove $ R_0 $ is a decreasing function of $ \s $ is to see that all the integrand's parts are always positive. Let:
\[
	f(s_1) = \int_{s_1}^{\infty} \frac{F}{G} e^{-\int_{s_1}^s \frac{\mu}{G} \, d\sigma} \, ds
\]
For any $ s_1 \leqslant s_2 $ we have:
\begin{align*}
	R_0(s_1) - R_0(s_2) &= e^{-\int_{0}^{s_1} \frac{\mu}{G} \, ds} \left[ \left( 1 - e^{-\int_{s_1}^{s_2} \frac{\mu}{G} \, ds} \right) f(s_2) + \int_{s_1}^{s_2} \frac{\mu}{G} \, ds \right] \\
		&\geqslant 0
\end{align*}
Therefore, $ R_0 $ is a decreasing function of $ \s $.

\subsubsection{Second assertion}
The average understorey growth for a canopy height $ \s $ is defined by
\[
	\bar{G} = \frac{1}{\s} \int_0^{\s} G(s, \s) \, ds
\]
Let $ G_1 $ and $ G_2 $ two growth functions. Since the overstorey growth is not changed, we just study the ratio:
\begin{equation} \label{eq::ratio_G}
	\frac{e^{\int_0^{\s} \frac{\mu}{G_1} \, ds}}{\int_0^{\s}e^{\frac{\mu}{G_2} \, ds}}
\end{equation}
We want to find a sufficient condition on $ G_1 $ and $ G_2 $ to get this ratio larger than 1, which means the net reproduction rate $ R_0 $ increases. Equation \eqref{eq::ratio_G} can be be rewritten:
\[
	e^{\int_0^{\s} \frac{\mu (G_1 - G_2)}{G_1 G_2} \, ds} \geqslant 1 \quad \Leftrightarrow \quad \int_0^{\s} \frac{\mu (G_1 - G_2)}{G_1 G_2} \, ds \geqslant 0
\]
Given $ \mu $ and $ G $ are positive functions, we can lower bound the integrand:
\begin{equation} \label{eq::minor}
	\frac{\mu (G_1 - G_2)}{G_1 G_2} \geqslant \frac{\max(\mu)}{\min(G_1 G_2)} \mathds{1}_{G_1 < G_2} (s) \times (G_1 - G_2) + \frac{\min(\mu)}{\max(G_1 G_2)} \mathds{1}_{G_1 \geqslant G_2}(s) \times (G_1 - G_2)
\end{equation}
Define
\[
	\begin{matrix}
		k_1 = \frac{\max(\mu)}{\min(G_1 G_2)} & &
			\mathscr{D}_1 = \{ s | G_1(s) < G_2(s) \} \\
		k_2 = \frac{\min(\mu)}{\max(G_1 G_2)} & &
				\mathscr{D}_2 = \{ s | G_1(s) \geqslant G_2(s) \}\\
	\end{matrix}
\]
equation \eqref{eq::minor} is:
\begin{align*}
	\int_0^{\s} \frac{\mu (G_1 - G_2)}{G_1 G_2} &\geqslant
		k_1 \int_{\mathscr{D}_1} G_1  \, ds + k_2 \int_{\mathscr{D}_2} G_1 \, ds +
		k_1 \int_{\mathscr{D}_1} G_2  \, ds + k_2 \int_{\mathscr{D}_2} G_2 \, ds \\
	&\geqslant \min (k_1, k_2) \left[\int_{\mathscr{D}_1 \cup \mathscr{D}_2} G_1 \, ds + \int_{\mathscr{D}_1 \cup \mathscr{D}_2} G_2 \, ds \right]
\end{align*}
Therefore, given $ \mathscr{D}_1 \cup \mathscr{D}_2 = [0, \s] $, a sufficient condition to have the ratio defined by equation \eqref{eq::ratio_G} larger than $ 1 $ is:
\begin{equation}
	\int_0^{\s} G_1 \, ds \geqslant \int_0^{\s} G_2 \, ds,
\end{equation}
that is to say:
\[
	\bar{G}_1 \geqslant \bar{G}_2
\]
Thus, when the understorey growth increases in average, so does the net reproduction rate $ R_0 $. A similar reasoning on $ \mu $ would prove that increasing the understorey mortality induce a depleted $ R_0 $. This conclude the proof of the second assertion.

\subsubsection{Third assertion}
In this article, we used the fecundity of \citet{Purves2008} which is proportional to sun-exposed crown area $ \A $:
\[
	F(s, \s) = f \A(s, \s)
\]
where $ f $ is the new recruits per unit time per crown area.
\begin{align*}
	\frac{\partial R_0}{\partial f} &= e^{-\int_0^{\s} \frac{\mu}{G} \, ds} \int_{\s}^{\infty} \frac{A}{G} e^{-\int_{\s}^{s} \frac{\mu}{G} \, d \sigma} \, ds \\
		&\geqslant 0
\end{align*}
Thus, increasing fecundity augments $ R_0 $, which proves the third assertion.

\subsection{Proofs of the relations between \citet{Purves2009} and our study}
In this section, we first provide the assumptions concerning the demography in \citet{Purves2009}, and match our notations with his article in the table \ref{tab::notations_purves2009}. Finally, we derive Purves's formul\ae{} from our $ R_0 $ equation 4 (in the paper). Except when specified, Purves refers to \citet{Purves2009} in the following sections.
\begin{table}[!h]
	\centering
	\caption{Link between our notations and \citet{Purves2009}'s notations. Note that he uses the flat-top version of the perfect-plasticity approximation, which implies the fecundity function to be proportional to the trunk area.}
	\label{tab::notations_purves2009}
	\begin{tabular}{@{}rll@{}}
	\toprule
	\textbf{Meaning} & \textbf{Our notation} & \textbf{\citet{Purves2009}} \\
	Species index & $ j $, but mostly dropped & $ j $ \\
	Space & $ x $ & $ R $ (not to be confound with $ R_0 $) \\
	Time & $ t $ & $ \tau $ \\
	Threshold diameter & $ \s, \, \s_c $ & $ D^{*} $ \\
	Net reprod. rate & $ R_0 (x, \s_c) $ & $ R_{0, j, R} $ \\
	Indiv. growth rate overstorey & $ G(s, \s, x), \, s \geqslant \s $ & $ G_{L, j, R} $ \\
	Indiv. growth rate understorey & $ G(s, \s, x), \, s < \s $ & $ G_{D, j, R} $ \\
	Mortality rate overstorey & $ \mu(s, \s, x), \, s \geqslant \s $ & $ \mu_{L, j, R} = \sfrac{1}{\rho_{L, j, R}} $ \\
	Mortality rate understorey & $ \mu(s, \s, x), \, s < \s $ & $ \mu_{D, j, R} = \sfrac{1}{\rho_{D, j, R}} $ \\
	Fecundity function & $ \F \A(s, \s) $ & $ F_{j, R}^{\text{capita}} \pi \sfrac{s^2}{4} $ \\
   \bottomrule
	\end{tabular}
\end{table}

\subsubsection{Additional assumptions to our model from \citet{Purves2009}}
We sum-up here some useful assumptions from Purves. The assumptions are better described in \citet[not to confound with 2009's paper]{Purves2008}.
\begin{assum}[Flat-top crown] \label{assum::flat-top}
	The crown is a flat-top disc expressed at the top of the tree, which implies that the area of the crown $ \A(s, \s) $ is independent of its second argument.
\end{assum}

\begin{assum}[Demographic rates]
	The individual growth and mortality rates are step functions with the discontinuity at $ \s $:
	\[
		G_{j, R}(s, \s) =
		\begin{cases}
			G_{L, j, R} & \text{if } s \geqslant \s \\
			G_{D, j, R} & \text{otherwise}
		\end{cases}
	\]
	\[
	\mu_{j, R}(s, \s) =
		\begin{cases}
			\mu_{L, j, R} & \text{if } s \geqslant \s \\
			\mu_{D, j, R} & \text{otherwise}
		\end{cases}
	\]
	where $ G_{L, j, R}, \, G_{D, j, R}, \, \mu_{L, j, R}, \, \mu_{D, j, R} $ are estimated constants.
\end{assum}

\begin{assum}[Fecundity function]
	Due to the flat-top assumption \ref{assum::flat-top}, the fecundity is simply proportional to the cross-section of the trunk area at breast height:
	\[
		F_{j, R} (s, \s) =
		\begin{cases}
			\frac{1}{10^4} F_{j, R}^{\text{capita}} \pi s^2 & \text{if } s \geqslant \s \\
			0 & \text{otherwise}
		\end{cases}
	\]
	The factor $ \frac{1}{10^4} $ corrects for the units of $ F_{j, R}^{\text{capita}} $ (basal area per year, dividing by $ 10^4 $ makes the basal area dimensionless).
\end{assum}

\begin{assum}[Mortality versus individual growth] \label{assum::negligible}
	The mortality rate is negligible compare to both 1 and the individual growth rate:
	\begin{align}
		\mu_{L, j, R} &\ll 1 \\
		\mu_{L, j, R} &\ll G_{L, j, R}
	\end{align}
\end{assum}

Note that in his supporting information, Purves integrates with respect to time rather than size. In the case trees are always in the overstorey, it is strictly equivalent:
\[
	\begin{cases}
		\frac{ds}{d \tau} = G_{L, j, R} \\
		s(0) = D^{*}
	\end{cases}
\]
which has the solution:
\[
	s(\tau) = \tau G_{L, j, R} + D^{*}
\]
This solution is for instance found in equation S11 \citep{Purves2009}.

\subsubsection{$ R_0 $ equation \citep[p. 1479]{Purves2009}}
Equation 4 (in the paper) is equivalent to equation S10.2 \citep[also p. 1479 in his article]{Purves2009} when $ \s_c = 0 $.
\begin{proof}
	Let $ \s_c = 0 $.
	\begin{align*}
		R_0 (x, 0) &= \overbrace{e^{-\int_0^{0}\frac{\mu(s, \s_c, x)}{G(s, \s_c, x)} \, ds}}^{= 1} \int_{0}^{\infty} \frac{1}{10^4} \frac{F_{j, R}^{\text{capita}} \pi (\sfrac{s}{2})^2}{G_{L, j, R}} e^{-\int_{0}^{s} \frac{\mu_{L, j, R}}{G_{L, j, R}} \, d\sigma} \, ds \\
			&= \frac{1}{10^4} \frac{\pi}{4} \frac{F_{j, R}^{\text{capita}}}{G_{L, j, R}} \int_{0}^{\infty} s^2 e^{-\frac{\mu_{L, j, R}}{G_{L, j, R}} s} \, ds \\
			&= \frac{1}{10^4} \frac{\pi}{2} F_{j, R}^{\text{capita}} \frac{G_{L, j, R}^2}{\mu_{L, j, R}^3} \\
			&= \frac{1}{10^4} \frac{\pi}{2} F_{j, R}^{\text{capita}} G_{L, j, R}^2 \rho_{L, j, R}^3
	\end{align*}
\end{proof}

\subsubsection{Proportion of the trees that make up to the canopy}
Let $ P^{\text{canop}} $ be the proportion of individuals that survive up to the canopy. The quantity $ P^{\text{canop}} $ is derived in Purves' supporting information, a little before equation S11; we can obtain the same result with the first part of equation 4 (in the paper).
\begin{proof}
	\begin{align*}
		P^{\text{canop}} &= e^{-\int_{0}^{D^{*}} \frac{\mu}{G} \, ds} \\
			&= e^{-\frac{\mu_{D, j, R}}{G_{D, j, R}} D^{*}} \\
			&= e^{-\frac{D^{*}}{G_{D, j, R}\rho_{D, j, R}}}
	\end{align*}
\end{proof}

\subsubsection{$ D^{*} $ at equilibrium \citep[equation S13.2]{Purves2009}}
Let $ \hat D^{*} $ be the threshold diameter when the (monospecific) population is at equilibrium. We derive equation S13.2 \citep{Purves2009} from our equation 4 (in the paper) using Purves' notations:
\begin{align*}
	R^{\text{canop}} &= e^{-\int_{0}^{D^{*}} \frac{\mu_{D, j, R}}{G_{D, j, R}} \, ds} \int_{D^{*}}^{\infty} \frac{1}{10^4} \frac{F_{j, R}^{\text{capita}} \pi (\sfrac{s}{2})^2}{G_{L, j, R}} e^{-\int_{D^{*}}^{s} \frac{\mu_{L, j, R}}{G_{L, j, R}} \, d\sigma} \, ds \\
		&= e^{\left(\frac{\mu_{L, j, R}}{G_{L, j, R}} - \frac{\mu_{D, j, R}}{G_{D, j, R}} \right) D^{*}} \frac{1}{10^4} \frac{\pi}{4 G_{L, j, R}} F_{j, R}^{\text{capita}} \int_{D^{*}}^{\infty} s^2 e^{-\frac{\mu_{L, j, R}}{G_{L, j, R}} s} \\
		&= \frac{1}{10^4} \frac{\pi}{4} F_{j, R}^{\text{capita}} e^{- \frac{\mu_{D, j, R}}{G_{D, j, R}} D^{*}} \frac{1}{\mu_{L, j, R}^3} (2 G_{L, j, R}^2 + 2 G_{L, j, R} \mu_{L, j, R} D^{*} + \mu_{L, j, R}^2 {D^{*}}^2)
\end{align*}
Using assumption \ref{assum::negligible}, we get a simplified version of $ R^{\text{canop}} $:
\begin{equation} \label{eq::RcanopApprox}
	R^{\text{canop}} \approx \frac{1}{10^4} \frac{\pi}{2} F_{j, R}^{\text{capita}} e^{- \frac{\mu_{D, j, R}}{G_{D, j, R}} D^{*}} \frac{1}{\mu_{D, j, R}^3} G_{L, j, R}^2
\end{equation}
At equilibrium, each individual replace itself once in average. Therefore, $ R^{\text{canop}} = 1 $. Solving equation \eqref{eq::RcanopApprox} at equilibrium for $ \hat D^{*} $, we get:
\[
	\hat D^{*} = G_{D, j, R} \rho_{D, j, R} \ln \left( \frac{1}{10^4} \frac{\pi}{2} F_{j, R}^{\text{capita}} G_{L, j, R}^2 \rho_{L, j, R}^3 \right) \text{, where } \frac{1}{\mu_{L, j, R}^3} = \rho_{L, j, R}^3,
\]
which is the equation S13.2 of Purves.
\end{onehalfspace}

\printbibliography[heading=subbibliography]
\end{refsection}

\end{document}

