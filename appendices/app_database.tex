
\documentclass[letterpaper, 12pt]{article}

%%%%%%%%%%%%%%%%%%%    PACKAGES    %%%%%%%%%%%%%%%%%%%
%% Package version
\listfiles % Then check the .log file

%% Font
\usepackage{fontspec}
\usepackage{marvosym}

%% Languages
\usepackage{polyglossia}
	\setdefaultlanguage[variant=british]{english}

%% Marges, space...
\usepackage[top=2.5cm, bottom=2.5cm, left=2.5cm, right=2.5cm]{geometry}
\usepackage{setspace} % [nodisplayskipstretch] pour option space in equation

\usepackage{indentfirst}

\usepackage[bottom]{footmisc}
\usepackage{footnote}

% https://tex.stackexchange.com/questions/279/how-do-i-ensure-that-figures-appear-in-the-section-theyre-associated-with
\usepackage[section]{placeins} % To place figures in the section it is declared

%% Graphics
\usepackage[luatex]{graphicx}
	\graphicspath{{../graphs/}}

\usepackage{caption}
\usepackage{subcaption}
	\captionsetup{subrefformat=parens}

\makeatletter 
\renewcommand{\thefigure}{S\thesection.\@arabic\c@figure}
\renewcommand{\thetable}{S\thesection.\@arabic\c@table}
\renewcommand{\theequation}{S.\arabic{equation}}
\makeatother

\usepackage[dvipsnames, svgnames]{xcolor}

\usepackage{tikz}
	\usetikzlibrary{arrows, plotmarks, decorations.markings}
	\tikzstyle{arrow} = [->,>=stealth,thick,rounded corners=4pt,line width=1pt]
	\usetikzlibrary{shadows}
	\usetikzlibrary{shadings}
	\usetikzlibrary{positioning} % relative coodinate
	\usetikzlibrary{tikzmark, calc} % calc, to calculate coordinate
	\usetikzlibrary{decorations.pathmorphing} % to snake an arrow
	\usetikzlibrary{shapes.arrows}
	\usetikzlibrary{patterns}
	\tikzset{
		invisible/.style={opacity=0},
		visible on/.style={alt={#1{}{invisible}}},
		alt/.code args={<#1>#2#3}{%
		\alt<#1>{\pgfkeysalso{#2}}{\pgfkeysalso{#3}} % \pgfkeysalso doesn't change the path
		},
	} % end tikzset. Code from http://tex.stackexchange.com/questions/136143/tikz-animated-figure-in-beamer

%% Table of content
\addto\captionsenglish{ % Replace "english" with the language used in babel
	\renewcommand{\contentsname}{Supporting Information Appendix S2 contents}
}

%% Format section
\usepackage{titlesec}
\titlelabel{S\thetitle~}
\usepackage{titletoc}
\titlecontents{section}[0pt]{}{\bfseries S\thecontentslabel~}{\bfseries}{\hspace{1em plus 1fill}\contentspage}


\usepackage{enumitem}% http://ctan.org/pkg/enumitem

%% Links
\usepackage{url}
\usepackage[luatex, colorlinks=true, linkcolor=NavyBlue, urlcolor=MidnightBlue, citecolor=PineGreen]{hyperref}

%% Table
\usepackage{booktabs}

%% bibliography
\usepackage{csquotes}
\usepackage[style=apa, natbib=true, sorting=ynt]{biblatex}
\addbibresource{bib_database.bib}

%% mathematics
\usepackage{amsthm}
\usepackage{amsmath}
	\allowdisplaybreaks % Autoriser découpe formules entres pages
\usepackage{amssymb}
\usepackage{bbold}
\usepackage{dsfont}
\usepackage{mathrsfs}
\usepackage{bm}
\usepackage{xfrac}
\usepackage{etoolbox} % For renumbering (cf below, counter for model)

\usepackage{thmbox} % cf after for "theorem" definitions

\usepackage{gensymb}

\usepackage{siunitx}

%%%%%%%%%%%%%%%%%   NEW COMMANDES    %%%%%%%%%%%%%%%%%
%% Text
\newcommand {\ie}{\textit{i.e., }}
\newcommand {\eg}{\textit{e.g., }}
\newcommand {\cf}{\textit{cf} }
\newcommand\bsc[1]{\textsc{\MakeLowercase{#1}}} % Only if there is no french babel
\newcommand {\thup}[1]{{#1}\textsuperscript{th}}

%% Math
\newcommand {\s}{{s}^{*}}
\newcommand {\sst}{\tilde{s}^{*}} % s* stable d'où le st
\newcommand {\N}{\tilde{N}}
\newcommand {\A}{\mathscr{A}}
\newcommand {\K}{\mathcal{K}}
\renewcommand{\S}{\mathscr{S}}
\newcommand{\R}{\mathds{R}}
\newcommand{\Prob}{\mathds{P}}
\newcommand{\F}{\mathcal{F}}

\DeclareMathOperator{\logit}{logit}

%%%%%%%%%%%%%%%%%   THEOREM STYLE    %%%%%%%%%%%%%%%%%
\newtheoremstyle{theo}{\topsep}{\topsep}{\itshape}{}{\bfseries}{.}{\newline}{\thmname{#1} \thmnumber{#2} \thmnote{~: \textit{#3}}}
\theoremstyle{theo}
\newtheorem{rem}{Remark}[section]
\newtheorem{defi}{Definition}[section]
\newtheorem{assum}{Assumption}[section]
\newtheorem{nota}{Notation}[section]

%%%%%%%%%%%%%%%%%   SET COUNTER		%%%%%%%%%%%%%%%%%
\setcounter{section}{1}

\begin{document}
\tableofcontents

\begin{refsection}
\begin{onehalfspace}

\section{Database} \label{app::database}
\subsection{Description of the data}
We used an abbreviation for each species, that we list in Tab. \ref{tab::species}. For each species, we list their spatial range in Tab. \ref{tab::spaceRange}, and we plot the ranges of temperature and precipitation that are used in the individual growth and mortality functions (Fig. \ref{fig::speciesClimRange}).  We also join the 19 bioclimatic variables used in the random forest and the demography models selection (Tab. \ref{tab::bioclim}). Finally, we mapped all the data we used, from 1963 to 2010 (Fig \ref{fig::mapDatabase}).

\begin{table}[h!]
\centering
\caption{Species list, and the abbreviation used in the other tables\label{tab::species}. The last three columns are the maximal dbh, the maximal age, and the page of \citet{Burns1990, Burns1990a} at which we found the informations.}
\label{tab::database}
\begin{tabular}{rlllll}
	\toprule
	\textbf{Species} & \textbf{Scientific name} & \textbf{Vernacular name} & $ \text{\textbf{dbh}}_{\bm{\max}} $ & $ \text{\textbf{age}}_{\bm{\max}} $ & \textbf{page} \\
	\midrule
	ABI-BAL & Abies balsamea & Balsam fir & 460 & 200 & 33 \\
	ACE-RUB & Acer rubrum & Red maple & 760 & 125 & 170 \\
	ACE-SAC & Acer saccharum & Sugar maple & 910 & 350 & 202 \\
	BET-ALL & Betula alleghaniensis & Yellow birch & 760 & 300 & 302 \\
	BET-PAP & Betula papyrifera & White birch & 300 & 150 & 348 \\
	FAG-GRA & Fagus grandifolia & American beech & 510 & 366 & 660 \\
	PIC-GLA & Picea glauca & White spruce & 900 & 275 & 410 \\
	PIC-MAR & picea mariana & Black spruce & 230 & 250 & 451 \\
	PIC-RUB & picea rubens & Red spruce & 610 & 400 & 499 \\
	PIN-BAN & Pinus banksiana & Jack pine & 250 &  80 & 566 \\
	PIN-STR & Pinus strobus & Eastern white pine & 1020 & 200 & 982 \\
	POP-TRE & Populus tremuloides & Quaking aspen & 300 & 200 & 1093 \\
	THU-OCC & Thuja occidentalis & Northern white cedar & 600 & 400 & 1197 \\
	TSU-CAN & Tsuga canadensis & Eastern hemlock & 1020 & 400 & 1246 \\
	\bottomrule
\end{tabular}
\end{table}

\begin{table}[ht]
\centering
\caption{Number of individual measurements in the database, and geographical extent for each species (before cropping, see Fig. \ref{fig::mapDatabase} for the crop extent). There are \num{3816854} individual measurements in total. We used the projection system WGS 84 (EPSG:4326) for this table.}
\label{tab::spaceRange}
\begin{tabular}{lrrrrr}
	\toprule
	~ & ~ & \multicolumn{2}{c}{\textbf{Longitude}} & \multicolumn{2}{c}{\textbf{latitude}} \\
	\cmidrule(lr){3-4} \cmidrule(lr){5-6}
	\bfseries{Species} & \bfseries{$ \# $ measures} & \bfseries{min} & \bfseries{max} & \bfseries{min} & \bfseries{max} \\
	\midrule
		ABI-BAL & \num{822265} & -96.07 & -57.31 & 39.04 & 52.90 \\
		ACE-RUB & \num{469195} & -96.47 & -63.86 & 25.91 & 49.49 \\
		ACE-SAC & \num{321776} & -97.65 & -63.96 & 30.66 & 49.44 \\
		BET-ALL & \num{104249} & -95.52 & -63.87 & 31.48 & 49.38 \\
		BET-PAP & \num{291701} & -154.01 & -57.99 & 37.92 & 61.42 \\
		FAG-GRA & \num{96990} & -95.04 & -64.59 & 30.36 & 51.29 \\
		PIC-GLA & \num{104342} & -151.78 & -58.19 & 37.70 & 61.44 \\
		PIC-MAR & \num{656273} & -151.40 & -57.31 & 41.70 & 61.13 \\
		PIC-RUB & \num{126063} & -84.35 & -62.61 & 35.31 & 51.43 \\
		PIN-BAN & \num{204463} & -99.09 & -64.49 & 38.95 & 52.57 \\
		PIN-STR & \num{90002} & -96.94 & -61.95 & 31.49 & 51.50 \\
		POP-TRE & \num{310514} & -151.15 & -59.58 & 32.43 & 61.42 \\
		THU-OCC & \num{153013} & -95.79 & -64.22 & 35.20 & 50.82 \\
		TSU-CAN & \num{66008} & -91.93 & -64.70 & 31.60 & 49.57 \\
	\bottomrule
\end{tabular}
\end{table}

\begin{figure}[htb]
    \centering
	%% First row
	\begin{subfigure}{0.98\textwidth}
		\input{../graphs/annual_mean_temperature-min_temperature_of_coldest_month_sp_range}
		\caption{Annual mean temperature $ T_a $ (left), and minimal temperature $ T_m $(right) range for each species. The vertical line is the average among species, and the dark circles are the species-specific mean.}
		\label{fig::annual_mean_temperature}
	\end{subfigure}
	\medskip
	%% Second row
	\begin{subfigure}{0.98\textwidth}
		\input{../graphs/annual_precipitation-precipitation_of_driest_quarter_sp_range}
		\caption{Annual precipitation $ P_a $ (left), and minimal precipitation $ P_m $(right) range for each species. The vertical line is the average among species, and the dark circles are the species-specific mean.}
		\label{fig::annual_precipitation}
	\end{subfigure}
\caption{Species-specific \subref{fig::annual_mean_temperature} temperature ranges and \subref{fig::annual_precipitation} precipitation ranges. The definitions of $ T_a, \, T_m, \, P_a, \, P_m $ are in table \ref{tab::bioclim}.}
\label{fig::speciesClimRange}
\end{figure}

\begin{table}[h!]
\centering
\caption{Bioclimatic variables list \citep{McKenney2011}. \label{tab::bioclim}}
\begin{tabular}{p{6.75cm}p{9cm}}
	\toprule
	\textbf{Name} & \textbf{Description} \\
	\midrule
	annual mean temperature $ T_a $ & avg of mean monthly temperatures \\
	mean diurnal range $ r_T $ & avg of monthly temperature ranges \\
	isothermality $ I $ & variable 2 divided by variable 7 \\
	temperature seasonality $ T_s $ & standard deviation of monthly-mean temperature estimates expressed as a percent of their mean \\
	max temperature of warmest month $ T_M $ & highest monthly maximum temperature \\
	min temperature of coldest month $ T_m $ & lowest monthly minimum temperature \\
	temperature annual range $ T_r $ & Variable 5 minus variable 6 \\
	mean temperature of wettest quarter $ T_w $ & avg temperature during 3 wettest months \\
	mean temperature of driest quarter $ T_d $ & avg temperature during 3 driest months \\
	mean temperature of warmest quarter $ T_h $ & avg temperature during 3 warmest months \\
	mean temperature of coldest quarter $ T_c $ & avg temperature during 3 coldest months \\
	annual precipitation $ P_a $ & sum of monthly precipitation values \\
	precipitation of wettest month $ P_M $ & precipitation of the wettest month \\
	precipitation of driest month $ P_m $ & precipitation of the driest month \\
	precipitation seasonality $ P_s $ & standard deviation of the monthly precipitation estimates expressed as a percent of their mean \\
	precipitation of wettest quarter$ P_w $  & total precipitation of 3 wettest months \\
	precipitation of driest quarter $ P_d $ & total precipitation of 3 driest months \\
	precipitation of warmest quarter $ P_h $ & total precipitation of 3 warmest months \\
	precipitation of coldest quarter $ P_c $ & total precipitation of 3 coldest months \\
	\bottomrule
\end{tabular}
\end{table}

\begin{figure}[htb]
    \centering
	\includegraphics[scale=0.5]{../graphs/mapDatabase}
	\caption{Map of the data we used, ranging from 1963 to 2010. Each point represents a location where there is at least one record. The orange rectangle is the bounding box of the data. We used it to crop all the maps concerning $ R_0 $. For aesthetic reasons and familiarity, the projection for this map is EPSG:4269. For all the maps in Supporting Information S5, the projection is EPSG:2163}
\label{fig::mapDatabase}
\end{figure}

\clearpage
\subsection{Variability in radial growth and mortality}
For each species, we separated the latitudes within three regions (Fig. \ref{fig::defineRegion}). The radial growth is described by three box plots per species with the following quantiles: $ 2.5 \% $, $ 25 \% $, the median ($ 50 \% $), $ 75 \% $, and $ 97.5 \% $ (Figs. \ref{fig::growthVar1-3}--\ref{fig::growthVar13-14}). For the mortality, we computed species-specific proportions of death events within each region (Figs. \ref{fig::mortalityVar1-3}--\ref{fig::mortalityVar13-14}).

\begin{figure}
	\centering
	\input{selectRegion}
	\caption{We defined the three regions for the Figs. \ref{fig::growthVar1-3}--\ref{fig::mortalityVar13-14} as follow: for species-specific data, (\textit{i}) compute the centroid of the data, (\textit{ii}) find the middle point between the maximum latitude of the data and the centroid to define the northern region, and (\textit{iii}) same with the minimum latitude of the data to define the southern region. Everything else is the middle region. \label{fig::defineRegion}}
\end{figure}

\begin{figure}
	\centering
	\input{growthVar1-3}
	\caption{Variance of the growth data. The three regions were determined following the method describe by the figure \ref{fig::defineRegion}. The box plots represent the $ 2.5 \%$ quantile, the $ 25 \% $, the median ($ 50 \% $), the $ 75 \% $ quantile, and the $ 97.5 \% $ quantile. The dots are outliers. \label{fig::growthVar1-3}}
\end{figure}

\begin{figure}
	\centering
	\input{growthVar4-6}
	\caption{Variance of the growth data. The three regions were determined following the method describe by the figure \ref{fig::defineRegion}. The box plots represent the $ 2.5 \%$ quantile, the $ 25 \% $, the median ($ 50 \% $), the $ 75 \% $ quantile, and the $ 97.5 \% $ quantile. The dots are outliers. \label{fig::growthVar4-6}}
\end{figure}

\begin{figure}
	\centering
	\input{growthVar7-9}
	\caption{Variance of the growth data. The three regions were determined following the method describe by the figure \ref{fig::defineRegion}. The box plots represent the $ 2.5 \%$ quantile, the $ 25 \% $, the median ($ 50 \% $), the $ 75 \% $ quantile, and the $ 97.5 \% $ quantile. The dots are outliers. \label{fig::growthVar7-9}}
\end{figure}

\begin{figure}
	\centering
	\input{growthVar10-12}
	\caption{Variance of the growth data. The three regions were determined following the method describe by the figure \ref{fig::defineRegion}. The box plots represent the $ 2.5 \%$ quantile, the $ 25 \% $, the median ($ 50 \% $), the $ 75 \% $ quantile, and the $ 97.5 \% $ quantile. The dots are outliers. \label{fig::growthVar10-12}}
\end{figure}

\begin{figure}
	\centering
	\input{growthVar13-14}
	\caption{Variance of the growth data. The three regions were determined following the method describe by the figure \ref{fig::defineRegion}. The box plots represent the $ 2.5 \%$ quantile, the $ 25 \% $, the median ($ 50 \% $), the $ 75 \% $ quantile, and the $ 97.5 \% $ quantile. The dots are outliers. \label{fig::growthVar13-14}}
\end{figure}

\begin{figure}
	\centering
	\input{mortalityVar1-3}
	\caption{Variance of the mortality data. The three regions were determined following the method describe by the figure \ref{fig::defineRegion}. \label{fig::mortalityVar1-3}}
\end{figure}

\begin{figure}
	\centering
	\input{mortalityVar4-6}
	\caption{Variance of the mortality data. The three regions were determined following the method describe by the figure \ref{fig::defineRegion}. \label{fig::mortalityVa4-6}}
\end{figure}

\begin{figure}
	\centering
	\input{mortalityVar7-9}
	\caption{Variance of the mortality data. The three regions were determined following the method describe by the figure \ref{fig::defineRegion}. \label{fig::mortalityVa7-9}}
\end{figure}

\begin{figure}
	\centering
	\input{mortalityVar10-12}
	\caption{Variance of the mortality data. The three regions were determined following the method describe by the figure \ref{fig::defineRegion}. \label{fig::mortalityVar10-12}}
\end{figure}

\begin{figure}
	\centering
	\input{mortalityVar13-14}
	\caption{Variance of the mortality data. The three regions were determined following the method describe by the figure \ref{fig::defineRegion}. \label{fig::mortalityVar13-14}}
\end{figure}

\end{onehalfspace}

\clearpage
\printbibliography[heading=subbibliography]
\end{refsection}

\end{document}