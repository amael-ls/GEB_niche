
% [ALS] Deleted references: Monsi2004, Pacala1994, Breiman2001, \citep[p. 121-122]{Zuur2009}, \citep{Vehtari2019}

\section{The Model}
\subsection{Model structure}
We model forest dynamics using a physiologically structured population model (PSPM), based on \citet{Strigul2008}, which we have made spatially explicit. We first provide general definitions that are related to PSPMs and then describe our own. A physiologically structured model distinguishes individuals who are at different stages of development. PSPMs are based upon individual states (hereafter, \textit{i}-states), namely, a collection of variables that exhibit two properties \citep[for an overview of PSPMs]{DeRoos1997}:
\begin{enumerate}[label=(\roman*)]
	\item \textit{i}-states completely determine the individual's growth rate, death rate and birth rate  at any given time (possibly together with the present environmental state), and its influence on the environment,
	\item \textit{i}-states future values are completely determined by their present values, together with the intervening environmental history as encountered by the individual of concern.
\end{enumerate}
The environment is accounted for by the environmental state (hereafter, \textit{e}-state). Formally, an \textit{e}-state is a collection of biotic and abiotic factors that characterise the environment in which an individual lives and that affect individual performance. In this paper, we consider two kinds of \textit{e}-states: (\textit{i}) feedback loops, that both influence and are influenced by individuals of all species; and (\textit{ii}) external forcing factors, which are imposed on the population. By definition, the former requires a dynamic descriptions. We ignore random variation among individuals of the same \textit{i}-state and which experience the same \textit{e}-state. As a result, the model represents cohorts of trees rather than individuals.

Cohort dynamics of species $ j $ and diameter $ s $ and that are located in $ x $ are modelled by a spatially-explicit version of the von Foerster--perfect plasticity approximation model \citep[hereafter, von Foerster--PPA]{Strigul2008}:
\begin{align}
	\frac{\partial N_j(s, x, t)}{\partial t} &= - \frac{\partial G_j \big(s, x; \s(x, t) \big) N_j(s, x, t)}{\partial s} - \mu_j\big(s, x; \s(x, t) \big) N_j(s, x, t) \label{eq::dynamics_x} \\
	N_j(0, x, t) &= \frac{1}{G_j(0, x; \s(x, t))} \int_0^{\infty} N_j(s, x, t) F_j\big(s; \s(x, t) \big) \, ds \label{eq::recruitment_x}
\end{align}
where $ N_j $ is the number of trees of species $ j $ per unit size per unit space, it as a density and only $ \int \int N \, ds dx $ can be considered as a number of individuals. $ G $ is the growth rate of individuals, $ \mu $ is the mortality rate, and $ F $ is the effective fecundity function (see Table \ref{tab::notations} for list of notations, definitions and units for each variable and parameter). Although we developed the model with a dispersal kernel, we decided to use a $ \delta $-Dirac distribution in this paper to maintain model tractability. Therefore, the dispersion is localised to patch $ x $, and does not appear in Equation \eqref{eq::recruitment_x}. The three demographic rates are affected by a size threshold $ \s $, which is a feedback loop that is defined formally below. External factors (at a location $ x $) only influence $ G $ and $ \mu $. Equation \eqref{eq::dynamics_x} describes cohort demography, while equation \eqref{eq::recruitment_x} describes recruitment. Together, \eqref{eq::dynamics_x}--\eqref{eq::recruitment_x} represent the structured-population dynamics of trees. We use tree diameter at breast height (\textit{dbh}) as a single \textit{i}-state and then employ allometric functions from \citet{Purves2007} to compute tree height and crown diameter.

% Work on the paragraph
Competition for light is the major driver of forest dynamics in north eastern North American forests, especially for saplings \citep{Pacala1996, Purves2007}. Nitrogen should have little influence in closed-canopy forests or in understorey conditions, and would be difficult to simulate \citep{Kobe2006}. We thus limit the feedback loop solely to light availability. Competition for water has a non-negligible effect on such forest dynamics at all light levels \citep{Kobe2006}. We also know that temperature impacts metabolic rates \citep{Brown2004}, therefore, we base the external factors on temperature and precipitation variables.

Competition for light is represented by a critical height that partitions the forest into the understorey and the overstorey (Fig. \ref{fig::ppa}). The vertical position of trees, thereafter, determines growth and mortality. The equation describing the feedback loop between cohort dynamics and light availability is:
\begin{equation}
	1 = \sum_{j = 1}^{n} \int_{\s(x)}^{\infty} N_j(s, x, t) \A_j \big(s; \s(x, t)\big) \, ds \label{eq::ppa_x}
\end{equation}
where $ \A $ denotes the cross-sectional area of the crown of an individual of size $ s $ (Fig \ref{fig::ppa}). This last equation \eqref{eq::ppa_x} defines a size threshold $ \s $, which differentiates the behaviour of individuals that are above $ \s $ from those that are below. Hence, $ \s $ is such that the sum of the area of individual crowns equals the area of the plot being considered. Trees with a height below $ \s $ are fully shaded whereas trees above $ \s $ are in the overstorey and receive direct sunlight. When the canopy is open, a positive value of $ \s $ cannot be attained and, therefore, $ \s $ is set to $ 0 $. It should be noted that when the size variable is the height of trees, then $ \s $ is independent of the species. However, when the dbh is the size variable, $ \s $ then becomes species-specific (due to the species-specificity of the allometric functions relating height to dbh). Equation \eqref{eq::ppa_x} is adapted from \citet{Strigul2008} and, therefore, obeys the \textit{perfect-plasticity approximation} assumption (PPA): the canopy is a collection of small crowns that can be reorganised such that the occupied area is maximised. Despite this optimisation of sun exposure is considered theoretical, the combination of PPA with the von-Foerster equations can accurately reproduce forest dynamics at stand scales, within a relatively homogeneous physical environment \citep{Strigul2008, Purves2008}. This set of three equations (\ref{eq::dynamics_x}--\ref{eq::ppa_x}) is a tractable PSPM, which allows us to derive analytically $ R_0 $.

\begin{figure}
	\centering
	\input{graphs/ppaStrigul}
	\caption{Traditionally, it is assumed that light availability decreases progressively from the canopy to the forest floor, obeying a Beer's law (left arrow). Here we assume there is a threshold $ \s $ that splits the forest into two parts, and defines therefore two light levels (right arrow): $ L_u $ and $ L_o $ for understorey and overstorey light respectively. Hence the light is a stepwise function of canopy height. The threshold $ \s $ is the maximum height at which the canopy is closed; at this particular height, the sum of all the cross section equals the plot area. $ \A(s, \sigma) $ is the area of the cross section of the crown at height $ \sigma $ of an individual of height $ s $. \label{fig::ppa}}
\end{figure}

\begin{table}[h]
	\centering
	\caption{Notations used in this paper (sorted in alphabetical order---using equivalents for the Greek letters). $ T $ stands for time unit, $ \ell $ for individual tree length unit, and $ a $ for the spatial unit (a length if the forest is one-dimensional, and an area, if it is a two-dimensional forest).} \label{tab::notations}
	\begin{tabular}{@{}cll@{}}
		\toprule
		\textbf{Symbol} & \multicolumn{1}{c}{\textbf{Definition}} & \multicolumn{1}{c}{\textbf{Unit}} \\
		$ \A $ & Area of the cross-section of the crown &  $ a $ \\
		$ \text{age}_{\max} $ & Maximum age from \citet{Burns1990, Burns1990a} & $ T $ \\
		$ \text{dbh}_{\max} $ & Maximum dbh from \citet{Burns1990, Burns1990a} & $ \ell $ \\
		$ \F $ & Number of seeds per tree's crown area per time & $ a^{-1} T^{-1} $ \\
		$ F $ & Effective fecundity function, \ie number of germinating seeds & $ T^{-1} $ \\
		$ F_{\text{Purves}} $ & Effective fecundity function (D. Purves, 2008) & $ T^{-1} $ \\
		$ G $ & Growth of individuals & $ \ell T^{-1} $ \\
		$ \mu $ & Mortality rate & $ T^{-1} $ \\
		$ N $ & `Density' of trees & $ \ell^{-1} a^{-1} $ \\
		$ \Omega $ & Landscape or expert map, $ \Omega \subseteq \mathds{R}^2 $ & $ a $ \\
		$ \varphi $ & Ratio of $ \Delta \text{AIC}_c $ for models with $ \text{VIF} < 20 $ & - \\
		$ R_0 $ & Net population growth rate & - \\
		$ \rho_0 $ & Net population growth rate using $ F_{\text{Purves}} $ and $ s_{\infty}^{x} $ & - \\
		$ \tilde \rho_0 $ & Standardised $ \rho_0 $ & - \\
		$ s $ & Size of individuals (either dbh or height) & $ \ell $ \\
		$ \s $ & Size threshold that separates the forest into two strata & $ \ell $ \\
		$ \s_c $ & Constant size threshold, value set to 0 m or 10 m for the maps & $ \ell $ \\
		$ s_{\max}^{x} $ & Maximal dbh that a tree would have at location $ x $ without competition & $ \ell $ \\
		$ s_{\infty}^{x} $ & Upper bound of integration in $ \rho_0 $ formula & $ \ell $ \\
		$ t $ & Time & $ T $ \\
		$ x $ & Space variable, $ x \in \Omega $ & - \\
		\bottomrule
	\end{tabular}
\end{table}

\subsection{Net reproduction rate $ R_0 $}
Henceforth, the height $ \s $ that separates shaded trees from sun-exposed trees is referred to as `competition'. We used the method of characteristics to calculate the net reproduction rate at a location $ x $ within a landscape $ \Omega $ as a function of a constant competition $ \s_c $. This mathematical technique is used to solve certain partial-differential equations, as commonly used in transport equations. For example, equation \eqref{eq::dynamics_x} describes the advection of trees growing at a non-constant speed $ G $ along the size axis (either height or diameter). Characteristics allow us to follow individuals throughout their lifespan, \ie they represent the trajectories of individuals in the time-size plane (fig. \ref{fig::chara} for an example). The derivation of $ R_0 $ is detailed in Supporting Information \ref{app::calc_R0::sec::R0}; it should be remembered three underlying assumptions that were made for this calculation: (\textit{i}) $ \s $ is considered fixed and known at a value $ \s_c $; (\textit{ii}) only trees larger than $ \s_c $ can reproduce \citep{Strigul2008}; and (\textit{iii}) dispersal is limited to the patch, as stated by equation \eqref{eq::recruitment_x}. Subsequently, the net reproduction rate in a patch $ x $ is consequently:
\begin{equation} \label{eq::R0sol}
	R_0 (x, \s_c) = \exp \left[-\int_0^{\s_c}\frac{\mu(s, \s_c, x)}{G(s, \s_c, x)} \, ds \right] \times \int_{\s_c}^{\infty} \frac{F(s, \s_c)}{G(s, \s_c, x)} \exp \left[-\int_{\s_c}^{s} \frac{\mu(\sigma, \s_c, x)}{G(\sigma, \s_c, x)} \, d\sigma \right] \, ds
\end{equation}
Equation \eqref{eq::R0sol} can be divided into two biological processes:
\begin{itemize}
	\item $ \exp\left[-\int_0^{\s_c}\frac{\mu(s, \s_c, x)}{G(s, \s_c, x)} \, ds \right] $ is the proportion of individuals that survive up to the canopy of height $ \s_c $ in plot $ x $,
	\item $ \int_{\s_c}^{\infty} \frac{F(s, \s_c)}{G(s, \s_c, x)} \exp \left[-\int_{\s_c}^{s} \frac{\mu(\sigma, \s_c, x)}{G(\sigma, \s_c, x)} \, d\sigma \right] \, ds $ is the expected production of offspring of an individual that is located in $ x $, during its lifespan. It has two subterms:
	\begin{itemize} % [label=$ \circ $]
		\item $ \frac{F(s, \s_c)}{G(s, \s_c, x)} $ is the number of offspring per unit time from individuals that grow at speed $ G $,
		\item $ \exp \left[-\int_{\s_c}^{s} \frac{\mu(\sigma, \s_c, x)}{G(\sigma, \s_c, x)} \, d\sigma \right] $ is the survivorship of trees of size $ s $. Since $ \mu $ and $ G $ are both positive functions, survivorship is a decreasing function of $ s $.
	\end{itemize}
\end{itemize}

From equation \eqref{eq::R0sol}, we understand that the reproduction rate $ R_0 $ can be strengthened by reducing the competition $ \s_c $, by accelerating the average understorey growth $ G(s < \s_c) $ or diminishing the average mortality rate $ \mu $, or by enhancing the fecundity $ F $. The mathematical proof of these three mechanisms is found in the Supporting Information \ref{app::calc_R0::sec::3asser}. The formula \eqref{eq::R0sol} is a generalisation of the $ R_0 $ that was derived by Purves and is now valid for any individual growth, mortality, or fecundity function. The same assertions can be drawn from this study (Supporting Information \ref{app::purves2009}); however, due to the complexity of the computations we cannot assert that an averagely faster overstorey growth rate leads to an increase in $ R_0 $. For the same reason, we cannot calculate the value of $ \s_c $ for a population at equilibrium (\ie when $ R_0 = 1 $), except when the demographic rates are easier to compute, as step functions \citep[Supporting Information \ref{app::purves2009} for the proof]{Purves2009}. The net reproductive rate $ R_0 $ derived here is a heuristic tool that relates the individual tree performance under a competition constraint $ \s_c $ to the population performance and is thus the key to addressing our second objective.

\subsection{Data}
We parameterised the demographic functions $ G $ (individual growth) and $ \mu $ (mortality) using data that were obtained from permanent sample plots of the Forest Inventory and Analysis (FIA, USDA Forest Service), the Ministère des Forêts, de la Faune et des Parcs du Québec, the Ministry of Natural Resources and Forestry of Ontario, Ministry of Natural Resources of New Brunswick, and the forest products company Domtar (Fig. \ref{fig::mapDatabase} for a map of the data). After removing plots that experienced fire or logging, there were \num{7704442} individual measurements (106 species distributed among \num{132240} plots). A record consists of the tree identity, the species, the year at which the individual has been measured, the \textit{dbh}, the latitude and longitude of the plot, and the tree's status (alive or dead). Measurements occurred between 1963 and 2010, and frequencies of measurement range from $ \sfrac{1}{1} \text{ yr}^{-1} $ to $ \sfrac{1}{40} \text{ yr}^{-1} $ (with $ 96 \, \% $ of the data between $ \sfrac{1}{3} \text{ yr}^{-1} $ to $ \sfrac{1}{15} \text{ yr}^{-1} $). Both radial growth and mortality are highly variable across the distributions of the individuals (Supporting Information \ref{app::database}).

Climate data were extracted for each plot using \textsc{anusplin} software \citep{McKenney2011} based upon the latitude and longitude coordinates of the permanent plots. Note that for privacy reasons, the FIA offset plot locations up to $ 1.6 $ km \citep{Gray2012} which might therefore imply mismatches between the real climate of the plot and the climate that we assigned. We selected 19 climatic variables (Supporting Information \ref{app::database}, Table \ref{tab::bioclim}) covering the period 1958-2010 with a spatial resolution of 60 arc seconds ($ \approx 2 \, \text{km}^2 $). To account for climate variability prior to each tree measurement, we averaged each temperature and precipitation variable over a period of 5 years using a moving average (5 years is the most frequent interval among measurements that are included in the database with $ 38.8 \, \% $ of observations).

We calculated individual tree height and crown area from allometric functions and parameters that are provided in \citet{Purves2007}. We considered the dbh of all trees, although most of Canadian inventories start with individuals having diameters greater than $ 100 $ mm, while USA inventories start at $ 127 $ mm (5 inches). Some trees have been recorded as dead and then alive; we considered the last living state being true and ignored `resurrection' events in mortality estimation. By definition, growth and mortality rates require at least two measurements from an individual tree. After calculating the threshold $ \s $ in each plot, single-measured trees were removed from the dataset. For the growth analysis, we eliminated dead trees and individuals that have either a non-positive dbh increment or a radial growth greater than 25mm per year. We parameterised the model for the 14 most abundant species in north-eastern North America (Table \ref{tab::species}), but we considered all of the 106 species in the database for the computation of $ \s $. The 14-species dataset that is used in this paper contains \num{69954} plots (75 \% in the USA, and 25 \% in Canada), for a total of \num{3816854} individual measurements. \textit{Abies balsamea} is the most measured species with \num{822265} individual measurements, whereas \textit{Tsuga canadensis} is the least frequently measured with \num{66008} individual measurements. The climatic and geographical ranges of each species can be found in Table \ref{tab::spaceRange} and Fig. \ref{fig::speciesClimRange} in Supporting Information \ref{app::database}.

%%%% RESTART HERE -----------------------------------------  L 711 in Word
\subsection{Parameterisation of demographic rate functions}
We used linear mixed models to parameterise the individual growth ($ G $) and mortality ($ \mu $) rates as a function of climate, canopy status (understorey if below $ \s $, overstorey otherwise), and size (dbh). For the fecundity, we used the functions and values from \citet{Purves2008}. We tested linear and quadratic functional forms for the temperature, precipitation, and dbh effects, which were all normalised using the standard score. To get an optimal climate in the quadratic case, it is necessary to have a negative (positive) slope in front of the squared climate variables for the growth (mortality). However, constraining the parameters would force the optimal climate to be within the data used for parameterisation, thus we did not set any constraint. The 19 variables of temperature and precipitation allow us to try different combinations, however we preselected certain climatic assemblages based on the interpretability of the models and the literature (Supporting Information \ref{app::glmm}). Each set of species-specific demographic parameters was estimated separately. We based the model comparison on information criteria and $ R^2 $. We ranked the models for each species, and selected the model that in average fits the best according to information criteria. Then, this model was imposed on the 14 species. The R scripts used to format the data and to estimate demographic parameters are available on github (see Data Availability Statement). 

\subsubsection{Growth}
For the individual growth rate, we assumed a lognormal distribution, hence $ \log(G) $ is normally distributed. We normalised the logarithm of growth
\[
	Y_{G} = \frac{\log(G) - \mathrm{E}[\log(G)]}{\mathrm{sd}[\log(G)]},
\]
and used the following model:
\begin{equation} \label{eq::glmm_growth}
\begin{split}
	E[Y_G^{i, j}] = & \beta_{j, (x, t)[i]}^{(0)} + \beta_j^{(1)}  \text{canopy status} \, + \\
	& \left(\beta_j^{(2)} + b_j^{(2)} \text{canopy status} \right) T \, +
	\left(\beta_j^{(3)} + b_j^{(3)} \text{canopy status} \right) T^2 \, + \\
	& \left(\beta_j^{(4)} + b_j^{(4)} \text{canopy status} \right) P \, +
	\left(\beta_j^{(5)} + b_j^{(5)} \text{canopy status} \right) P^2 \, + \\
	& \left( \beta_j^{(6)} + b_j^{(6, 1)} T + b_j^{(6, 2)} T^2 +
		b_j^{(6, 3)} P + b_j^{(6, 4)} P^2 \right) \text{dbh} \, + \\
	& \left( \beta_j^{(7)} + b_j^{(7, 1)} T + b_j^{(7, 2)} T^2 +
		b_j^{(7, 3)} P + b_j^{(7, 4)} P^2 \right) \text{dbh}^2
\end{split}
\end{equation}
where $ E[Y_G] $ is the expected value of the normalised logarithm of growth, and $ T $ and $ P $ are the associated explanatory temperature and precipitation, respectively. The canopy status is a boolean (true for the canopy trees, and false otherwise). The indices $ i $ and $ j $ stand for individual and species respectively, and $ (x, t)[i] $ index denotes the group effects (plot $ x $ and year $ t $ of the $ i^{\text{th}} $ individual). The random effects (with a maximum of three effects: spatial, temporal, and plot-specific temporal respectively) are normally distributed:
\begin{align*}
	\beta_{j}^{(0, x)} &\sim \mathcal{N}(0, \bm{\sigma}_{x}^{\text{growth}}) \\
	\beta_{j}^{(0, t)} &\sim \mathcal{N}(0, \bm{\sigma}_{t}^{\text{growth}}) \\
	\beta_{j}^{(0, xt)} &\sim \mathcal{N}(0, \bm{\sigma}_{xt}^{\text{growth}})
\end{align*}
Mixed models allow us to group individuals by plots and years of the second measurement to consider spatial and temporal structures. The plot effect comprehend the variation driven by local factors such as soil condition and disturbance history, while the year (within plot identity) represents the temporal variation that is not included in climate. The $ \beta $s are the regression coefficients, and the $ b $s correspond to different variable interactions. Climate interacts with the crowding effect (canopy status) to account for climate response variation to the individual's neighbourhood. Lastly, as bigger trees may be favoured or disadvantaged by climate, dbh interacts with climatic variables $ T $ and $ P $. \\

We used the package \textit{lme4} \citep{lme4} to estimate the parameters ($ \beta $s and $ b $s). We tested sub-models of equation \eqref{eq::glmm_growth} to evaluate the impact of competition and climate on $ G $. We used a top-down strategy to select first the random effect structure, and secondly the fixed structure. We compared models with the Akaike Information Criteria ($ \text{AIC}_c $), but kept models with a maximum Variance Inflation Factor (VIF) lower than $ 20 $ to avoid correlations between variables \citep{Zuur2010}. We chose $ 20 $ because of one model for growth that have reasonable maximum VIFs (below 6) for 11 species over 14, and a VIF of 19 for \textit{Fagus grandifolia}. We judged that a VIF beyond $ 20 $ implies too much collinearity (F. Guillaume Blanchet, \textit{discussion}). We denote by $ \Delta \text{AIC}_c $ the difference between a model $ i $ and the model that have the smallest $ \text{AIC}_c $:
\[
	\Delta \text{AIC}_c = \text{AIC}_c^{(i)} - \text{AIC}_c^{(\min)}
\]
The best model has the lowest $ \text{AIC}_c $, or equivalently $ \Delta \text{AIC}_c = 0 $. The $ \Delta \text{AIC}_c $ are calculated unconstrainedly with respect to the best model, but, due to the VIF constraint, the selected model is not the model with $ \Delta \text{AIC}_c = 0 $. Therefore, using exclusively the models satisfying the constraint $ \text{VIF} < 20 $, we computed the common logarithm of the ratio between each model's $ \Delta \text{AIC}_c $ and the minimum $ \Delta \text{AIC}_c $:
\begin{equation} \label{eq::ratio}
	\varphi = \log_{10}\left[ \frac{\Delta \text{AIC}_c}{\min \left( \Delta \text{AIC}_c \right)} \right]
\end{equation}
This ratio represents, within the subset of models satisfying the constraint $ \text{VIF} < 20 $, how many times (in power of 10) is the best model compared to the other models. The common logarithm provides a convenient scale to compare the models, and the best constrained model has $ \varphi = \log_{10}(1) = 0 $.

\subsubsection{Mortality}
For the mortality rate, the response variable $ Y $ is a boolean describing the transition state between two records (true if there is a transition from alive to dead, and false if the individual stays alive). The observation of a mortality event depends on the survey interval $ \Delta t $ \citep{Lines2010}. To survive from $ t_0 $ to $ t_1 = t_0 + \Delta t $, an individual $ i $ must survive each year:
\[
	\Prob[\text{survival } i: \, t_0 \to t_1] = \big(1 - \Prob[\text{annual mortality } i]\big)^{\Delta t}
\]
where $ \Prob $ stands for probability. Thus, the probability of observing a mortality event within a span $ \Delta t $ is:
\[
	\Prob[\text{mortality } i: \, t_0 \to t_1] = 1 - \big(1 - \Prob[\text{annual mortality } i]\big)^{\Delta t}
\]

We assumed $ Y $ follows a binomial distribution and used the complementary log-log link function $ g $ to account for the time between two surveys (offset on the intercept, Supporting Information \ref{app::glmm} for a short study of the clog-log function):
\[
	g \big(E[Y_\mu^{i, j}] \big) = \log \left( -\log(1 - E[Y_\mu^{i, j}]) \right)
\]
We worked with the following model:
\begin{equation}\label{eq::glmm_mortality}
\begin{split}
	g \big(E[Y_\mu^{i, j}] \big) = & \beta_{j}^{(0)}  + \beta_j^{(1)}  \text{canopy status} +
		\text{offset}\big(\log(\Delta t_i) \big) \, + \\
	& \left(\beta_j^{(2)} + b_j^{(2)} \text{canopy status} \right) T \, +
	\left(\beta_j^{(3)} + b_j^{(3)} \text{canopy status} \right) T^2 \, + \\
	& \left(\beta_j^{(4)} + b_j^{(4)} \text{canopy status} \right) P \, +
	\left(\beta_j^{(5)} + b_j^{(5)} \text{canopy status} \right) P^2 \, + \\
	& \beta_j^{(6)} \text{dbh} \, + \beta_j^{(7)} \text{dbh}^2
\end{split}
\end{equation}
with the same notations to the growth model. We did not include any group effect, as some plots have only one record for certain years and species, while other plots have no dead trees recorded (problem also faced in \citet{Kunstler2019}). To minimise the uncertainty of death events and to have enough measures per species per time interval, we limited the dataset to measurements with $ \Delta t \in [5, 11] $ ($ 74.9 \% $ of the mortality database). Since the probability of transition from dead to dead is 1, we kept records up to the first death event only (or all the measurements if there were no death event). \\

For the mortality, none of the GLMMs from the package \textit{lme4} converged. We suspect that despite the amount of data, there is little information due to the rarity of tree mortality. We therefore used the package \textit{rstanarm} \citep{rstanarm} which provides GLMMs in a Bayesian framework, and removed the climate-dbh interactions. We used 4 chains and 3000 iterations for each chain. We kept the default priors of \textit{rstanarm}, that are Gaussian distributions for the regression coefficients $ \beta $s, and exponential distribution for the standard deviation. Parameter values used later in the analysis are the medians of the posterior distributions. We used the Widely Applicable Information Criteria, WAIC \citep[and reference therein]{Hooten2015} to chose the best model. WAIC is based on the posterior predictive distribution (the distribution to predict new data) and is valid for hierarchical models. However, it assumes that the data are independent given a set of parameters, which could be a problem for our spatial data. Similarly to radial growth, we compared the WAICs on the common logarithm scale:
\begin{equation} \label{eq::psi}
	\psi = \log_{10}(\Delta \text{WAIC} + 1)
\end{equation}
The best model has $ \Delta \text{WAIC} = 0 $, which implies $ \psi $ is also null for the best model. Although there are $ R^2 $ for Bayesian regression models, they cannot be compared: the explained variance can only be interpreted in the context of a single model \citep{Gelman2018}. Hence, we exclusively based our choice on WAIC. We checked the convergence of the selected model for all the parameters and species with the Gelman-Rubin statistic \citep[R-hat diagnostic]{Gelman1992}. Usually a chain with a R-hat larger than 1.05 is considered non-convergent; at convergence, R-hat should be 1.

\subsubsection{Fecundity}
The forest inventory data have few records of trees with a dbh smaller than 10 cm, thus we could not parameterise the fecundity function $ F(s, \s) $. Instead, we used the fecundity function defined and parameterised in \citet{Purves2008}:
\begin{align} \label{eq::fecundity_fct}
	F_{\text{Purves}}(s, \s) = \F \times \A(s, \s) && \F = 0.0071
\end{align}
where $ \F $ is the number of seeds produced per sun-exposed tree crown area per unit time (Tab. \ref{tab::notations}). In this particular case, we defined a new quantity $ \rho_0 $ as the net reproduction rate $ R_0 $ with the reproduction function $ F_{\text{Purves}} $. The notation $ R_0 $ is strictly reserved to the general case, where the fecundity function is not restricted to the function of Purves.

By definition, $ \rho_0 $ corresponds to the net reproduction rate when the fecundity function is independent of climate (\ie spatially constant). Once $ \rho_0 $ is calculated over the landscape $ \Omega $, we associate a convenient normalised quantity:
\begin{equation} \label{eq::rho0_scaled}
	\tilde \rho_0(x, \s) = \frac{\rho_0(x, \s) - \min_{x \in \Omega} \big(\rho_0(x, \s) \big)}{\max_{x \in \Omega} \big(\rho_0(x, \s) \big) - \min_{x \in \Omega} \big(\rho_0(x, \s) \big)}
\end{equation}
This cancels the value of $ \F $, which was a difficult parameter to estimate in \citet{Purves2008}, and bounds $ \tilde \rho_0 $ between $ 0 $ and $ 1 $.

\subsection{Occurrence probability}
We evaluated for each species the correlation between $ \tilde \rho_0 $ and the probability of occurrence $ P_{occ} $ derived from a random forest \citep[R package]{randomForest}. This is a way to transform discrete presence and absence data into probabilities which are continuous data. It has been shown that random forests perform well in predicting tree species distributions \citep{Prasad2006}. Using a continuous probability rather than the presence and absence data can be useful in the case where climatic conditions are favourable but species are absent because of stochasticity, and alternatively where species are present but should not occur.

We trained the algorithm with coordinates of sample plots data set where there is at least one species recorded. We associated a 0 to each couple of coordinates--species if there were no record for that species in $ (x, y) $ after 1996, and 1 otherwise. We set the random forest with $ 2000 $ trees, and $ 12 $ predictors (over 19) to be chosen at each tree node. More trees add precision and bound the generalisation error (that is, the true error of the population as opposed to the training error only), without overfitting the data \citep{Prasad2006}. The explanatory variables set contained 19 bioclimatic variables (Supporting Information \ref{app::database} for the variables and \ref{app::randomForest} for the equation). The performance of the random forest was evaluated with the $ R^2 $ from \citet{Tjur2009}, later denoted by $ R_{\text{Tjur}}^2 $.

\subsection{Species distribution maps}
We downloaded an expert range map $ \Omega_j $ \citep{Prasad2003, Little1971} for each species $ j $, and calculated $ \tilde \rho_0 $ for each location within $ \Omega_j $ for an averaged climate over 5 years (from 2006 to 2010) and a canopy height $ \s $ of either $ 0 $ m (no competition) or $ 10 $ m (the average of the distribution of $ \s $ of the database is $ 9.6 $ m). Hereafter, we drop the index $ j $ and only use $ \Omega $, although the expert range maps are species-specific.

We investigated within the bounding box of our data (orange rectangle in Fig. \ref{fig::mapDatabase}), how $ \tilde \rho_0 $ relates to the orthodromic distance from the closest edge of the distribution defined by \citet{Little1971}, and mapped the variations of $ \tilde \rho_0 $. In order to avoid spatial extrapolations of the demographic rates, we decided to limit our study to the region of parameterisation (for instance \textit{Betula papyrifera} goes up to Alaska). To evaluate in which direction $ \tilde \rho_0 $ increases, we computed its gradient for each pixel of $ \Omega $ using the algorithm of \citet{Ritter1987}. Then, we separated $ \Omega $ into two regions---northern and southern to the centroid of $ \Omega $---and we averaged the gradients for each region. If $ \tilde \rho_0 $ decreases towards range limits, then both gradients should point toward the centre of the distribution. That is to say, the gradient of the northern region should point southward, and the gradient of the southern region should point northward. \\

We must integrate up to infinite to compute $ \tilde \rho_0 $ (see equations \eqref{eq::R0sol} and \eqref{eq::rho0_scaled}), but we found numerically that the integral beyond a height of 45 meters is negligible (because $ \lim_{\text{s} \to \infty} G(s) = 0 $ and $ \mu > 0 $). To be closer to real maximum dbh and height of trees for a given location $ x \in \Omega $ (rather than arbitrarily setting a maximum height of 45 m), we determined the upper boundary $ s_{\infty}^{x} $ of the integral of $ \tilde \rho_0 $ in three steps:
\begin{enumerate}
	\item Get species-specific maximal age ($ \text{age}_{\max} $) and maximal dbh ($ \text{dbh}_{\max} $) from \citet{Burns1990, Burns1990a}. The data are in Tab. \ref{tab::database}, Supporting Information \ref{app::database}
	\item Compute which dbh a tree would have at the location $ x $ with the associated climate $ \text{clim}_x $ if it were spending all its life in the overstorey up to the maximal age (using ODE45 from Matlab):
	\[
		s_{\max}^{x} = \int_0^{\text{age}_{\max}} G^{\text{(overstorey)}}\big(s(t), \text{clim}_x \big) \, dt
	\]
	\item Set the local `infinite dbh':
	\begin{equation} \label{eq::s_inf}
		s_{\infty}^{x} = \min(\text{dbh}_{\max}, \text{s}_{\max}^{x})
	\end{equation}
	and replace $ \infty $ in equation \eqref{eq::R0sol} by $ s_{\infty}^{x} $. If $ s_{\infty} < \s_c $, which happen when trees cannot reach the canopy before reaching their maximal age, we set $ R_0 $ to 0.
\end{enumerate}

Therefore, equation \eqref{eq::rho0} combines the equations \eqref{eq::R0sol}, \eqref{eq::fecundity_fct}, and \eqref{eq::s_inf} to measure species performance:
\begin{equation} \label{eq::rho0}
	\rho_0(x, \s_c) = \exp \left[-\int_0^{\s_c}\frac{\mu(s, \s_c, x)}{G(s, \s_c, x)} \, ds \right] \times \int_{\s_c}^{s_{\infty}^{x}} \frac{\F \A(s, \s_c)}{G(s, \s_c, x)} \exp \left[-\int_{\s_c}^{s} \frac{\mu(\sigma, \s_c, x)}{G(\sigma, \s_c, x)} \, d\sigma \right] \, ds
\end{equation}
from which we can derive $ \tilde \rho_0 $ (using equation \eqref{eq::rho0_scaled}). Equation \eqref{eq::rho0} accounts for spatial differences in the maximum dbh that are not included in the radial growth model. Its major difference with \eqref{eq::R0sol} is the upper bound of the expected production of offsprings, which makes equation \eqref{eq::rho0} specific to this paper, while equation \eqref{eq::R0sol} is a generalisation of \citet{Purves2009}.
