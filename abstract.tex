
\begin{abstract}
\noindent \textbf{Aim} Dynamic range models are proposed to investigate species distribution and project range shift under climate change. They are based on the Hutchinsonian niche theory, specifying that the occurrence of a species in an environmental space should be limited to positions where the intrinsic growth rate is positive. Evaluating population growth rate is however difficult for physiologically structured populations, such as forest stands, due to size-induced individual variation in performance. Therefore, we still have a limited understanding of which aspect of tree demography contributes the most to their geographical range limit. Here, we develop and test a model for tree species distribution based on demography. \\ % 110 words

\noindent \textbf{Location} North America (57--124 \degree W, 26--52 \degree N). \\ % 3 words

\noindent \textbf{Time period} 1963--2010. \\ % 2 words

\noindent \textbf{Major taxa studied} 14 tree species. \\ % 6 words

\noindent \textbf{Methods} We develop a structured-population model based upon the McKendrick--von Foerster equations to represent forest dynamics. We then derive lifetime reproduction per individual $ R_0 $ in the absence of density-dependence. Using forest inventory data, we assess for each species how tree demography varies with climate. We test the model by comparing $ R_0 $ and a probability of occurrence within species ranges. \\ % 71 words

\noindent \textbf{Results} Our analysis shows that $ R_0 $ is a non-linear relationship of individual growth, mortality, and fecundity. We find that both growth and mortality rates vary across species distributions, yet climate explains little of the observed variation. Individual size and neighbourhood competition are the primary explanatory variables of tree demography. Finally, we find that $ R_0 $ weakly relates to occurrence probability, with no systematic decline in population growth rates towards the range limits. \\ % 73

\noindent \textbf{Main conclusions} The spatial and size-induced variations in tree growth and mortality does not explain range limits. We propose that range limits of trees could either be explained by variation in establishment or by processes happening at larger spatial scales.  \\ % 41 words

\noindent {\bf Keywords}: Climate, Competition, Demography, Population growth rate, Niche theory, Species distribution, Structured-population model, Upscaling

\end{abstract}
