% Please use a structured Abstract, not longer than 300 words, with the following headings: Aim, Location, Time period, Major taxa studied, Methods, Results, Main conclusions.
\begin{abstract}
\noindent \textbf{Aim} Dynamic range models are proposed to investigate species distribution and project range shifts under climate change. They are based on the Hutchinsonian niche theory, specifying that the occurrence of a species in an environmental space should be limited to positions where the intrinsic growth rate is positive. Evaluating population growth rate is however difficult for physiologically structured populations, such as forest stands, due to size-induced individual variation in performance. Therefore, we still have a limited understanding of which aspect of tree demography contributes the most to their geographical range limit. We develop an index of demographic performance and study its variation across a climatic gradient. Then after we investigate the relationship between the demographic performance index and species distribution.  \\ % 120 words

\noindent \textbf{Location} North America (57--124\degree W, 26--52\degree N). \\ % 5 words

\noindent \textbf{Time period} 1963--2010. \\ % 3 words

\noindent \textbf{Major taxa studied} 14 tree species. \\ % 6 words

\noindent \textbf{Methods} We represent forest dynamics with a structured-population model based upon the McKendrick--von Foerster equations. We then derive the lifetime reproduction per individual $ R_0 $ in the absence of density-dependence. Using forest inventory data, we assess for each species how tree demography varies with climate. We test the model by comparing $ R_0 $ and a probability of occurrence within species ranges. \\ % 59 words

\noindent \textbf{Results} We find that both growth and mortality rates vary across species distributions, yet climate explains little of the observed variation. Individual size and neighbourhood competition are the primary explanatory variables of tree demography. Finally, we find that $ R_0 $ weakly relates to occurrence probability, with no systematic decline in population growth rates towards the range limits. \\ % 56

\noindent \textbf{Main conclusions} The spatial and size-induced variations in tree growth and mortality does not explain range limits and are not enough to understand tree dynamics. We propose that phenomena perceived mostly at the metapopulation scale should also be considered. \\ % 39 words

\noindent {\bf Keywords}: Climate, Competition, Demography, Population growth rate, Niche theory, Species distribution, Structured-population model, Upscaling
\end{abstract}

% We propose that we should not rely exclusively on `the shoulders' of the demographic rates determined by local spatial processes to understand tree dynamics, but also on certain phenomena that can be perceived mostly at the metapopulation scale. \\ % 55 words
