
% Deleted references: Canham2016, GRUBB1977, Valladares2002, BarkerPlotkin2017,


%%%% Notes:
% The two major differences between our model and DRMs are that uncertainties of our model lie only at the demographic level but we account for pop-struct
% To compare with DRM, we have a two level hierarchical model
% How about stochastics PDEs? Do we converge to an IPM?

\section{Discussion}

Overall, our results demonstrate the extreme variability in the demographic rates and the difficulties to relate demography to species distribution. We showed that our capacity to relate the occurrence of a species $ P_{occ} $ to its performance $ \tilde \rho_0 $ varies greatly among species, from negative to positive correlations. This result has also been found with a simpler measure of tree growth \citep{McGill2012}. However, our method allows us to reckon explicitly with competition and demography. For most species, we found no support to the hypothesis of a declining species performance towards species range limits.

\subsection{Effect of climate on individual growth and mortality}
We found little correlation between individual's rates and the 2 km\textsuperscript{2} resolution climate across eastern North America, although we controlled for both individual size and position into the canopy. Even though tree growth and mortality may respond to climate, as often documented in dendrochronological studies (e.g. \citet{Aussenac2017}), they are highly unpredictable at continental scale. Short-term events like frosts, gusts, fires, or pests and pathogens outbreaks, may impact these rates more than average conditions, as does the micro-topography \citep{Loehle1996}. For biogeography studies, scale can be a central issue: the relevant spatio-temporal scales for individuals often are meters and minutes rather than kilometres and months \citep{Urban2016}, but this is beyond the scope of most current datasets for studies at continental extent (although microclimatic datasets start to appear \citep{Lembrechts2019}). On the other side, in biogeography we want to erase individual variability in order to extract the global behaviour of individuals, as does the Hutchinsonian niche theory when delimiting a species-specific N-dimension volume within the environmental space where the demographic rates allow a species to remain `indefinitely'. What seems to happen here, is that either we did not take enough environmental variables to describe the demography, or that even on a large extent, the disturbances and the micro-environment matters more than average climate conditions.

\subsection{Importance of competition}
Our findings highlight that radial growth and mortality are strongly influenced by the asymmetrical competition for light, which is an important species interaction for trees. We have found a relationship between the shade tolerance and the effect of light competition for both growth $ G $ and mortality $ \mu $. It suggests that not only understorey tree survival, but also understorey tree growth could help to classify trees, although with caution \citep{Feng2018}. Notwithstanding that the response of growth to climate is mediated by the canopy status, we did not explore further into the climate--competition relationship. We think it would require a dedicated study, as counter-intuitive responses have already been observed for plants \citep{Holmgren2012}. \\

We found no signal 

\subsection{A demographic-based population performance index}

\subsection{Limitation of the study and research agenda}

\subsection{Recruitment niche}
There is little evidence that sapling and canopy tree survival play an important role in the distribution of 13 of our 14 tree species \citep[\textit{Picea mariana} is missing]{Canham2017}. In our study, demographic rates were rather strongly dependent on size, highlighting the importance of a size-structured population model. A size-structured model makes the individual growth and mortality rates closer to long-term demographic data (such as forest inventories), and considers the possible stage-dependency of the niche. However, it is difficult to cover the full range of stages and study their specific relationships to climate. Our model does not account for spatial variability in seed production, seed survival and germination, for which very little is known and documented. Our study also overlooked seedling growth and survival because there are no available data for individual trees under a certain diameter. We had therefore to extrapolate the early stages of individual growth and mortality. This ensemble of processes, from seed to seedling survival, refers to the recruitment niche, which is defined by the requirements that allow a seed to germinate and establish \citep{Valdez2019}. Different studies that either focus on a single species \citep[\textit{Acer saccharum}]{Solarik2016}, along a longitudinal gradient \citep{CLARK2011}, or a latitudinal gradient \citep {Boisvert-Marsh2019}, provide good evidences of the role of climate in shaping the recruitment niche. More specifically, seed production, not tree growth and mortality, was found to be the most responsive to spring temperature and summer drought \citep[11 sites in the Appalachians, Piedmont, and North Carolina]{CLARK2011}. Seedling recruitment and survival are also expected to be strongly correlated to species distribution \citep{Vanderwel2013, Valdez2019, Solarik2019}. Soil properties and pathogens might constrain even more the regeneration of trees than climate \citep{Brown2014}, which corroborates the need for finer resolution data. Therefore, a major improvement of our study would be to better describe the spatial variation of the recruitment niche (\ie the dependence of effective fecundity to climate). We would need more data on the juvenile stage to avoid extrapolating demographic rates estimated mostly on adult trees ($ dbh \geqslant 100 $ mm) to saplings. \\


The effect of dbh and size-induced competition on individual growth and mortality rates illustrate the potential of using structured-population models in population dynamics.


If this variability where unimportant,

variability in the demographic rates and their

%%%%%%%%%%%%%%%%%%%%%%%%%%%%%%%%%%%%%%%%%%%%%%%%%%%%%%%%%%%%%%%%%%%%%%%%%%%%%%%
%%%%%%%%%%%%%%%			Old version bellow, untouched			%%%%%%%%%%%%%%%
%%%%%%%%%%%%%%%%%%%%%%%%%%%%%%%%%%%%%%%%%%%%%%%%%%%%%%%%%%%%%%%%%%%%%%%%%%%%%%%


\subsection{An unruled occurrence}

In this article, we developed a model to investigate how climate and
competition determine individual tree performance, and how this performance
relates to species distribution. Individual tree performance was measured with
three demographic rates, namely radial growth, mortality, and effective
fecundity. We estimated the 2 former rates, and took a fecundity function independent of climate and species from the
literature. Tree growth and mortality models accounted for climate using
polynomials of order 2. Some tree species had counter-intuitive positive
coefficients with respect to the squared temperature or precipitation,
although near zero; it indicates that we were not able to detect an optimal
climatic condition within our data for these species. In general, we found that climate
variation is a poor predictor of growth and mortality, and that competition
and forest structure dominate the climate effects on these demographic rates.
The demographic estimates allow us to measure the population performance $
\tilde \rho_0 $ as the sum of expected contributions of each individual. We
found for most species, no support to the hypothesis of a declining species
performance towards species range limits. There was also no rule relating $
\tilde \rho_0 $ to the probability of occurrence.

The measure $ \tilde \rho_0
$ formally integrates three biological mechanisms in the modelling of
biogeographical distributions, namely demography, environment, and species
interactions. These three mechanisms commonly shape species responses to
climate change, along with dispersal, evolution and physiology
\citep{Urban2016}. Yet, adding demography, environment, and species
interactions comes with trade-offs: more detailed models are data-intensive,
might require specific data, and increase the parameters estimation
complexity. Our approach was computationally challenging, as it combines
\num{3816854} tree measurements to climate data and computed competition. \\

The lack of a clear relationship between climate, growth and mortality was
surprising given the geographical extent of the data. Although tree growth and
mortality may respond to climate, as often documented in dendrochronological
studies (e.g. \citet{Aussenac2017}), they are highly unpredictable at
continental scale. Short-term events like frosts, gusts, fires, or pests and
pathogens outbreaks, may impact these rates more than average conditions, as
does the micro-topography \citep{Loehle1996}. Hence, coarse-scale data are in
general not sufficient to disentangle climate effect on the demography from
the turmoil engendered by disturbances and the micro-environment, as supported
by \citet[and references therein]{Lembrechts2019}. For individuals, the
relevant spatio-temporal scales often are meters and minutes rather than
kilometres and months \citep{Urban2016}, but this is beyond the scope of most
current datasets for studies at continental extent (although microclimatic
datasets start to appear \citep{Lembrechts2019}). The absence of climatic
signal in the demographic data has also been found in \citet{Subedi2013},
while it was found stronger on the Tibetan plateau \citep{Liang2010}. For
instance, even though \citeauthor{Subedi2013}'s climate variables are significant, their root mean
square error of radial growth decreases at most about \num{1.6}\%. One way to
overcome hazy species-specific demographic responses to climate has been to
group species into plant functional types \citep{Vanderwel2013}, at the cost
of loosing the species specificness. \\

There is little evidence that sapling and canopy tree survival play an
important role in the distribution of 13 of our 14 tree species
\citep[\textit{Picea mariana} is missing]{Canham2017}. In our study,
demographic rates were rather strongly dependent on size, highlighting the
importance of a size-structured population model. A size-structured model
makes the individual growth and mortality rates closer to long-term
demographic data (such as forest inventories), and considers the possible
stage-dependency of the niche. However, it is difficult to cover the full
range of stages and study their specific relationships to climate. Our model
does not account for spatial variability in seed production, seed survival and
germination, for which very little is known and documented. Our study also
overlooked seedling growth and survival because there are no available
data for individual trees under a certain diameter. We had therefore to
extrapolate the early stages of individual growth and mortality. This ensemble
of processes, from seed to seedling survival, refers to the recruitment niche,
which is defined by the requirements that allow a seed to germinate and
establish \citep{Valdez2019}. Different studies that either focus on a single
species \citep[\textit{Acer saccharum}]{Solarik2016}, along a longitudinal
gradient \citep{CLARK2011}, or a latitudinal gradient \citep {Boisvert-Marsh2019}, provide good evidences of the role of climate in shaping the
recruitment niche. More specifically, seed production, not tree growth and
mortality, was found to be the most responsive to spring temperature and
summer drought \citep[11 sites in the Appalachians, Piedmont, and North Carolina]{CLARK2011}. Seedling recruitment and survival are also expected to
be strongly correlated to species distribution \citep{Vanderwel2013, Valdez2019, Solarik2019}. Soil properties and pathogens might constrain even
more the regeneration of trees than climate \citep{Brown2014}, which
corroborates the need for finer resolution data. Therefore, a major
improvement of our study would be to better describe the spatial variation of
the recruitment niche (\ie the dependence of effective fecundity to climate).
We would need more data on the juvenile stage to avoid extrapolating
demographic rates estimated mostly on adult trees ($ dbh \geqslant 100 $ mm)
to saplings. \\

We also found that radial growth and mortality are determined by the
asymmetrical competition for light, which is an important species interaction
for trees. We have found a relationship between the shade tolerance and the
effect of light competition for both growth $ G $ and mortality $ \mu $. It
suggests that not only understorey tree survival, but also understorey tree
growth could help to classify trees, although with caution \citep{Feng2018}.
Notwithstanding that the response of growth to climate is mediated by the
canopy status, we did not explore further into the climate--competition
relationship. We think it would require a dedicated study, as counter-intuitive responses have already been observed for plants
\citep{Holmgren2012}. \\

The net reproduction rate $ R_0 $ is a heuristic tool to summarise how
individual growth, mortality, and seed production together define species
persistence. We showed with equation \eqref{eq::R0sol} that individual growth
and mortality might contribute non-linearly to species range limits. More
specifically, a sensitivity analysis of $ R_0 $ and species shifts with
respect to the demography should emphasise which demographic parameter is the
most important in a context of climate change, and thus provide a future research agenda on these demographic rates.
It is however too early to use our $ R_0 $ in assisted migration debates and forest management.
Currently, we found 4 levers on which we can act to change $ R_0 $: the
competition, the average understorey growth, the average mortality rate, and
the fecundity function. We could not prove the role of the overstorey growth rate, however it increases $ R_0 $ when the overstorey survival $ \exp \left[ -\int_{\s}^{s} \sfrac{\mu}{G} \, d \sigma \right] $ is more beneficial than the decrease in the number of recruit per unit of dbh $ \sfrac{F}{G} $. \\

% Depending on the desired effect, such as reducing the
% population growth rate of an invasive species, or increasing $ R_0 $ for
% commercial or endangered species, different actions affecting the demography
% at different life stages can be taken. It is the aforementioned sensitivity
% analysis that will provide on which demographic rate, and at what life stage
% we should act. \\

Our analysis was focused on local dynamics and we have put aside the impact of
spatial dependence of species populations, which can be maintained by nearby
sources or that are more vulnerable when isolated \citep{Pulliam2000}. We made
the hypothesis of a local dispersion, using a Dirac distribution (\ie the
dispersion does not appear in our equation) for the sake of tractability,
albeit we developed the model for any dispersal kernel $ \K $. Thus, $ R_0 $ is a quantity derived only from demographic processes. For short-term projections (\ie as long as dispersal is negligible), we expect $ R_0 $ to be close to any measure also accounting for dispersal, as most of the seed dispersion is local \citep{Nathan2000}.

% Our first
% approximation shall not be a problem when working at short temporal scale with
% slow-growing species like trees. Indeed, tree population expansion rates range
% from $ 150 \, m/yr $ to $ 500 \, m/yr $ in the early Holocene, which was a
% fast migration period \citep{Clark1998a}. Moreover, most of the seed
% dispersion is local, long-distance dispersal being rather a rare event
% \citep{Nathan2000}. Thus, our results should be interpreted as short-term
% projections. \\

We exclusively considered individual variation that has no effect on
evolution. Evolution would be difficult to implement explicitly, but could
implicitly be introduced via local adaptation in the parameterisation of
demography. Nevertheless, it would requires additional data to separately
estimate vital rate responses to climate drivers within multiple populations
\citep{Peterson2019}. We assumed the species are genetically homogeneous, but
local adaptations affect fitness components of germination and establishment
\citep{Alberto2013}, and by extension the population growth rate. For
instance, it has been shown that \textit{Acer saccharum}'s seeds demonstrate
local adaptation along a latitudinal gradient \citep{Solarik2016} or that tree
growth and dormancy cycles synchronise with local seasonal temperature
\citep{Aitken2008}. Hence, some underlying variation in demography may be
hidden in the spatial random effect of our demographic models and are not
explicitly considered in $ R_0 $. Similarly, our study does not include
physiology. We do not think adding physiological processes would add precision
in the niche theory, although it would shed light on the mechanisms governing
the demography. Indeed, physiology is very important to understand the
demographic rates in extreme environments \citep{Chaves2009} and would be
insightful to better understand range limits. Local adaptation and phenotypic
plasticity could also be the reason of the weak individual--climate
interactions \citep[and references therein]{Miner2005}, as populations would
locally define their optimal climate conditions.

\subsection{Implications for modelling species distribution}
Integrated models
such as ours, are particularly well designed to understand the role of each demographic component in tree
species distributions and migration rates, and to understand errors propagation. We showed that our capacity to
relate the occurrence of a species $ P_{occ} $ to its performance $ \tilde
\rho_0 $ varies greatly among species, from negative to positive correlations.
This result has also been found with a simpler measure of tree growth
\citep{McGill2012}. However, our method allows us to reckon explicitly with
competition and demography. With competition, the correlation between $
P_{occ} $ and $ \tilde \rho_0 $ seems even weaker. This suggests that climate
modelling distribution of trees should be interpreted carefully when species
interactions are ignored. \\

Demographic approaches have been proved to be more biologically realistic than
abundance-based analysis to study species--habitat relations \citep{Bin2016}.
However, our study stresses that local demography is not enough to explain
distribution. Therefore, climatic niche should be used circumspectly as the
underlying processes of species occurrence remain unclear. Regional scale
processes such as major disturbances, intervene with and might prevail stand
processes in the control of trees population growth. The study of
\citet{Talluto2017a} is for instance very successful at predicting species
distributions using metapopulation theory. The fact that important mechanisms
operate on individuals does not mean that only individual-level data provide
insight \citep{Clark2011}. We proved that actually, individual processes in
the way we estimated them, play a minor role. Hence, a good way to infer how
species distributions emerge from individuals and spatial structures could be
a structured-population model combined with the metapopulation framework. In this way, the demographic rates would not rely exclusively on
‘the shoulders’ of local spatial processes, but also on certain phenomena that can be perceived mostly at the metapopulation scale, such as fire dynamics.
