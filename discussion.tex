
% Deleted references: Canham2016, GRUBB1977, Valladares2002, BarkerPlotkin2017,

\section{Discussion}
We developed a model to investigate how climate and competition determine continental scale variation in tree demography. We modelled variation in radial growth and mortality and, combined with values from the literature for the effective fecundity, we derived population growth rate $ R_0 $ for size-structured populations. We renamed $ R_0 $ into $ \rho_0 $ when using \citet{Purves2008}'s fecundity function and scaled it (\ie $ \tilde \rho_0 $) to emphasize spatial variation in demography under constant fecundity. We then correlated $ \tilde \rho_0 $ to occurrence probabilities with and without competition, and found significant variation among species and competition levels, from negative to positive. Our method advance previous analyses of ontogenic growth \citep{McGill2012, Thuiller2014} with explicit representation of the complex history of forest stands (\ie tree cohorts), the abiotic environment, and species interactions. These three mechanisms commonly shape species responses to climate change, along with dispersal, evolution and physiology \citep{Urban2016}. Yet, adding demography, environment, and species interactions for size-structured population models comes with trade-offs: more detailed models are data-intensive, might require specific information, and increase the parameters estimation complexity. Our approach was computationally challenging, as it combines \num{3816854} tree measurements to climate data and competition (computed using \num{7704442} measures). Finally, we tested whether species performance declines towards species range limits, accounting for competition, and found no support for this hypothesis for most species. Overall, our results demonstrate an extreme variability in growth and mortality rates and the difficulties to relate tree demography to species distribution.

\subsection{From climate to occurrence through demography}
\subsubsection{Climate effect on radial growht and mortality}
Both individual growth and mortality were highly variable along the vast climatic gradient we investigated. There is a long tradition in dendrochonological studies (e.g. \citet{Aussenac2017}) to relate variations in annual growth of individual trees to inter annual variations in climate. It appears however that the effect of climate on demography is more difficult to reveal when comparing the average performance of individual trees across a large biogeographic area, and in particular when we integrate growth and mortality over the entire life cycle of a forest stand. Many other factors may condition forest dynamics beyond the effect of climate \citep{Zhang2015}. Tree demography is likely a high-dimensional process \citep{Clark2011} affected by several individual level constraints such as genetics, soil properties, micro-topography, forest composition, pests, tree and stand history. Tree mortality is also a slow and cumulative process that may be difficult to represent with punctual information such as diameter, average climate and neighbourhood composition. Models such as the ones we investigated could not properly account for external causes of death such as physical damage \citep[uprooting, stem breakage, crushing by other falling trees]{Larson2010} or dieback. Our observation that plot random effects are large relative to the effects of climatic variables is good evidence that this individual/site level variation is driving much of the uncertainty in tree demography across the climatic gradient. Including all of these extra variables in growth and mortality models (provided we have proper information and functions to represent their effects) would certainly help reducing the uncertainty in the functions relating demography to climate. That said, individual conditions such as past history and micro-site properties are highly variable and unpredictable in nature, which means that even though we could improve further growth and mortality models by trimming residual variation, this uncertainty in some of the conditioning variables would propagate and still make demography highly variable across climatic gradients. In other words, the stochastic variation in individual conditions is simply overwhelming the deterministic effect that climate has on growth and mortality and therefore precluding any significant impact of demography on species distribution.

\subsubsection{Competition effect on radial growht and mortality}
Despite the substantial variability we observed, we found that radial growth and mortality are strongly influenced by the asymmetrical competition for light. Although not surprising, this observation confirms that our simplified representation of competition for light may be sufficient to model forest dynamics. Yet, competition requires more investigation to understand the lack of an effect on species distributions. We have found a relationship between shade tolerance and the effect of light competition for both growth ($ G $) and mortality ($ \mu $). It suggests that not only understorey tree survival, but also understorey tree growth could help to quantify tree species competitive ability, although with caution \citep{Feng2018}. Notwithstanding that the response of growth to climate is mediated by the canopy status, we did not explore further into the interaction between climate and light competition for our 14 species. We already know that the growth response of these species to adverse climate conditions is buffered by neighbourhood interspecific competition \citep{Aussenac2019}. Understanding the effect of light competition on tree demography is therefore of primary importance and requires dedicated studies, as light and water availability act non-linearly on plant performance \citep{Holmgren2012}. The domination of size and competition over climate in our models demonstrates the importance of considering population structure in forest dynamics. Our derivation of an integrated measure of performance across the life cycle is therefore needed to properly investigate the variation in demography across large areas. \\

In absence of competition, we found that all the species exhibit positive population growth rates across most of their range (Table S5.3). When adding competition (canopy height $ \s_c = 10 $ m), the boreal species had a substantial reduction in the proportion of the range with positive growth, from $ 0.01 \% $ (\textit{Thuja occidentalis}) to $ 99 \% $ (\textit{Pinus banksiana}). It is possible that we underestimated $ \tilde \rho_0 $ up north because we set a standard canopy height of $ 10 $ m across regions, and we did not consider variation in the allometric relationships we used. Among the other species, \textit{Populus tremuloides} lost $ 66 \% $ of suitable locations within its range $ \Omega $ with the addition of competition, while \textit{Acer rubrum} lost $ 28 \% $. \textit{Populus tremuloides} is classified as very shade-intolerant \citep{Burns1990a} and therefore the decline in growth due to competition could be expected because recruitment does not occur continuously in the understorey. Disturbances, which we did not include in our model, also play a major role in maintaining this pioneer species in Eastern Canada \citep{Nlungu-Kweta2017}. Population growth rates were positive for all the other species within their distribution, with and without competition, which support previous studies showing that species geographic distributions ($ \Omega $) are within species' ecological niches \citep{Lee-Yaw2016, Csergo2017}.

\subsubsection{Relationship between demography and distribution}
The net reproduction rate $ \rho_0 $ (and equivalently $ \tilde \rho_0 $) is a heuristic tool to summarise how individual growth, mortality, and seed production together define species persistence. We found four levers on which we can act to change $ \rho_0 $: the competition, the average understorey growth, the average mortality rate, and the fecundity function. The combination of these four levers can represent the growth-survival trade-off (\eg the ratio of mortality over radial growth $ \sfrac{\mu}{G} $ appearing in equation \eqref{eq::rho0}) and the stature-recruitment trade-off (distinguishes long-lived pioneers from short-lived breeders). These two trade-offs have been shown to be important to understand forest dynamics \citep[for tropical forest]{Ruger2020}. \\

We found little support for the hypothesis that tree species should distribute where they perform the best: there are very low positive correlations between $ \tilde \rho_0 $ and occurrence probabilities. We compared the effects of using random forests rather than presence/absence data, but found no difference in the trends (Supporting Information Appendix S6). Given the high $ R_{\text{Tjur}}^{2} $, we cannot attribute the lack of correlation to the random forest algorithm used to smooth the occurrence probabilities for the different tree species across their distribution. Our study adds to a growing literature body on this subject \citep[and references therein]{Holt2020}. For instance, similar lack of correlations have been found for European trees \citep[using matrix projection models]{Csergo2017} and for western North American species along water gradients \citep{Bohner2019}. Local interactions are hypothesised precluding SDMs to predict population growth rate when fitted with macroclimate \citep{Csergo2017}. Our results show that even though competition for light is important, there might be other mechanisms underlying abundance or population growth rates. For example, certain species undergone negative density dependence \citep[for rare species]{Yenni2012}, while certain common species are limited by plant-soil feedbacks \citep[\textit{Acer saccharum}]{Solarik2019}. \\

Competition influenced correlations of the population growth rate $ \tilde \rho_0 $ with probability of occurrence ($ P_{occ} $) and with distance to the closest edge (Figs. \ref{fig::3correls}; Supporting Information Appendix S6). For all the boreal species but \textit{Abies balsamea}, competition attenuated the signal between $ P_{occ} $ and $ \tilde \rho_0 $ by making the correlation closer to 0. As already discussed above, a canopy height of $ 10 $ m might be too high above a certain latitude. For the same reason the correlations between $ \tilde \rho_0 $ and distance to closest edge might have been blurred by competition. For instance, for \textit{Betula papyrifera, Picea mariana} and \textit{Thuja occidentalis}, we found support for declining performance toward species range edge exclusively in open canopies. Hence, competition might have an important role at the northern border of certain species through its influence on demography. The effect of competition at range edges have recently been investigated for European trees, with a different outcome: although competition was a strong determinant of vital rates, its effect was not stronger at the edge than at the centre of the distribution \citep{Kunstler2019}. In spite of this, the authors also found a weak support of declining performance at the range edge (the Abundant Centre Hypothesis). We rather found for most species a monotonically increasing $ \tilde \rho_0 $ either toward their northern or southern boundary but not toward their centre (Fig. \ref{fig::grad_cols} and Supporting Information S5).


\subsection{Tree demography beyond growth and mortality}
\subsubsection{Fecundity}
It is difficult to cover the full range of trees' life cycle and study their specific relationships to climate. Our model does not account for spatial variability in seed production, seed survival and germination, for which very little is known and documented. We used $ \tilde \rho_0 $ instead of $ \rho_0 $, in order to tease apart the fecundity term. This measure can be compared directly between species to observe trends of population performance in a landscape when species' fecundity is kept constant. However, while the threshold value allowing species persistence is always 1 when using unscaled population growth rates $ \rho_0 $, it becomes species-specific when using $ \tilde \rho_0 $. Our study also overlooked seedling growth and survival because there are no available data for individual trees under a certain diameter. The ensemble of processes, from seed to seedling survival, refers to the recruitment niche, which defines the requirements that allow a seed to germinate and establish \citep{Valdez2019}. Different studies that either focus on a single species \citep[\textit{Acer saccharum}]{Solarik2016}, along a longitudinal gradient \citep{CLARK2011}, or a latitudinal gradient \citep{Boisvert-Marsh2019}, provide good evidences of the role of climate in shaping the recruitment niche. More specifically, seed production, not tree growth and mortality, was found to be the most responsive to spring temperature and summer drought \citep[11 sites in the Appalachians, Piedmont, and North Carolina]{CLARK2011}. Soil properties and pathogens were found to constrain the regeneration of trees even more than climate \citep[Mont M\'{e}gantic, Qu\'{e}bec]{Brown2014}. These studies corroborate the plethora of mechanisms underlying population growth rates, and underline our need for more data on the juvenile stage, at least to avoid extrapolating demographic rates estimated mostly on adult trees ($ dbh \geqslant 100 $ mm) to saplings. A sensitivity analysis of $ \tilde \rho_0 $ with respect to the three vital rates is necessary to go further and distinguish which demographic parameter is the most important in a context of climate change, for which species, at which stage, and at which part of the range.

\subsubsection{Dispersal}
We made the hypothesis of a local dispersal, using a Dirac distribution (\ie the dispersion does not appear in our equation) for the sake of tractability. Therefore, $ \tilde \rho_0 $ is a quantity derived only from demographic processes, neglecting potential impacts of source and sink dynamics on distribution. The role of dispersal in tree distribution is controversial: certain studies report that long-distance dispersal determines the migration rate of trees and can explain some species shifts \citep[and references therein]{Nathan2002}, while other studies argue that microrefugia (small patches out of the `main' distribution) play a major role in colonisation \citep[and references therein]{Feurdean2013}. These hypotheses could be investigated numerically with our model by developing an algorithm adapted to transport equations like \eqref{eq::dynamics_x}. However, retrieving analytical results would requires simpler radial growth and mortality functions.

\subsection{Looking forward}
Even though $ \rho_0 $ has nice theoretical properties (Supporting Information Appendix S1), it is too early to use it in assisted migration debates and forest management. Nevertheless, we note that some species presented low $ \rho_0 $ in the south of their distribution and northward increase, which might be a sign of future mismatches between their niche and their distribution. This is particularly true for \textit{Acer rubrum} which is projected to have extinction debts in the south \citep{Talluto2017a}. Species that are poorly dispersed and highly persistent might be more subject to niche-distribution mismatches \citep[study on shrubs of the Proteaceae family in South Africa, not on trees]{Pagel2020}. It is, therefore, important to discern which scale is the best to understand how tree distribution emerges from forest dynamics. \\

Tree demography is a multidimensional process resulting from individual's characteristics (\eg genetics, history, size), and from local conditions entangled with regional processes. One way to acknowledge our incomplete understanding or the high variability of demography is to use stochasticity in modelling processes, at the price of losing the analytical population growth rate. Integral Projection Models (IPMs) are stage- or size-structured models, like ours, and are promising tools to model forest dynamics \citep{Vindenes2012, Kunstler2019, Merow2014}. IPMs can integrate stochasticity in the demography, which is a major feature as chance plays a significant role in lifetime reproductive success variance and, by extension, in $ R_0 $'s variance \citep{Snyder2016, Snyder2018}.

% Demographic approaches have been proved to be more biologically realistic than abundance-based analysis to study species-habitat relations \citep{Bin2016}.

\section{Conclusion}
Our study stresses that using climate through demographic rates is not enough to explain species distribution, even though we accounted for forest's structure and competition for light via an (over)simplified way. Therefore, climatic niche should be used circumspectly as the underlying processes of species occurrence remain unclear. If climate plays a minor role in tree demography, then the individual level is not the most relevant scale for predicting tree range limits. We proved that actually, individual processes in the way we estimated them, are little involved. Therefore, we should not rely exclusively on the shoulders of the demographic rates determined by local spatial processes to understand tree dynamics, but also on certain phenomena that can be perceived mostly at the metapopulation scale, such as fire dynamics. Our study underlines the need of mixing scales and using integrated population modelling \citep{Isaac2020}.
% , and we think combining Dynamic Range Models (DRMs) with Integral Projection Models is highly relevant.